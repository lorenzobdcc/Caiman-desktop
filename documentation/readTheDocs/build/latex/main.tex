%% Generated by Sphinx.
\def\sphinxdocclass{report}
\documentclass[a4paper,12pt,french]{sphinxmanual}
\ifdefined\pdfpxdimen
   \let\sphinxpxdimen\pdfpxdimen\else\newdimen\sphinxpxdimen
\fi \sphinxpxdimen=.75bp\relax
%% turn off hyperref patch of \index as sphinx.xdy xindy module takes care of
%% suitable \hyperpage mark-up, working around hyperref-xindy incompatibility
\PassOptionsToPackage{hyperindex=false}{hyperref}
%% memoir class requires extra handling
\makeatletter\@ifclassloaded{memoir}
{\ifdefined\memhyperindexfalse\memhyperindexfalse\fi}{}\makeatother

\PassOptionsToPackage{warn}{textcomp}

\catcode`^^^^00a0\active\protected\def^^^^00a0{\leavevmode\nobreak\ }
\usepackage{cmap}
\usepackage{fontspec}
\defaultfontfeatures[\rmfamily,\sffamily,\ttfamily]{}
\usepackage{amsmath,amssymb,amstext}
\usepackage{babel}



\setmainfont{FreeSerif}[
  Extension      = .otf,
  UprightFont    = *,
  ItalicFont     = *Italic,
  BoldFont       = *Bold,
  BoldItalicFont = *BoldItalic
]
\setsansfont{FreeSans}[
  Extension      = .otf,
  UprightFont    = *,
  ItalicFont     = *Oblique,
  BoldFont       = *Bold,
  BoldItalicFont = *BoldOblique,
]
\setmonofont{FreeMono}[
  Extension      = .otf,
  UprightFont    = *,
  ItalicFont     = *Oblique,
  BoldFont       = *Bold,
  BoldItalicFont = *BoldOblique,
]


\usepackage[Sonny]{fncychap}
\ChNameVar{\Large\normalfont\sffamily}
\ChTitleVar{\Large\normalfont\sffamily}
\usepackage{sphinx}

\fvset{fontsize=\small}
\usepackage{geometry}


% Include hyperref last.
\usepackage{hyperref}
% Fix anchor placement for figures with captions.
\usepackage{hypcap}% it must be loaded after hyperref.
% Set up styles of URL: it should be placed after hyperref.
\urlstyle{same}

\addto\captionsfrench{\renewcommand{\contentsname}{table des matières}}

\usepackage{sphinxmessages}
\setcounter{tocdepth}{1}



\title{Documentation Technique de Caiman}
\date{juin 10, 2021}
\release{1.1.0}
\author{Lorenzo Bauduccio}
\newcommand{\sphinxlogo}{\vbox{}}
\renewcommand{\releasename}{Version}
\makeindex
\begin{document}

\ifdefined\shorthandoff
  \ifnum\catcode`\=\string=\active\shorthandoff{=}\fi
  \ifnum\catcode`\"=\active\shorthandoff{"}\fi
\fi

\pagestyle{empty}
\sphinxmaketitle
\pagestyle{plain}
\sphinxtableofcontents
\pagestyle{normal}
\phantomsection\label{\detokenize{index::doc}}



\chapter{Introduction}
\label{\detokenize{introduction:introduction}}\label{\detokenize{introduction::doc}}

\section{Contexte}
\label{\detokenize{introduction:contexte}}
\sphinxAtStartPar
Caiman a été réalisé dans le cadre du travail de diplôme de technicien deuxième année. La réussite du travail de diplôme est l’une des conditions pour recevoir son titre de technicien en informatique.


\section{Motivations}
\label{\detokenize{introduction:motivations}}
\sphinxAtStartPar
Le choix de ce travail de diplôme découle de mon envie de faire un travail de diplôme qui serait intéressant, ludique et en lien avec les jeux vidéo. La réalisation d’un jeu vidéo étant trop compliquée pour un travail de diplôme, j’ai décidé de créer une application a mi\sphinxhyphen{}chemin entre un launcher de jeu comme Steam ou GOG et un gestionnaire de VM.

\sphinxAtStartPar
J’utilise depuis des années des émulateurs et je me heurte souvent aux mêmes problèmes, perte de sauvegarde dû à une réinstallation de windows, problème pour ripper mes jeux, téléchargement de chaque émulateur un par un. Donc, J’ai eu l’idée de créer Caiman pour pallier ces différents problèmes.

\sphinxAtStartPar
Un autre aspect important pour moi est la simplicité. Je voulais que l’utilisation de l’application soit le plus simple possible, une personne sans connaissance particulière en informatique doit pouvoir utiliser Caiman sans difficulté. Une autre spécification est que l’utilisation de Caiman puisse se faire à la manette. Pour créer une interface ergonomique, je me suis basé sur l’interface Big Picture de Steam qui a justement été créé pour être utilisé  à la manette.


\section{Résumé}
\label{\detokenize{introduction:resume}}
\sphinxAtStartPar
Ce document décrit le processus de conception de mon projet Caiman. Caiman est inspiré par des applications comme Retropie, RetroArch ou Steam. RetroPie est une distribution pour raspberry comprenant des émulateurs pour d’anciennes consoles de jeu. RetroArch est lui une application servant de frontend pour émulateur, RetroArch est disponible sur plusieurs plateformes(Windows, macOS, Xbox).

\sphinxAtStartPar
Caiman est un projet comprenant deux parties distinctes. La première partie consiste en un site web PHP permettant de se créer un compte et de s’identifier. Le site web permet de consulter des informations sur les jeux présent sur le store ainsi que de pouvoir administrer les sites et le store.

\sphinxAtStartPar
La deuxième partie du projet est une application C\# inspirée par le mode Big Picture de steam. Caiman sert de frontend pour différents émulateurs (Dolphin, PCSX2,etc). L’application permet de télécharger les différents jeux qui ont été ajoutés au store prévu pour le projet. Les jeux pris en charge sont ajoutés par des administrateurs via le site internet de Caiman. L’application permet de synchroniser les sauvegardes de l’utilisateur et cela peut importe sur quel pc, il lance l’application.

\sphinxAtStartPar
\sphinxincludegraphics{{schema_intro_2}.jpg}


\section{Abstract}
\label{\detokenize{introduction:abstract}}
\sphinxAtStartPar
This document describes the design process of my project Caiman. Caiman is inspired by applications like Retropie, RetroArch or Steam. RetroPie is a raspberry distribution including emulators for old game consoles. RetroArch is a frontend application for emulators, RetroArch is available on several platforms (Windows, macOS, Xbox).

\sphinxAtStartPar
Caiman is a project with two distinct parts. The first part consists of a PHP website allowing the visitor to create an account and to identify himself. The website allows you to consult information about the games on the store and to administer the site and the store.

\sphinxAtStartPar
The second part of the project is a C\# application inspired by the Big Picture mode of Steam. Caiman is used as a frontend for different emulators (Dolphin, PCSX2, etc). The application allows you to download the different games that have been added to the store provided for the project. Supported games are added by administrators via the Caiman website. The application allows synchronizing the user’s save, no matter on which PC the user runs the application.


\section{Légalité du projet}
\label{\detokenize{introduction:legalite-du-projet}}

\subsection{Caiman}
\label{\detokenize{introduction:caiman}}
\sphinxAtStartPar
Je pense qu’il est nécessaire d’aborder le sujet de la légalité de Caiman. Selon le droit Suisse, il n’est pas légal de mettre à disposition des fichiers sous droit d’auteur. Il est par contre légal de télécharger des films, séries ou jeux à condition de les utiliser dans un but privé. Caiman ne répondant pas à ces descriptions l’utilisation de l’application n’est pas conforme au droit Suisse.


\subsection{Utilisation d’émulateurs}
\label{\detokenize{introduction:utilisation-demulateurs}}
\sphinxAtStartPar
L’utilisation d’émulateurs n’est pas illégale en soi, en sachant que le code des émulateurs n’est pas la propriété de Sony ou de Nintendo. Par contre, il y a un petit point technique à détailler. L’émulateur PCSX2 doit pour fonctionner utiliser un BIOS de console de Playstation 2. Le BIOS de la Playstation 2 étant soumis à des droits d’auteur, il n’est en théorie pas légal de le distribuer en Suisse.

\sphinxAtStartPar
Je tiens donc à spécifier que Caiman est développé dans un seul but pédagogique. Il n’a aucune vocation à être distribué après la fin de mon travail de diplôme. L’application sera supprimée du serveur de l “école à la fin de mon travail. Par contre, il serait légal de distribuer le projet si je supprime la fonctionnalité de téléchargement de jeu.


\section{Installation}
\label{\detokenize{introduction:installation}}

\subsection{Installation de Caiman utilisateur}
\label{\detokenize{introduction:installation-de-caiman-utilisateur}}
\sphinxAtStartPar
Pour utiliser Caiman, il faut le télécharger depuis \sphinxhref{http://caiman.cfpt.info}{caiman.cfpt.info}. Quand on télécharge l’application, l’utilisateur télécharge un fichier caiman.zip contenant l’application ainsi que les émulateurs, il n’y a donc rien d’autre à télécharger.

\sphinxAtStartPar
Quand l’utilisateur décompresse le fichier caiman.zip, il se retrouve avec un dossier Caiman contenant deux dossiers. Un contenant les émulateurs et un autre contenant les fichiers de Caiman.

\sphinxAtStartPar
\sphinxincludegraphics{{dossier_caiman}.png}

\sphinxAtStartPar
Pour pouvoir exécuter Caiman, il faut se rendre dans le dossier “\textbackslash{}bin\textbackslash{}Debug\textbackslash{}” et ensuite l’utilisateur va trouver le fichier “Caiman.exe”. Ce fichier permet de lancer l’application.

\sphinxAtStartPar
Caiman est donc une application “portable”, c’est\sphinxhyphen{}à\sphinxhyphen{} dire qu’elle ne nécessite pas d’une installation pour pouvoir être lancée.


\subsection{Installation de Caiman serveur}
\label{\detokenize{introduction:installation-de-caiman-serveur}}
\sphinxAtStartPar
Pour pouvoir déployer le site et l’API de caiman, il faut que le serveur ait les services suivant:
\begin{itemize}
\item {} 
\sphinxAtStartPar
Apache2

\item {} 
\sphinxAtStartPar
PHP 7.4 (minimum)

\item {} 
\sphinxAtStartPar
mysql

\end{itemize}

\sphinxAtStartPar
Pour la base de données, il faut l’importer dans mysql sans faire de manipulation particulière.

\sphinxAtStartPar
Le site web et l’api doivent se trouver dans le même dossier, en sachant qu’il a des chemins relatifs entre les deux dossiers. ils ne peuvent donc pas être séparés pour l’instant.

\sphinxAtStartPar
Pour pouvoir uploader des jeux, il faut augmenter la taille que le PHP peut recevoir par formulaire, il faut également le configurer pour recevoir minimum 8GB. Pour finir, Il faut autoriser l’écriture dans les dossiers des sauvegardes et des jeux.


\section{Poster}
\label{\detokenize{introduction:poster}}
\sphinxAtStartPar
\sphinxincludegraphics{{poster}.png}


\chapter{Cahier des charges}
\label{\detokenize{cdc:cahier-des-charges}}\label{\detokenize{cdc::doc}}

\section{Objectifs}
\label{\detokenize{cdc:objectifs}}
\sphinxAtStartPar
Caiman est une application pour Windows regroupant plusieurs émulateurs de console de jeu. L’utilisateur peut exécuter les jeux depuis Caiman.

\sphinxAtStartPar
Caiman a pour but d’être simple d’utilisation, aucune connaissance n’est requise pour l’utiliser. Caiman permet une utilisation complète à la manette ou au clavier/souris. L’avantage de Caiman est qu’il ne requiert aucune configuration. Caiman est utilisable instantanément par l’utilisateur, contrairement  d’un émulateur traditionnel qui requiert une configuration relativement compliquée.

\sphinxAtStartPar
Le téléchargement des jeux se fait directement depuis Caiman, un serveur nommé le “Bunker” contient les fichiers des différents jeux.

\sphinxAtStartPar
Un site web permet de créer un compte pour accéder à l’application , de voir les informations des joueurs ainsi que de télécharger Caiman.


\section{Spécification}
\label{\detokenize{cdc:specification}}

\subsection{Emulateurs et contrôles}
\label{\detokenize{cdc:emulateurs-et-controles}}
\sphinxAtStartPar
Les différents émulateurs qui permettent d’exécuter des jeux dans Caiman sont:
\begin{itemize}
\item {} 
\sphinxAtStartPar
PCSX2 (playstation 2)

\item {} 
\sphinxAtStartPar
Dolphin (gamecube / wii)

\end{itemize}

\sphinxAtStartPar
L’utilisateur doit utiliser une manette de Xbox pour utiliser Caiman à la manette (une seule manette est requise pour jouer mais plusieurs peuvent être connectées, pour jouer en multijoueurs local).

\sphinxAtStartPar
Caiman est utilisable à la manette à l’exception des formulaires (création de compte, importation de jeux).


\subsection{Création d’un compte utilisateur}
\label{\detokenize{cdc:creation-dun-compte-utilisateur}}
\sphinxAtStartPar
L’utilisateur a l’obligation de créer un compte pour pouvoir utiliser Caiman.

\sphinxAtStartPar
L’utilisateur peut créer un compte directement depuis Caiman s’il n’en possède pas.


\subsection{Interface}
\label{\detokenize{cdc:interface}}
\sphinxAtStartPar
Dans l’interface, un système de catégories existe pour permettre de se retrouver plus aisément dans la liste des jeux à télécharger. Les différentes catégories sont créées par un administrateur.

\sphinxAtStartPar
Exemple de catégories:
\begin{itemize}
\item {} 
\sphinxAtStartPar
jeux Mario

\item {} 
\sphinxAtStartPar
jeux Zelda

\item {} 
\sphinxAtStartPar
jeux d’aventure

\item {} 
\sphinxAtStartPar
jeux de réflexion

\item {} 
\sphinxAtStartPar
jeux multijoueur

\item {} 
\sphinxAtStartPar
jeux favoris

\item {} 
\sphinxAtStartPar
etc..

\end{itemize}

\sphinxAtStartPar
L’utilisateur peut ajouter depuis Caiman des jeux favoris pour pouvoir les retrouver plus facilement. ( une catégorie “favoris” est visible contenant les jeux favoris de l’utilisateur authentifié )


\subsection{Paramètre graphiques}
\label{\detokenize{cdc:parametre-graphiques}}
\sphinxAtStartPar
L’utilisateur a la possibilité de modifier certains paramètres d’émulation:
\begin{itemize}
\item {} 
\sphinxAtStartPar
La définition des jeux (définition native, proche de 1080p, proche de 4K).

\item {} 
\sphinxAtStartPar
La langue des machines.

\item {} 
\sphinxAtStartPar
Si les jeux doivent être lancés en plein écran ou non.

\item {} 
\sphinxAtStartPar
Format d’écran 16/9 ou 4/3.

\end{itemize}


\subsection{Gestion des sauvegardes}
\label{\detokenize{cdc:gestion-des-sauvegardes}}
\sphinxAtStartPar
Il existe plusieurs cas où les sauvegardes de l’utilisateur se synchronisent:
\begin{enumerate}
\sphinxsetlistlabels{\arabic}{enumi}{enumii}{}{.}%
\item {} 
\sphinxAtStartPar
Le joueur sauvegarde sa partie directement dans un jeu depuis Caiman. La sauvegarde que l’utilisateur vient de créer est envoyée sur le Bunker et remplace la version présente.

\item {} 
\sphinxAtStartPar
Quand le joueur lance Caiman, une vérification est faite pour savoir si une version plus récente des sauvegardes du joueur sont présentes, si c’est le cas, elles sont téléchargées sur la machine du joueur.

\end{enumerate}


\subsection{Spécifications du “Bunker”}
\label{\detokenize{cdc:specifications-du-bunker}}
\sphinxAtStartPar
Le Bunker contient les fichiers des différents jeux au format (.ISO).

\sphinxAtStartPar
Le Bunker contient les fichiers de sauvegardes des utilisateurs. Ces sauvegardes seront envoyées de la machine de l’utilisateur vers le Bunker quand l’utilisateur sauvegarde sa partie depuis un jeu.

\sphinxAtStartPar
Les données des utilisateurs (email, jeux favoris, heures de jeux) sont stockées dans une base de données.


\subsection{Spécification site web (site présent dans le Bunker)}
\label{\detokenize{cdc:specification-site-web-site-present-dans-le-bunker}}
\sphinxAtStartPar
Sur le site web, il est possible de créer un compte en spécifiant un mail et un mot de passe.

\sphinxAtStartPar
L’interface web permet de rechercher des utilisateurs pour voir leur profil.

\sphinxAtStartPar
L’affichage d’un profil permet de voir les jeux favoris d’un utilisateur ainsi que son nombre d’heures de jeu sur chaque jeu.

\sphinxAtStartPar
Le site web permet à un administrateur d’ajouter des jeux et de modifier les informations des jeux.

\sphinxAtStartPar
Le site web permet de modifier les informations de compte d’un utilisateur.

\sphinxAtStartPar
Un utilisateur peut faire la demande de réinitialiser son mot de passe.


\subsection{Installation}
\label{\detokenize{cdc:installation}}
\sphinxAtStartPar
Caiman est téléchargeable depuis le site web. L’installateur contient les émulateurs, il n’y a donc rien à télécharger d’autre.


\section{Limites du Projet}
\label{\detokenize{cdc:limites-du-projet}}
\sphinxAtStartPar
Caiman sera limité aux jeux que j’ai pu ajouter moi même ainsi qu’aux émulateurs que j’aurais importé, la totalité des consoles de jeu ne sera donc pas prise en charge.


\section{Calendrier}
\label{\detokenize{cdc:calendrier}}
\sphinxAtStartPar
Le début du travail de diplôme est fixé au lundi 19 avril et le rendu est fixé au 11 juin à 12h00.

\sphinxAtStartPar
Le travail est à exécuter soit en présentiel et en télétravail du lundi au vendredi et cela 8h00 par jour.


\section{Matériel}
\label{\detokenize{cdc:materiel}}
\sphinxAtStartPar
J’ai à ma disposition pour la réalisation de Caiman:
\begin{itemize}
\item {} 
\sphinxAtStartPar
Mon pc pour le développement
\begin{itemize}
\item {} 
\sphinxAtStartPar
Intel i7\sphinxhyphen{}7700 3.6 GHz

\item {} 
\sphinxAtStartPar
32 GB de ram

\item {} 
\sphinxAtStartPar
GTX 1060 3GB

\end{itemize}

\item {} 
\sphinxAtStartPar
Un serveur pour héberger le site et les fichiers du store

\item {} 
\sphinxAtStartPar
Une manette de Xbox Series

\end{itemize}


\chapter{Étude d’opportunité}
\label{\detokenize{opportunite:etude-dopportunite}}\label{\detokenize{opportunite::doc}}

\section{RetroPie}
\label{\detokenize{opportunite:retropie}}
\sphinxAtStartPar
\sphinxincludegraphics{{retro_pie}.jpg}

\sphinxAtStartPar
RetroPie est une distribution Linux prévue pour Raspberry et PC. C’est un frontend pour jouer aux jeux d’arcade, d’anciennes consoles et de jeux pc. Retropie est à la différence de Caiman un OS à part entière et chaque jeu doit être ajouté indépendamment par l’utilisateur sur le Rasberry/PC.


\section{RetroArch}
\label{\detokenize{opportunite:retroarch}}
\sphinxAtStartPar
\sphinxincludegraphics{{RetroArch}.jpg}

\sphinxAtStartPar
RetroArch est un frontend pour émulateurs, moteur de jeu et media players. Il embarque un très grand nombre d’émulateurs et a l’avantage d’être disponible sur un très grand nombre de plateformes.

\sphinxAtStartPar
RetroArch possède des fonctionnalités très utiles comme des Shaders intégré, le jeux en réseau, le calcul du temps entre les frames, et d’autres. C’est surement le Frontend pour émulateur le plus complet disponible.

\sphinxAtStartPar
L’interface de RetroArch est inspirée par le XMB de la Playstation 3 et de la PSP, ce qui le rend compatible avec une manette.

\sphinxAtStartPar
RetroArch est disponible sur:
\begin{itemize}
\item {} 
\sphinxAtStartPar
Windows

\item {} 
\sphinxAtStartPar
Linux

\item {} 
\sphinxAtStartPar
Raspberry

\item {} 
\sphinxAtStartPar
Android

\item {} 
\sphinxAtStartPar
Apple macOS (ARM/x64)

\item {} 
\sphinxAtStartPar
Apple macOS/OSX

\item {} 
\sphinxAtStartPar
IOS/Apple TV

\item {} 
\sphinxAtStartPar
Xbox Series / One

\item {} 
\sphinxAtStartPar
OpenDingux

\item {} 
\sphinxAtStartPar
PSvita

\item {} 
\sphinxAtStartPar
PSP

\item {} 
\sphinxAtStartPar
PS2

\item {} 
\sphinxAtStartPar
PS3

\item {} 
\sphinxAtStartPar
PS4

\item {} 
\sphinxAtStartPar
Switch

\item {} 
\sphinxAtStartPar
WII U

\item {} 
\sphinxAtStartPar
WII

\item {} 
\sphinxAtStartPar
Gamecube

\item {} 
\sphinxAtStartPar
3DS / 2DS

\end{itemize}


\section{RomStation}
\label{\detokenize{opportunite:romstation}}
\sphinxAtStartPar
\sphinxincludegraphics{{RomStation}.png}

\sphinxAtStartPar
RomStation est un frontend pour émulateur facilitant l’émulation de nombreuses consoles. RomStation a la différence de RetroPie ou RetroArch, l’utilisateur a la possibilité de  télécharger des jeux directement depuis l’application. L’application a une formule payante qui permet d’améliorer la vitesse des téléchargements des jeux. RomStation est disponible sur Windows (x64)(x86), macOS.


\section{Steam}
\label{\detokenize{opportunite:steam}}
\sphinxAtStartPar
\sphinxincludegraphics{{steam}.png}

\sphinxAtStartPar
Steam est une plateforme de distribution de jeux et d’applications en ligne. Steam est spécialisé dans les jeux vidéo contrairement aux autres projets que j’ai cité ci\sphinxhyphen{}dessus. C’est une application commerciale à but lucratif.

\sphinxAtStartPar
Steam n’est pas directement liée à l’utilisation d’émulateurs, mais j’ai décidé d’en parler car je m’inspire de son interface. Elle est créée spécifiquement pour une utilisation à la manette et la gestion que fais steam dès sauvegardes des utilisateurs.


\section{Conclusion de l’analyse de l’existant}
\label{\detokenize{opportunite:conclusion-de-lanalyse-de-l-existant}}
\sphinxAtStartPar
Il existe un grand nombre de frontend pour émulateur, ils permettent de simplifier l’utilisation pour l’utilisateur mais je n’en ai pas trouvé qui essaie d’atteindre le même but que moi. Le but de Caiman est de permettre la simplification de l’utilisation d’émulateurs et de permettre de synchroniser des sauvegardes tout en fournissant la possibilité de télécharger des jeux.

\sphinxAtStartPar
Steam représente un niveau de finition dans son interface BigPicture que j’aimerais atteindre. La synchronisation des sauvegardes sur steam est peut\sphinxhyphen{}être la plus performante, si l’on compare avec d’autres boutiques comme l’Epic Game Store ou Origin.


\chapter{Analyse fonctionnelle}
\label{\detokenize{fonctionnelle:analyse-fonctionnelle}}\label{\detokenize{fonctionnelle::doc}}

\section{Base de données}
\label{\detokenize{fonctionnelle:base-de-donnees}}

\subsection{Schéma}
\label{\detokenize{fonctionnelle:schema}}
\sphinxAtStartPar
Le schéma suivant représente la structure de la base de données. La base de données est commune au site web et à l’application Caiman. Les tables sont principalement utilisées pour stocker les informations des utilisateurs et des différents jeux.

\sphinxAtStartPar
\sphinxincludegraphics{{caiman_database}.png}


\subsection{Table categorie}
\label{\detokenize{fonctionnelle:table-categorie}}
\sphinxAtStartPar
Cette table est utilisée pour stocker les différentes catégories auxquelles les jeux peuvent appartenir.


\subsection{Table console}
\label{\detokenize{fonctionnelle:table-console}}
\sphinxAtStartPar
Cette table sert à stocker les différentes consoles prises en charge par l’application Caiman. Chaque console doit être reliée à un émulateur.


\subsection{Table emulator}
\label{\detokenize{fonctionnelle:table-emulator}}
\sphinxAtStartPar
La table émulateur sert à lister les différents émulateurs disponible pour l’application, il est possible qu” un émulateur soit compatible avec plusieurs consoles.


\subsection{Table favoritegame}
\label{\detokenize{fonctionnelle:table-favoritegame}}
\sphinxAtStartPar
Sert à lister les jeux favoris des utilisateurs.


\subsection{Table file}
\label{\detokenize{fonctionnelle:table-file}}
\sphinxAtStartPar
La table file sert à lister le nom d’un fichier, cela peut être un fichier .iso d’un jeu, un fichier de sauvegarde ou un fichier de configuration utilisateur.


\subsection{Table filesave}
\label{\detokenize{fonctionnelle:table-filesave}}
\sphinxAtStartPar
Sert  faire le lien entre un émulateur, un utilisateur, et un fichier de sauvegarde. Alors elle permet de connaître les sauvegardes pour un émulateur particulier d’un utilisateur.


\subsection{Table game}
\label{\detokenize{fonctionnelle:table-game}}
\sphinxAtStartPar
Table qui liste les jeux disponibles. Chaque jeu a plusieurs informations: un nom, une description, une console. La console permet de à l’application de savoir quel émulateur doit être utilisé.


\subsection{Table gamehascategorie}
\label{\detokenize{fonctionnelle:table-gamehascategorie}}
\sphinxAtStartPar
Sert à assigner des catégories aux jeux.


\subsection{Table rôle}
\label{\detokenize{fonctionnelle:table-role}}
\sphinxAtStartPar
Sert à lister les différents rôles (utilisateur, administrateur), elle ne sert que pour le site web sachant que toute la partie administration est sur le site.


\subsection{Table timeingame}
\label{\detokenize{fonctionnelle:table-timeingame}}
\sphinxAtStartPar
Sert à stocker le temps de jeu en minutes de chaque utilisateur. Le temps de jeu est spécifique à chaque jeu. Il est mis à jour directement depuis l’application Caiman.


\subsection{Table user}
\label{\detokenize{fonctionnelle:table-user}}
\sphinxAtStartPar
Cette table sert à stocker les informations de compte de chaque utilisateur de Caiman. Le compte est commun au site web et à l’application. Le mot de passe de l’utilisateur est crypté avec les fonctions de sécurité de php.


\subsection{Table userconfigfile}
\label{\detokenize{fonctionnelle:table-userconfigfile}}
\sphinxAtStartPar
Cette table sert à faire le lien entre un fichier de configuration et un utilisateur. Le fichier de configuration sert à connaître la configuration.


\section{Site web Caiman}
\label{\detokenize{fonctionnelle:site-web-caiman}}

\subsection{Création de compte}
\label{\detokenize{fonctionnelle:creation-de-compte}}
\sphinxAtStartPar
L’utilisateur du site a la possibilité de créer un compte qui sera commun au site et à l’application. La création de compte nécessite de renseigner son email, de donner un nom d’utilisateur ainsi que un mot de passe.

\sphinxAtStartPar
\sphinxincludegraphics{{SignIn}.png}


\subsection{connexion}
\label{\detokenize{fonctionnelle:connexion}}
\sphinxAtStartPar
La connexion à son compte utilisateur permet de modifier nos informations de compte et d’ajouter ou de supprimer des jeux à la liste de favoris.

\sphinxAtStartPar
Si l’utilisateur oublie son mot de passe, il a la possibilité de le réinitialiser. L’utilisateur qui se décide de réinitialiser son mot de passe reçoit un mail contenant un lien de réinitialisation.

\sphinxAtStartPar
\sphinxincludegraphics{{login}.png}


\subsection{modification des informations d’un utilisateur}
\label{\detokenize{fonctionnelle:modification-des-informations-dun-utilisateur}}
\sphinxAtStartPar
Un utilisateur connecté a la possibilité de modifier ces informations:
\begin{itemize}
\item {} 
\sphinxAtStartPar
mot de passe

\item {} 
\sphinxAtStartPar
liste de jeux favoris

\item {} 
\sphinxAtStartPar
la visibilité de son profil pour les autres utilisateurs

\end{itemize}

\sphinxAtStartPar
\sphinxincludegraphics{{usersInfos}.png}


\subsection{affichage des jeux}
\label{\detokenize{fonctionnelle:affichage-des-jeux}}
\sphinxAtStartPar
Tous les utilisateurs ont la possibilité d’afficher la liste de jeux disponible. Il n’y a pas de restriction particulière.

\sphinxAtStartPar
Les informations disponibles pour chaque jeu sont les suivantes:
\begin{itemize}
\item {} 
\sphinxAtStartPar
Nom

\item {} 
\sphinxAtStartPar
Description

\item {} 
\sphinxAtStartPar
Catégories du jeu

\end{itemize}

\sphinxAtStartPar
\sphinxincludegraphics{{gamesDetails}.png}


\subsection{affichage d’un profil utilisateur}
\label{\detokenize{fonctionnelle:affichage-dun-profil-utilisateur}}
\sphinxAtStartPar
Il est possible de consulter la page personnelle d’un utilisateur si celui\sphinxhyphen{}ci a rendu son compte publique. Les informations disponibles sont celle\sphinxhyphen{}ci:
\begin{itemize}
\item {} 
\sphinxAtStartPar
Nom d’utilisateur

\item {} 
\sphinxAtStartPar
Jeux favoris

\item {} 
\sphinxAtStartPar
Nombres d’heures de jeux sur chaque jeu

\end{itemize}

\sphinxAtStartPar
\sphinxincludegraphics{{UserFavoritesGames}.png}

\sphinxAtStartPar
\sphinxincludegraphics{{UserTimeInGame}.png}


\subsection{ajout d’un jeu à la base de données / sur le Bunker pour le fichier .ISO}
\label{\detokenize{fonctionnelle:ajout-dun-jeu-a-la-base-de-donnees-sur-le-bunker-pour-le-fichier-iso}}
\sphinxAtStartPar
L’ajout d’un jeu se fait grâce à un formulaire, plusieurs champs sont à renseigner:
\begin{itemize}
\item {} 
\sphinxAtStartPar
le nom du jeu

\item {} 
\sphinxAtStartPar
une description

\item {} 
\sphinxAtStartPar
une image

\item {} 
\sphinxAtStartPar
la console du jeu qui est uploadé

\item {} 
\sphinxAtStartPar
le nom que va porter le jeu sur le Bunker

\item {} 
\sphinxAtStartPar
le fichier de base du jeu

\end{itemize}

\sphinxAtStartPar
L’ajout à la base va créer deux entrées. Une dans la table Game et une autre dans la table file. Après avoir été ajouté depuis le site web, le jeu devient accessible depuis l’application Caiman et le site web pour de la consultation.

\sphinxAtStartPar
\sphinxincludegraphics{{FormAddGame}.png}


\subsection{modification d’un jeu}
\label{\detokenize{fonctionnelle:modification-dun-jeu}}
\sphinxAtStartPar
La modification d’un jeu ne peut être faite que par un administrateur. Les modifications possibles sont les suivantes:
\begin{itemize}
\item {} 
\sphinxAtStartPar
nom

\item {} 
\sphinxAtStartPar
description

\item {} 
\sphinxAtStartPar
console

\item {} 
\sphinxAtStartPar
catégories

\end{itemize}

\sphinxAtStartPar
\sphinxincludegraphics{{GameDetailsAdmin}.png}


\subsection{Administration}
\label{\detokenize{fonctionnelle:administration}}
\sphinxAtStartPar
Les administrateurs ont la possibilité de faire plusieurs choses. Donc, je vais les lister ici, il n’est pas nécessaire de les détailler.
\begin{itemize}
\item {} 
\sphinxAtStartPar
modifier un jeu

\item {} 
\sphinxAtStartPar
ajouter des catégories

\item {} 
\sphinxAtStartPar
ajouter ou supprimer des catégories à un jeu

\end{itemize}

\sphinxAtStartPar
\sphinxincludegraphics{{FormAddCategory}.png}


\subsection{Téléchargement}
\label{\detokenize{fonctionnelle:telechargement}}
\sphinxAtStartPar
L’un des intérêts du site est de pouvoir télécharger l’application Caiman. Le téléchargement de cette application nécessite d’être authentifié sur le site. Si un invité se rend sur la page de téléchargement sans être authentifié, une invitation lui sera faite de s’authentifier ou de créer un compte.

\sphinxAtStartPar
\sphinxincludegraphics{{downloadNotConnected}.png}

\sphinxAtStartPar
\sphinxincludegraphics{{download}.png}


\subsection{Fonctionnalitées manquante}
\label{\detokenize{fonctionnelle:fonctionnalitees-manquante}}
\sphinxAtStartPar
Malheureusement dû au temps imposé j’ai dû faire des choix organisationnel. J’ai donc décidé de me focaliser sur L’application Caiman au lieu du site internet ce qui explique que certaines fonctionnalités ne soient pas finalisées ou mises en place. Je vais lister les ce qui n’a pas été fini complètement ou abandonné.
\begin{itemize}
\item {} 
\sphinxAtStartPar
Modification complète des données des jeux

\item {} 
\sphinxAtStartPar
Modification d’une catégorie

\item {} 
\sphinxAtStartPar
Réinitialisation de mot de passe

\item {} 
\sphinxAtStartPar
Création de compte administrateur

\end{itemize}


\subsection{API Caiman}
\label{\detokenize{fonctionnelle:api-caiman}}
\sphinxAtStartPar
L’API sert à pouvoir accéder à la base de données depuis l’application Caiman. Je vais détailler les endpoint et la structure de l’API.


\subsection{Structure de l’API}
\label{\detokenize{fonctionnelle:structure-de-lapi}}
\sphinxAtStartPar
\sphinxincludegraphics{{schema-API}.png}

\sphinxAtStartPar
Pour expliquer la structure de l’API, je vais expliquer étape par étape comment un appel se passe.
\begin{enumerate}
\sphinxsetlistlabels{\arabic}{enumi}{enumii}{}{.}%
\item {} 
\sphinxAtStartPar
L’utilisateur envoie une requête à la page index.php de api.caiman.cfpt.info.

\item {} 
\sphinxAtStartPar
La requête est réceptionnée par index.php. L’url est ensuite traité par le .htaccess pour savoir où doit envoyer à quel controller.

\item {} 
\sphinxAtStartPar
Le contrôleur décide selon les informations reçues quelle méthode il doit exécuter.

\item {} 
\sphinxAtStartPar
Le DAO est appelé et va rechercher dans la base de données les données demandé

\item {} 
\sphinxAtStartPar
Le DAO créé la réponse grâce au model.

\item {} 
\sphinxAtStartPar
La réponse est envoyée à l’utilisateur par l’intermédiaire de la page index.php

\end{enumerate}


\subsection{Categories}
\label{\detokenize{fonctionnelle:categories}}

\subsection{GET}
\label{\detokenize{fonctionnelle:get}}
\sphinxAtStartPar
Permet de recevoir la liste des catégories disponibles.

\sphinxAtStartPar
Les informations reçues sont les suivantes:
\begin{itemize}
\item {} 
\sphinxAtStartPar
id

\item {} 
\sphinxAtStartPar
nom

\end{itemize}


\subsection{Games}
\label{\detokenize{fonctionnelle:games}}

\subsection{GET}
\label{\detokenize{fonctionnelle:id1}}
\sphinxAtStartPar
Retourne la liste des jeux.

\sphinxAtStartPar
Les informations reçues sont les suivantes:
\begin{itemize}
\item {} 
\sphinxAtStartPar
id

\item {} 
\sphinxAtStartPar
description

\item {} 
\sphinxAtStartPar
nom de l’image

\item {} 
\sphinxAtStartPar
id de la console

\item {} 
\sphinxAtStartPar
id du fichier du jeu

\end{itemize}


\subsection{GET(?byName)}
\label{\detokenize{fonctionnelle:get-byname}}
\sphinxAtStartPar
Retourne la liste des jeux qui dans le nom contient ce que l’utilisateur a demandé.

\sphinxAtStartPar
Les informations reçues sont les suivantes:
\begin{itemize}
\item {} 
\sphinxAtStartPar
id

\item {} 
\sphinxAtStartPar
description

\item {} 
\sphinxAtStartPar
nom de l’image

\item {} 
\sphinxAtStartPar
id de la console

\item {} 
\sphinxAtStartPar
id du fichier du jeu

\end{itemize}


\subsection{GET(?byCategory)}
\label{\detokenize{fonctionnelle:get-bycategory}}
\sphinxAtStartPar
Retourne la liste des jeux qui appartiennent à une catégorie.

\sphinxAtStartPar
Il faut spécifier l’id de la catégorie qui est demandée.

\sphinxAtStartPar
Les informations reçues sont les suivantes:
\begin{itemize}
\item {} 
\sphinxAtStartPar
id

\item {} 
\sphinxAtStartPar
description

\item {} 
\sphinxAtStartPar
nom de l’image

\item {} 
\sphinxAtStartPar
id de la console

\item {} 
\sphinxAtStartPar
id du fichier du jeu

\end{itemize}


\subsection{GET(?byFavoriteUser)}
\label{\detokenize{fonctionnelle:get-byfavoriteuser}}
\sphinxAtStartPar
Retourne la liste des jeux favoris d’un utilisateur.

\sphinxAtStartPar
Il faut spécifier l’id de l’utilisateur.

\sphinxAtStartPar
Les informations reçues sont les suivantes:
\begin{itemize}
\item {} 
\sphinxAtStartPar
id

\item {} 
\sphinxAtStartPar
description

\item {} 
\sphinxAtStartPar
nom de l’image

\item {} 
\sphinxAtStartPar
id de la console

\item {} 
\sphinxAtStartPar
id du fichier du jeu

\end{itemize}


\subsection{GET(?byUserTime)}
\label{\detokenize{fonctionnelle:get-byusertime}}
\sphinxAtStartPar
Retourne la liste des jeux auxquels un joueur a joué.

\sphinxAtStartPar
Il faut spécifier l’id de l’utilisateur.

\sphinxAtStartPar
Les informations reçues sont les suivantes:
\begin{itemize}
\item {} 
\sphinxAtStartPar
id

\item {} 
\sphinxAtStartPar
description

\item {} 
\sphinxAtStartPar
nom de l’image

\item {} 
\sphinxAtStartPar
id de la console

\item {} 
\sphinxAtStartPar
id du fichier du jeu

\item {} 
\sphinxAtStartPar
nombre de minutes en jeu

\end{itemize}


\subsection{GET(?gameFileName)}
\label{\detokenize{fonctionnelle:get-gamefilename}}
\sphinxAtStartPar
Retourne le nom du fichier d’un jeu.

\sphinxAtStartPar
Les informations reçues sont les suivantes:
\begin{itemize}
\item {} 
\sphinxAtStartPar
filename

\end{itemize}


\subsection{GET(?gameConsole)}
\label{\detokenize{fonctionnelle:get-gameconsole}}
\sphinxAtStartPar
Retourne la console d’un jeu.

\sphinxAtStartPar
Les informations reçues sont les suivantes:
\begin{itemize}
\item {} 
\sphinxAtStartPar
name

\item {} 
\sphinxAtStartPar
folderName

\end{itemize}


\subsection{GET(?idGame\&apiKey)}
\label{\detokenize{fonctionnelle:get-idgame-apikey}}
\sphinxAtStartPar
Retourne le fichier d’un jeu.

\sphinxAtStartPar
Les informations reçues sont les suivantes:
\begin{itemize}
\item {} 
\sphinxAtStartPar
fichier.iso

\end{itemize}


\subsection{GET(?idGameTime\&idUser)}
\label{\detokenize{fonctionnelle:get-idgametime-iduser}}
\sphinxAtStartPar
Retourne le temps de jeu sur un jeu.

\sphinxAtStartPar
Les informations reçues sont les suivantes:
\begin{itemize}
\item {} 
\sphinxAtStartPar
minutes

\end{itemize}


\subsection{GET(?idEmulator\&idUser\&apiKey)}
\label{\detokenize{fonctionnelle:get-idemulator-iduser-apikey}}
\sphinxAtStartPar
Retourne un fichier zip contenant les sauvegardes d’un joueur pour un émulateur particulier

\sphinxAtStartPar
Les informations reçues sont les suivantes:
\begin{itemize}
\item {} 
\sphinxAtStartPar
fichier.zip

\end{itemize}


\subsection{POST(?idEmulator\&idUser\&apiKey)}
\label{\detokenize{fonctionnelle:post-idemulator-iduser-apikey}}
\sphinxAtStartPar
Upload un fichier zip contenant les sauvegardes d’un joueur pour un émulateur particulier


\subsection{POST(idGameAdd\&idUser)}
\label{\detokenize{fonctionnelle:post-idgameadd-iduser}}
\sphinxAtStartPar
Ajouter un jeu en favoris pour un utilisateur particulier


\subsection{POST(idGameRemove\&idUser)}
\label{\detokenize{fonctionnelle:post-idgameremove-iduser}}
\sphinxAtStartPar
Supprime un jeu en favoris pour un utilisateur particulier


\subsection{POST(idGameCheck\&idUser)}
\label{\detokenize{fonctionnelle:post-idgamecheck-iduser}}
\sphinxAtStartPar
Vérifie si un jeu est déjà en favoris et retourne un booléen


\subsection{POST(idGameTimeAdd\&idUser)}
\label{\detokenize{fonctionnelle:post-idgametimeadd-iduser}}
\sphinxAtStartPar
Ajouter une minute de jeu à un jeu particulier pour un utilisateur


\subsection{Users}
\label{\detokenize{fonctionnelle:users}}

\subsection{GET(sans paramétres)}
\label{\detokenize{fonctionnelle:get-sans-parametres}}
\sphinxAtStartPar
Retourne la liste des utilisateurs.

\sphinxAtStartPar
Les informations reçues sont les suivantes:
\begin{itemize}
\item {} 
\sphinxAtStartPar
id

\item {} 
\sphinxAtStartPar
username

\end{itemize}


\subsection{POST(avec apitoken)}
\label{\detokenize{fonctionnelle:post-avec-apitoken}}
\sphinxAtStartPar
Retourne un utilisateur en particulier

\sphinxAtStartPar
Les informations reçues sont les suivantes:
\begin{itemize}
\item {} 
\sphinxAtStartPar
id

\item {} 
\sphinxAtStartPar
username

\item {} 
\sphinxAtStartPar
password

\item {} 
\sphinxAtStartPar
salt

\item {} 
\sphinxAtStartPar
apitoken

\item {} 
\sphinxAtStartPar
caimanToken

\item {} 
\sphinxAtStartPar
email

\item {} 
\sphinxAtStartPar
idRole

\end{itemize}


\subsection{User/connection}
\label{\detokenize{fonctionnelle:user-connection}}

\subsection{POST(?username, password)}
\label{\detokenize{fonctionnelle:post-username-password}}
\sphinxAtStartPar
Permet de vérifier les informations de connexion d’un utilisateur

\sphinxAtStartPar
Les informations reçues sont les suivantes:
\begin{itemize}
\item {} 
\sphinxAtStartPar
id

\item {} 
\sphinxAtStartPar
username

\item {} 
\sphinxAtStartPar
password

\item {} 
\sphinxAtStartPar
salt

\item {} 
\sphinxAtStartPar
apitoken

\item {} 
\sphinxAtStartPar
caimanToken

\item {} 
\sphinxAtStartPar
email

\item {} 
\sphinxAtStartPar
idRole

\end{itemize}


\subsection{POST(caimanToken)}
\label{\detokenize{fonctionnelle:post-caimantoken}}
\sphinxAtStartPar
Permet de recevoir les informations d’un utilisateur grâce à un token généré à chaque connexion.

\sphinxAtStartPar
Les informations reçues sont les suivantes:
\begin{itemize}
\item {} 
\sphinxAtStartPar
id

\item {} 
\sphinxAtStartPar
username

\item {} 
\sphinxAtStartPar
password

\item {} 
\sphinxAtStartPar
salt

\item {} 
\sphinxAtStartPar
apitoken

\item {} 
\sphinxAtStartPar
caimanToken

\item {} 
\sphinxAtStartPar
email

\item {} 
\sphinxAtStartPar
idRole

\end{itemize}


\section{Application  Caiman C\#}
\label{\detokenize{fonctionnelle:application-caiman-c}}

\subsection{Connexion}
\label{\detokenize{fonctionnelle:id2}}
\sphinxAtStartPar
L’utilisateur de Caiman doit obligatoirement être connecté pour pouvoir utiliser caiman. La connexion se fait avec le nom d’utilisateur et le mot de passe. Une fois la connexion valable, elle est maintenue tant que l’utilisateur ne se déconnecte pas ou tant que l’utilisateur ne s’est pas connecté sur un autre ordinateur.

\sphinxAtStartPar
\sphinxincludegraphics{{login_caiman}.png}


\subsection{Inscription}
\label{\detokenize{fonctionnelle:inscription}}
\sphinxAtStartPar
L’inscription n’est pas disponible sur caiman mais un bouton est disponible pour être redirigé sur le site web de Caiman.


\subsection{Téléchargement de jeu}
\label{\detokenize{fonctionnelle:telechargement-de-jeu}}
\sphinxAtStartPar
L’utilisateur a la possibilité de télécharger des jeux. Les jeux disponibles dans l’application sont ajoutés depuis le site internet de Caiman. Les téléchargements des jeux se font les un après les autres.

\sphinxAtStartPar
\sphinxincludegraphics{{Caiman_non_downloaded_game}.png}

\sphinxAtStartPar
\sphinxincludegraphics{{Caiman_download_queu}.png}


\subsection{Lancement d’un jeu}
\label{\detokenize{fonctionnelle:lancement-dun-jeu}}
\sphinxAtStartPar
Caiman inclut deux émulateurs Dolphin un PCSX2. Ces deux émulateurs permettent d’exécuter des jeux de Gamecube et Wii pour Dolphin et de Playstation 2 pour PCSX2. Pour lancer un jeu, il suffit de le télécharger, puis de cliquer sur Play. Il n’est donc pas nécessaire de lancer soit même un émulateur.

\sphinxAtStartPar
\sphinxincludegraphics{{caiman_downloaded_game}.png}


\subsection{Synchronisation des sauvegarde entre le pc client et le Bunker}
\label{\detokenize{fonctionnelle:synchronisation-des-sauvegarde-entre-le-pc-client-et-le-bunker}}
\sphinxAtStartPar
Les sauvegardes de l’utilisateur sont synchronisées entre les différents pc qu’il utilise. La synchronisation se fait à la connexion, les sauvegardes sont stockées sur les serveurs de Caiman. La copie des sauvegardes des utilisateurs est envoyée automatiquement donc l’utilisateur n’a pas de manipulation à faire. L’envoie se passe dès que l’utilisateur sauvegarde, cela permet d’éviter de perte de sauvegardes si un problème se produit durant le moment où le joueur est en train de jouer.


\subsection{Modification de la configuration utilisateur}
\label{\detokenize{fonctionnelle:modification-de-la-configuration-utilisateur}}
\sphinxAtStartPar
L’utilisateur a la possibilité de modifier plusieurs paramètres graphiques. Il a la possibilité de choisir en trois mode de configuration global:
\begin{itemize}
\item {} 
\sphinxAtStartPar
Original

\item {} 
\sphinxAtStartPar
1080p

\item {} 
\sphinxAtStartPar
4K

\end{itemize}

\sphinxAtStartPar
C’est différent mode modifie l’antialiasing et la définition native de l’émulateur qui va être utilisé.

\sphinxAtStartPar
L’utilisateur a aussi la possibilité de choisir entre lancer le jeu plein écran ou non et il peut choisir si le jeu doit être en 16/9 ou en 4/3.

\sphinxAtStartPar
\sphinxincludegraphics{{caiman_configuration}.png}


\subsection{Application de la configuration utilisateur}
\label{\detokenize{fonctionnelle:application-de-la-configuration-utilisateur}}
\sphinxAtStartPar
La configuration est appliquée avant de lancer un jeu, c’est à dire que la configuration n’est pas appliquée si l’utilisateur est déjà en jeu.


\subsection{Ajout/suppresion de jeu en favoris}
\label{\detokenize{fonctionnelle:ajout-suppresion-de-jeu-en-favoris}}
\sphinxAtStartPar
L’utilisateur a la possibilité de modifier ses jeux favoris directement depuis caiman.

\sphinxAtStartPar
\sphinxincludegraphics{{caiman_add_favorite}.png}

\sphinxAtStartPar
Suppression de jeu des favoris

\sphinxAtStartPar
\sphinxincludegraphics{{caiman_remove_favorite}.png}


\subsection{Affichage de jeux par catégories}
\label{\detokenize{fonctionnelle:affichage-de-jeux-par-categories}}
\sphinxAtStartPar
L’utilisateur a la possibilité d’afficher les jeux par catégories. Les catégories sont créées manuellement par un administrateur. Il y a certaines catégories « spéciales », par exemple les jeux favoris et les jeux uniques téléchargés par chaque utilisateur.

\sphinxAtStartPar
\sphinxincludegraphics{{caiman_categories}.png}


\subsection{Nombre d’heures de jeu}
\label{\detokenize{fonctionnelle:nombre-dheures-de-jeu}}
\sphinxAtStartPar
Le nombre de minutes de jeu d’un utilisateur est mis à jour à chaque minute de jeu. le nombre de minutes de jeu est visible sur la page de chaque jeu si l’utilisateur n’a pas encore joué.

\sphinxAtStartPar
\sphinxincludegraphics{{caiman_time_played}.png}


\subsection{Gestion des manettes}
\label{\detokenize{fonctionnelle:gestion-des-manettes}}
\sphinxAtStartPar
Pour Caiman, les manettes supportées sont les manettes pour xbox qui fonctionne avec Xinput. Il est possible d’utiliser d’autres manettes pour cela, il faut passer par un programme qui va convertir les inputs de la manette non compatible en input de manette xbox.

\sphinxAtStartPar
Pour gérer les déplacements, j’utilise les touches “haut, bas, gauche,droite”, le stick gauche, la validation se fait avec la touche “A” et le retour arrière avec la touche “B”.


\section{Interface utilisateur utilisable à la manette}
\label{\detokenize{fonctionnelle:interface-utilisateur-utilisable-a-la-manette}}

\subsection{Réception des input des manettes connecté}
\label{\detokenize{fonctionnelle:reception-des-input-des-manettes-connecte}}
\sphinxAtStartPar
L’utilisateur de Caiman a la possibilité de pouvoir utiliser l’application au clavier souris mais aussi à la manette. Pour ce faire, j’ai utilisé le paquet nuGet “SharpDX.XInput”. Ce paquet me permet de connaître les manettes connectées au pc ainsi que les touches appuyées par l’utilisateur.

\sphinxAtStartPar
La seule manette qui peut se déplacer dans l’application est la manette 1. Pour connaître les boutons de la manette, j’utilise la fonction getInput(). Cette fonction me permet de connaître les touches qui sont pressées à un instant T.  Je vais chercher les inputs toutes les 2ms pour être sûr de ne pas louper d’inputs.

\sphinxAtStartPar
Les inputs sont ensuite traités par l’interface de l’application qui décide quoi en faire selon le contexte.


\subsection{Transformation des input de la manette en événement}
\label{\detokenize{fonctionnelle:transformation-des-input-de-la-manette-en-evenement}}
\sphinxAtStartPar
Les inputs de la manette sont analysés par la form principale de Caiman. Selon la ou les touches qui sont pressées, l’application exécute des actions différentes. Par exemple, quand la touche “A” est pressée, alors le programme envoie la touche ENTER à l’application ce qui me permet de cliquer sur des boutons.

\sphinxAtStartPar
Pour gérer les événements, je fais un test pour savoir si l’utilisateur utilise l’application ou non . Si l’application n’est pas focus par l’utilisateur, seul une partie des actions sont possibles pour éviter que des actions inattendues puissent arriver alors que l’application n’est plus visible.


\subsection{Structure de l’affichage}
\label{\detokenize{fonctionnelle:structure-de-laffichage}}
\sphinxAtStartPar
L’affichage est constitué d’un “XboxMainForm”. Il sert à contenir tous les autres panels, il est aussi chargé de la gestion des inputs de la manette de l’utilisateur 1.

\sphinxAtStartPar
Un “XboxMainForm” contient une liste de XboxUserController qui elles contiennent différentes choses comme des boutons des images ou des labels.

\sphinxAtStartPar
Le XboxMainForm est aussi responsable de la gestion des demandes de l’utilisateur, par exemple si l’utilisateur veut afficher l’accueil de Caiman, il va lui passer un objet contenant sa demande. Il va donc traiter les demandes et afficher les panels selon les besoins de l’utilisateur.


\subsection{Déplacement dans un panel de l’application}
\label{\detokenize{fonctionnelle:deplacement-dans-un-panel-de-lapplication}}
\sphinxAtStartPar
L’application est conçue avec des “panels”, c’est\sphinxhyphen{}à\sphinxhyphen{}dire une liste de listes de controls. Cette liste de controls est propre à chaque panel. Les panels contiennent aussi une variable position\_x et position\_y qui permettent de connaître le control actuellement sélectionné par l’utilisateur. Quand l’utilisateur décide de se déplacer, il demande au panel de modifier ses variables x et y. Avant de valider ce changement, le panel regarde si le déplacement demandé par l’utilisateur est possible ou non.

\sphinxAtStartPar
Il existe 3 possibilités:
\begin{enumerate}
\sphinxsetlistlabels{\arabic}{enumi}{enumii}{}{.}%
\item {} 
\sphinxAtStartPar
Le déplacement est possible, alors la position sur l’axe x,y est modifiée.

\item {} 
\sphinxAtStartPar
Le déplacement est impossible car il n’y a rien à l’emplacement demandé. Dans ce cas, le panel va décider de bouger le curseur sur un des emplacements possibles.

\item {} 
\sphinxAtStartPar
L’utilisateur est à la fin du panel et “sors du panel” dans ce cas il va se diriger dans un autre panel s’il y en a un dans la direction demandée.

\end{enumerate}


\subsection{Déplacement de panel en panel}
\label{\detokenize{fonctionnelle:deplacement-de-panel-en-panel}}
\sphinxAtStartPar
Chaque “panel” possède un pointeur sur le panel du haut, du bas, de droite et de gauche.

\sphinxAtStartPar
Ces pointers ne sont pas forcément utilisés, ils ont le droit d’être nul.

\sphinxAtStartPar
Si on prend l’exemple suivant:

\sphinxAtStartPar
\sphinxincludegraphics{{exemple_3_panel}.jpeg}

\sphinxAtStartPar
Nous avons 3 panels différents qui contiennent chacun plusieurs controls.

\sphinxAtStartPar
Le panel 1 possède donc deux pointeurs différents, un sur le panel 2 et un autre sur le panel 3.

\sphinxAtStartPar
trop confus

\sphinxAtStartPar
Si l’utilisateur se trouve en bas du panel 1 et qu’il décide de se déplacer encore plus  bas, il ne pourra pas car aucune case n’est disponible dans ce panel. C’est pourquoi une vérification  sera faite pour savoir si un panel est indiqué comme le panel “down”, si tel est le cas le focus va changer de panel.

\sphinxAtStartPar
Un autre cas possible est que l’utilisateur va peut\sphinxhyphen{}être décider de retourner sur le panel 1. La contrainte est de savoir où doit pointer le panel haut du panel 3. Actuellement, un seul panel peut être défini par côté, mais la solution est de créer de petits panels pour éviter que ces situations arrivent.


\section{Serveur Debian 10}
\label{\detokenize{fonctionnelle:serveur-debian-10}}
\sphinxAtStartPar
J’ai utilisé un serveur sous Debian10 pour héberger le site web de caiman, l’API et les fichiers des différents jeux.


\subsection{Configuration de PHP}
\label{\detokenize{fonctionnelle:configuration-de-php}}
\sphinxAtStartPar
J’ai dû installer PHP pour le site web et l’API. Donc, je vais détailler la configuration que j’ai faite pour Caiman.
\begin{itemize}
\item {} 
\sphinxAtStartPar
version 7.4

\item {} 
\sphinxAtStartPar
max\_input\_time = 600

\item {} 
\sphinxAtStartPar
max\_execution\_time = 300

\item {} 
\sphinxAtStartPar
post\_max\_size = 8G

\item {} 
\sphinxAtStartPar
upload\_max\_filesize = 10G

\end{itemize}

\sphinxAtStartPar
J’ai dû grandement augmenter la taille d’upload pour pouvoir uploader les fichiers des jeux sur le serveur.


\subsection{Modification des droits des dossiers}
\label{\detokenize{fonctionnelle:modification-des-droits-des-dossiers}}
\sphinxAtStartPar
J’ai modifié les droits d’écriture pour le dossier où les images sont stockées et les dossiers où sont stockés les fichiers des jeux. Le dossier où sont stockées les sauvegardes des utilisateurs a lui aussi eu ces droits modifiés.

\sphinxAtStartPar
Chaque émulateur qui est supporté par caiman a un dossier ou les jeux compatibles sont stockés. Le dossier pour les sauvegardes contient les sauvegardes pour chaque émulateur. Il n’est donc pas nécessaire de créer plusieurs sous\sphinxhyphen{}dossiers.


\subsection{Services installé}
\label{\detokenize{fonctionnelle:services-installe}}
\sphinxAtStartPar
J’ai aussi installer un service FTP pour pouvoir transférer les fichiers sur le serveur plus facilement que par ssh. J’ai aussi installé mySQL et apache2.


\subsection{Création de virtual host}
\label{\detokenize{fonctionnelle:creation-de-virtual-host}}
\sphinxAtStartPar
Pour simplifier l’accès au site de Caiman et à l’API, les deux ne possèdent pas la même URL. Pour cela, j’ai dû configurer un sous\sphinxhyphen{}domaine pour l’API. Le nom de domaine de “base” est “caiman.cfpt.info” mais j’ai créé le sous\sphinxhyphen{}domaine “api.caiman.cfpt.info” pour les requêtes à l’API.

\sphinxAtStartPar
Le chemin où pointe “caiman.cfpt.info“ est “/var/www/caimanWeb/www” et le chemin pour “api.caiman.cfpt.info” est “/var/www/caimanAPI/public”. Pour créer les deux domaines j’ai dû modifier le fichier “/etc/apache2/sites\sphinxhyphen{}enabled/000\sphinxhyphen{}default.conf”. et ajouter les deux chemin et leur nom.


\chapter{Analyse organique}
\label{\detokenize{organique:analyse-organique}}\label{\detokenize{organique::doc}}

\section{Technologies utilisées}
\label{\detokenize{organique:technologies-utilisees}}

\subsection{HTML}
\label{\detokenize{organique:html}}

\subsection{CSS}
\label{\detokenize{organique:css}}

\subsection{PHP}
\label{\detokenize{organique:php}}
\sphinxAtStartPar
J’ai utilisé le php pour le site web de Caiman ainsi que pour l’API.


\subsection{C\#}
\label{\detokenize{organique:c}}
\sphinxAtStartPar
L’application Caiman a été développé en C\#, j’ai utilisé les paquets suivants:
\begin{itemize}
\item {} 
\sphinxAtStartPar
Microsoft.ASPNet.WEbApi.Client

\item {} 
\sphinxAtStartPar
Newtonsoft.Json

\item {} 
\sphinxAtStartPar
SharpDX

\end{itemize}


\section{Architecture du projet}
\label{\detokenize{organique:architecture-du-projet}}
\sphinxAtStartPar
Le projet se divise en 3 parties: l’application Caiman pour windows, le site web et l’API. Chaque partie du projet contient une documentation doxygene qui lui est propre.


\subsection{Caiman}
\label{\detokenize{organique:caiman}}
\sphinxAtStartPar
Je vais expliquer ci\sphinxhyphen{}dessous l’arborescence des fichiers de l’application Caiman et comment les émulateurs y sont intégrés.

\sphinxAtStartPar
\sphinxincludegraphics{{structure_caiman}.png}

\sphinxAtStartPar
Je vais expliquer ce que contiennent les différents dossiers.

\sphinxAtStartPar
\sphinxstylestrong{Database}

\sphinxAtStartPar
Contient les classes servant à appeler la base de données.

\sphinxAtStartPar
\sphinxstylestrong{interfaceG}

\sphinxAtStartPar
Ce dossier contient les fichiers de base de l’affichage.

\sphinxAtStartPar
\sphinxstylestrong{interfaceG\textbackslash{}usercontrol}

\sphinxAtStartPar
Ce dossier contient les différents panels qui sont utilisés dans Caiman pour créer l’affichage.

\sphinxAtStartPar
\sphinxstylestrong{interfaceG\textbackslash{}XboxControl}

\sphinxAtStartPar
Ce dossier contient une version modifiée de plusieurs contrôles de base de Windows Form.

\sphinxAtStartPar
\sphinxstylestrong{logique}

\sphinxAtStartPar
Ce dossier contient les classes servant à gérer les émulateurs et les sauvegardes et plus généralement toute la logique de l’application.

\sphinxAtStartPar
\sphinxstylestrong{models}

\sphinxAtStartPar
Ce dossier contient les modèles pour simplifier l’interaction entre la base de données et Caiman.

\sphinxAtStartPar
\sphinxstylestrong{emulators}

\sphinxAtStartPar
Le dossier est bien présent mais n’est pas visible sur la photo. Ce dossier contient tous les fichiers des émulateurs.


\subsection{Site web}
\label{\detokenize{organique:site-web}}
\sphinxAtStartPar
Le site web contient en plus des fichiers du site les fichiers des jeux et des sauvegardes des utilisateurs.

\sphinxAtStartPar
\sphinxincludegraphics{{dossier_caimanweb}.png}

\sphinxAtStartPar
\sphinxstylestrong{games}

\sphinxAtStartPar
Ce dossier contient dans un sous dossier spécifique à chaque émulateur les fichiers des jeux.

\sphinxAtStartPar
\sphinxstylestrong{release}

\sphinxAtStartPar
Ce dossier contient le fichier .zip de la version téléchargeable de Caiman depuis le site.

\sphinxAtStartPar
\sphinxstylestrong{saves}

\sphinxAtStartPar
Ce dossier contient les fichiers de sauvegardes des utilisateurs.

\sphinxAtStartPar
\sphinxstylestrong{www}

\sphinxAtStartPar
Ce dossier contient les fichiers du site web de Caiman.

\sphinxAtStartPar
\sphinxincludegraphics{{dossier_caimanweb_WWW}.png}

\sphinxAtStartPar
\sphinxstylestrong{common}

\sphinxAtStartPar
Ce dossier contient les fichiers de base de la vue  comme le footer ou le header.

\sphinxAtStartPar
\sphinxstylestrong{controllers}

\sphinxAtStartPar
Ce dossier contient les différents controllers de l’application.

\sphinxAtStartPar
\sphinxstylestrong{css}

\sphinxAtStartPar
contient les fichiers de css

\sphinxAtStartPar
\sphinxstylestrong{img}

\sphinxAtStartPar
contient les images des jeux

\sphinxAtStartPar
\sphinxstylestrong{models}

\sphinxAtStartPar
contient les fichiers pour simplifier l’accès à la base de données


\subsection{API}
\label{\detokenize{organique:api}}
\sphinxAtStartPar
\sphinxincludegraphics{{dossier_api_caiman}.png}

\sphinxAtStartPar
Le dossier de base de l’api contient deux dossiers importants .


\subsubsection{app}
\label{\detokenize{organique:app}}
\sphinxAtStartPar
\sphinxincludegraphics{{dossier_api_caiman_app}.png}

\sphinxAtStartPar
\sphinxstylestrong{Controllers}

\sphinxAtStartPar
Ce dossier contient les différents controllers de l’API.

\sphinxAtStartPar
\sphinxstylestrong{DataAccessObject}

\sphinxAtStartPar
Ce dossier contient les différents fichiers servant à exécuter des requêtes à la base de données.

\sphinxAtStartPar
\sphinxstylestrong{Models}

\sphinxAtStartPar
Ce dossier contient les modèles utilisés par l’API.

\sphinxAtStartPar
\sphinxstylestrong{System}

\sphinxAtStartPar
Ce dossier contient des fichiers servant à se connecter à la base de données.


\subsubsection{public}
\label{\detokenize{organique:public}}
\sphinxAtStartPar
Je ne vais pas développer davantage sur le contenu du dossier, en sachant que le détail des différents endpoint est disponible dans la documentation. Néaumoins, ce dossier comporte les points d’accès de l’API.


\section{Description technique: Site web}
\label{\detokenize{organique:description-technique-site-web}}

\subsection{Création de compte}
\label{\detokenize{organique:creation-de-compte}}
\sphinxAtStartPar
Caiman propose aux utilisateurs de se créer un compte.Se compte est commun au site internet et à l’application. Pour pouvoir se créer un compte, il faut renseigner plusieurs champs.
\begin{itemize}
\item {} 
\sphinxAtStartPar
un nom d’utilisateur

\item {} 
\sphinxAtStartPar
un mail

\item {} 
\sphinxAtStartPar
un mot de passe

\item {} 
\sphinxAtStartPar
un validation du mot de passe

\end{itemize}

\sphinxAtStartPar
Dans l’état actuel de l’application, l’émail n’est pas utilisé.

\begin{sphinxVerbatim}[commandchars=\\\{\}]
\PYG{x}{    // check if email alredy used}
\PYG{x}{    \PYGZdl{}sqlrequestEmail = \PYGZdq{}SELECT * FROM user WHERE email = :search\PYGZus{}email \PYGZdq{};}
\PYG{x}{    \PYGZdl{}this\PYGZhy{}\PYGZgt{}psCheckEmail = \PYGZdl{}this\PYGZhy{}\PYGZgt{}dbh\PYGZhy{}\PYGZgt{}prepare(\PYGZdl{}sqlrequestEmail);}
\PYG{x}{    \PYGZdl{}this\PYGZhy{}\PYGZgt{}psCheckEmail\PYGZhy{}\PYGZgt{}setFetchMode(PDO::FETCH\PYGZus{}ASSOC);}

\PYG{x}{    // check if username alredy used}
\PYG{x}{    \PYGZdl{}sqlRequestUsername = \PYGZdq{}SELECT * FROM user WHERE username = :search\PYGZus{}username \PYGZdq{};}
\PYG{x}{    \PYGZdl{}this\PYGZhy{}\PYGZgt{}psCheckUsername = \PYGZdl{}this\PYGZhy{}\PYGZgt{}dbh\PYGZhy{}\PYGZgt{}prepare(\PYGZdl{}sqlRequestUsername);}
\PYG{x}{    \PYGZdl{}this\PYGZhy{}\PYGZgt{}psCheckUsername\PYGZhy{}\PYGZgt{}setFetchMode(PDO::FETCH\PYGZus{}ASSOC);}

\PYG{x}{    \PYGZdl{}sqlInsert = \PYGZdq{}INSERT INTO user  (username, password, email, salt,apitocken)}
\PYG{x}{    }
\PYG{x}{    VALUES (:insert\PYGZus{}username, :insert\PYGZus{}password, :insert\PYGZus{}email, :insert\PYGZus{}salt, :insert\PYGZus{}apiTocken)\PYGZdq{};}
\PYG{x}{    \PYGZdl{}this\PYGZhy{}\PYGZgt{}psInsert = \PYGZdl{}this\PYGZhy{}\PYGZgt{}dbh\PYGZhy{}\PYGZgt{}prepare(\PYGZdl{}sqlInsert);}
\PYG{x}{    \PYGZdl{}this\PYGZhy{}\PYGZgt{}psInsert\PYGZhy{}\PYGZgt{}setFetchMode(PDO::FETCH\PYGZus{}ASSOC);}
\end{sphinxVerbatim}


\subsection{Connexion}
\label{\detokenize{organique:connexion}}
\sphinxAtStartPar
Pour se connecter au site web de caiman, il faut renseigner son nom d’utilisateur et son mot de passe.

\begin{sphinxVerbatim}[commandchars=\\\{\}]
\PYG{x}{ \PYGZdl{}this\PYGZhy{}\PYGZgt{}psLogin\PYGZhy{}\PYGZgt{}execute(array(\PYGZsq{}:search\PYGZus{}username\PYGZsq{} =\PYGZgt{} \PYGZdl{}this\PYGZhy{}\PYGZgt{}search\PYGZus{}username));}
\PYG{x}{            \PYGZdl{}result = \PYGZdl{}this\PYGZhy{}\PYGZgt{}psLogin\PYGZhy{}\PYGZgt{}fetchAll();}
\PYG{x}{            if (\PYGZdl{}result != null) \PYGZob{}}
\PYG{x}{                if (md5(\PYGZdl{}result[0][\PYGZdq{}salt\PYGZdq{}].\PYGZdl{}password\PYGZus{}verify) == \PYGZdl{}result[0][\PYGZdq{}password\PYGZdq{}]  ) \PYGZob{}}
\PYG{x}{                    \PYGZdl{}returnArray = \PYGZdl{}result;}
\PYG{x}{                    \PYGZdl{}\PYGZus{}SESSION[\PYGZsq{}error\PYGZsq{}] = \PYGZdq{}Welcome back: \PYGZdq{}. \PYGZdl{}result[0][\PYGZsq{}username\PYGZsq{}];}
\PYG{x}{                \PYGZcb{}else}
\PYG{x}{                \PYGZob{}}
\PYG{x}{                    \PYGZdl{}\PYGZus{}SESSION[\PYGZsq{}error\PYGZsq{}] = \PYGZdq{}Invalid log in\PYGZdq{};}
\PYG{x}{                \PYGZcb{}}
\PYG{x}{            \PYGZcb{}}
\end{sphinxVerbatim}


\subsection{Fonctionnalité disponible une fois connecté}
\label{\detokenize{organique:fonctionnalite-disponible-une-fois-connecte}}
\sphinxAtStartPar
Quand un utilisateur se connecte à l’application, il a accès à de nouvelles fonctionnalités. L’utilisateur a maintenant la possibilité de modifier son mot de passe, ajouter/supprimer des jeux de ses favoris et de modifier la visibilité de son profil (si le profil est visible, il sera affiché dans la liste des utilisateurs).

\begin{sphinxVerbatim}[commandchars=\\\{\}]
\PYG{x}{public function updatePassword(string \PYGZdl{}newPassword, string \PYGZdl{}newPasswordRepeat, string \PYGZdl{}oldPassword)}
\PYG{x}{    \PYGZob{}}
\PYG{x}{        \PYGZdl{}dbh = new PDO(\PYGZsq{}mysql:host=\PYGZsq{} . HOST . \PYGZsq{};dbname=\PYGZsq{} . DBNAME, USER, PASSWORD, array(}
\PYG{x}{            PDO::MYSQL\PYGZus{}ATTR\PYGZus{}INIT\PYGZus{}COMMAND =\PYGZgt{} \PYGZdq{}SET NAMES utf8\PYGZdq{},}
\PYG{x}{            PDO::ATTR\PYGZus{}PERSISTENT =\PYGZgt{} true}
\PYG{x}{        ));}

\PYG{x}{        \PYGZdl{}hasBeenUpdated = 1;}
\PYG{x}{        if (md5(\PYGZdl{}\PYGZus{}SESSION[\PYGZsq{}user\PYGZsq{}]\PYGZhy{}\PYGZgt{}salt.\PYGZdl{}oldPassword) == \PYGZdl{}this\PYGZhy{}\PYGZgt{}getUserPassword()) \PYGZob{}}

\PYG{x}{            if (\PYGZdl{}newPasswordRepeat == \PYGZdl{}newPassword) \PYGZob{}}
\PYG{x}{                try \PYGZob{}}
\PYG{x}{                    \PYGZdl{}salt = rand(1,10000);}
\PYG{x}{                    \PYGZdl{}sqlUpdatePassword = \PYGZdq{}UPDATE user  SET password = :update\PYGZus{}password, salt = :update\PYGZus{}salt WHERE id = :id\PYGZus{}user\PYGZdq{};}
\PYG{x}{                    \PYGZdl{}psUpdatePassword = \PYGZdl{}dbh\PYGZhy{}\PYGZgt{}prepare(\PYGZdl{}sqlUpdatePassword);}
\PYG{x}{                    \PYGZdl{}psUpdatePassword\PYGZhy{}\PYGZgt{}execute(array(\PYGZsq{}:update\PYGZus{}password\PYGZsq{} =\PYGZgt{} md5(\PYGZdl{}salt.\PYGZdl{}newPassword), \PYGZsq{}:id\PYGZus{}user\PYGZsq{} =\PYGZgt{} \PYGZdl{}this\PYGZhy{}\PYGZgt{}idUser, \PYGZsq{}:update\PYGZus{}salt\PYGZsq{} =\PYGZgt{} \PYGZdl{}salt));}

\PYG{x}{                    \PYGZdl{}hasBeenUpdated = 0;}
\PYG{x}{                \PYGZcb{} catch (PDOException \PYGZdl{}e) \PYGZob{}}
\PYG{x}{                    print \PYGZdq{}Erreur !: \PYGZdq{} . \PYGZdl{}e\PYGZhy{}\PYGZgt{}getMessage() . \PYGZdq{}}\PYG{x}{\PYGZlt{}}\PYG{x}{br\PYGZgt{}\PYGZdq{};}
\PYG{x}{                    die();}
\PYG{x}{                \PYGZcb{}}
\PYG{x}{            \PYGZcb{} else \PYGZob{}}
\PYG{x}{                \PYGZdl{}hasBeenUpdated = 2;}
\PYG{x}{            \PYGZcb{}}
\PYG{x}{        \PYGZcb{} else \PYGZob{}}
\PYG{x}{            \PYGZdl{}hasBeenUpdated = 4;}
\PYG{x}{        \PYGZcb{}}

\PYG{x}{        return \PYGZdl{}hasBeenUpdated;}
\PYG{x}{    \PYGZcb{}}
\end{sphinxVerbatim}


\subsection{Ajout de jeu}
\label{\detokenize{organique:ajout-de-jeu}}
\sphinxAtStartPar
L’ajout de jeu ne peut se faire que par le site et seulement par les administrateurs du site.

\sphinxAtStartPar
Pour pouvoir ajouter un jeu, il faut au préalable avoir sur le disque dur de l’administrateur un fichier .iso du jeu, il faut aussi envoyer une image. Cette image sera affichée sur le site web et dans l’application Caiman.

\sphinxAtStartPar
Pour pouvoir envoyer le formulaire, il faut remplir les champs suivants:
\begin{itemize}
\item {} 
\sphinxAtStartPar
nom du jeu

\item {} 
\sphinxAtStartPar
description

\item {} 
\sphinxAtStartPar
nom de l’image(le nom doit respecter un certain format)

\item {} 
\sphinxAtStartPar
le fichier de l’image

\item {} 
\sphinxAtStartPar
la console sur lequel le jeu tourne

\item {} 
\sphinxAtStartPar
le nom du fichier du jeu(le nom doit respecter un certain format)

\item {} 
\sphinxAtStartPar
le fichier du jeu

\end{itemize}

\sphinxAtStartPar
Le serveur est configuré pour recevoir des fichiers jusqu’à 8GB. Donc pour les fichiers de Gamecube qui ne peuvent faire que 1.4GB et les fichiers de Playstation 2 qui peuvent faire jusqu’à 7 GB, il n’y a pas de problèmes.

\begin{sphinxVerbatim}[commandchars=\\\{\}]
\PYG{x}{public function uploadGame(\PYGZdl{}gameFileName, \PYGZdl{}consoleId)}
\PYG{x}{    \PYGZob{}}
\PYG{x}{        \PYGZdl{}uploadIsValid = false;}
\PYG{x}{        \PYGZdl{}target\PYGZus{}dir = \PYGZdq{}../games/\PYGZdq{} . \PYGZdl{}this\PYGZhy{}\PYGZgt{}getConsoleFolderName(\PYGZdl{}consoleId) . \PYGZdq{}/\PYGZdq{};}

\PYG{x}{        \PYGZdl{}target\PYGZus{}file =  basename(\PYGZdl{}\PYGZus{}FILES[\PYGZdq{}fileGame\PYGZdq{}][\PYGZdq{}name\PYGZdq{}]);}
\PYG{x}{        \PYGZdl{}uploadOk = 1;}
\PYG{x}{        \PYGZdl{}fileType = strtolower(pathinfo(\PYGZdl{}target\PYGZus{}file, PATHINFO\PYGZus{}EXTENSION));}

\PYG{x}{        //rename file}
\PYG{x}{        \PYGZdl{}newfilename = \PYGZdl{}gameFileName ;}


\PYG{x}{        // Check if file already exists}
\PYG{x}{        if (file\PYGZus{}exists(\PYGZdl{}target\PYGZus{}file)) \PYGZob{}}
\PYG{x}{            echo \PYGZdq{}Sorry, file already exists.\PYGZdq{};}
\PYG{x}{            \PYGZdl{}uploadOk = 0;}
\PYG{x}{        \PYGZcb{}}
\PYG{x}{        if (\PYGZdl{}uploadOk == 0) \PYGZob{}}
\PYG{x}{            echo \PYGZdq{}Sorry, your file was not uploaded.\PYGZdq{};}
\PYG{x}{            // if everything is ok, try to upload file}
\PYG{x}{        \PYGZcb{} else \PYGZob{}}
\PYG{x}{            if (move\PYGZus{}uploaded\PYGZus{}file(\PYGZdl{}\PYGZus{}FILES[\PYGZdq{}fileGame\PYGZdq{}][\PYGZdq{}tmp\PYGZus{}name\PYGZdq{}], \PYGZdl{}target\PYGZus{}dir . \PYGZdl{}newfilename)) \PYGZob{}}
\PYG{x}{                \PYGZdl{}uploadIsValid = true;}
\PYG{x}{            \PYGZcb{} else \PYGZob{}}
\PYG{x}{                //Sorry, there was an error uploading your file}
\PYG{x}{            \PYGZcb{}}
\PYG{x}{        \PYGZcb{}}
\PYG{x}{        return \PYGZdl{}uploadIsValid;}
\PYG{x}{    \PYGZcb{}}
\end{sphinxVerbatim}


\subsection{Ajout / suppression de catégories au jeux}
\label{\detokenize{organique:ajout-suppression-de-categories-au-jeux}}
\sphinxAtStartPar
Chaque jeu peut appartenir à une ou plusieurs catégories. Seuls les administrateurs ont le droit d’ajouter des catégories aux jeux.

\sphinxAtStartPar
Pour pouvoir modifier les catégories d’un jeu, il faut qu’un administrateur aille sur la page du jeu qui doit être modifiée.

\begin{sphinxVerbatim}[commandchars=\\\{\}]
\PYG{x}{public function addCategorieToGame(int \PYGZdl{}idGame, int \PYGZdl{}idCategorie)}
\PYG{x}{    \PYGZob{}}
\PYG{x}{        \PYGZdl{}result = null;}
\PYG{x}{        try \PYGZob{}}
\PYG{x}{            \PYGZdl{}this\PYGZhy{}\PYGZgt{}psCheckIfGameHasCategorie\PYGZhy{}\PYGZgt{}execute(array(\PYGZsq{}:insert\PYGZus{}idCategorie\PYGZsq{} =\PYGZgt{} \PYGZdl{}idCategorie, \PYGZsq{}:insert\PYGZus{}idGame\PYGZsq{} =\PYGZgt{} \PYGZdl{}idGame));}
\PYG{x}{            \PYGZdl{}result = \PYGZdl{}this\PYGZhy{}\PYGZgt{}psCheckIfGameHasCategorie\PYGZhy{}\PYGZgt{}fetchAll();}
\PYG{x}{        \PYGZcb{} catch (PDOException \PYGZdl{}e) \PYGZob{}}
\PYG{x}{            print \PYGZdq{}Erreur !: \PYGZdq{} . \PYGZdl{}e\PYGZhy{}\PYGZgt{}getMessage() . \PYGZdq{}}\PYG{x}{\PYGZlt{}}\PYG{x}{br\PYGZgt{}\PYGZdq{};}
\PYG{x}{            die();}
\PYG{x}{        \PYGZcb{}}
\PYG{x}{        if (\PYGZdl{}result == null) \PYGZob{}}
\PYG{x}{            try \PYGZob{}}
\PYG{x}{                \PYGZdl{}this\PYGZhy{}\PYGZgt{}psAddCategorieToGame\PYGZhy{}\PYGZgt{}execute(array(\PYGZsq{}:insert\PYGZus{}idCategorie\PYGZsq{} =\PYGZgt{} \PYGZdl{}idCategorie, \PYGZsq{}:insert\PYGZus{}idGame\PYGZsq{} =\PYGZgt{} \PYGZdl{}idGame));}
\PYG{x}{            \PYGZcb{} catch (PDOException \PYGZdl{}e) \PYGZob{}}
\PYG{x}{                print \PYGZdq{}Erreur !: \PYGZdq{} . \PYGZdl{}e\PYGZhy{}\PYGZgt{}getMessage() . \PYGZdq{}}\PYG{x}{\PYGZlt{}}\PYG{x}{br\PYGZgt{}\PYGZdq{};}
\PYG{x}{                die();}
\PYG{x}{            \PYGZcb{}}
\PYG{x}{        \PYGZcb{}}
\PYG{x}{    \PYGZcb{}}
\end{sphinxVerbatim}


\subsection{Ajout de catégories}
\label{\detokenize{organique:ajout-de-categories}}
\sphinxAtStartPar
La liste des catégories est définie par les administrateurs, ils ont la possibilité d’en ajouter depuis le menu d’administration. Pour ajouter une catégorie, il suffit de spécifier le nom que va porter la nouvelle catégorie.

\begin{sphinxVerbatim}[commandchars=\\\{\}]
\PYG{x}{public function addCategorie(string \PYGZdl{}categorieName)}
\PYG{x}{    \PYGZob{}}
\PYG{x}{        try \PYGZob{}}
\PYG{x}{            \PYGZdl{}this\PYGZhy{}\PYGZgt{}psAddCategorie\PYGZhy{}\PYGZgt{}execute(array(\PYGZsq{}:categorie\PYGZus{}name\PYGZsq{} =\PYGZgt{} \PYGZdl{}categorieName));}
\PYG{x}{            \PYGZdl{}result = \PYGZdl{}this\PYGZhy{}\PYGZgt{}psAddCategorie\PYGZhy{}\PYGZgt{}fetchAll();}
\PYG{x}{        \PYGZcb{} catch (PDOException \PYGZdl{}e) \PYGZob{}}
\PYG{x}{            print \PYGZdq{}Erreur !: \PYGZdq{} . \PYGZdl{}e\PYGZhy{}\PYGZgt{}getMessage() . \PYGZdq{}}\PYG{x}{\PYGZlt{}}\PYG{x}{br\PYGZgt{}\PYGZdq{};}
\PYG{x}{            die();}
\PYG{x}{        \PYGZcb{}}
\PYG{x}{        return \PYGZdl{}result;}
\PYG{x}{    \PYGZcb{}}
\end{sphinxVerbatim}


\subsection{Téléchargement de Caiman}
\label{\detokenize{organique:telechargement-de-caiman}}
\sphinxAtStartPar
Le téléchargement de Caiman se fait sous forme de fichier .zip. Pour pouvoir télécharger l’application, il faut être authentifié. Actuellement, pour pouvoir modifier le fichier caiman.zip que l’utilisateur télécharge, il faut passer par le FTP. Le site web ne permet pas de le mettre à jour.

\begin{sphinxVerbatim}[commandchars=\\\{\}]
\PYG{x}{public function downloadCaiman()}
\PYG{x}{    \PYGZob{}}

\PYG{x}{        \PYGZdl{}zipFile = \PYGZdq{}../release/Caiman.zip\PYGZdq{};}

\PYG{x}{        \PYGZdl{}file\PYGZus{}name = basename(\PYGZdl{}zipFile);}

\PYG{x}{        header(\PYGZdq{}Content\PYGZhy{}Type: application/zip\PYGZdq{});}
\PYG{x}{        header(\PYGZdq{}Content\PYGZhy{}Disposition: attachment; filename=\PYGZdl{}file\PYGZus{}name\PYGZdq{});}
\PYG{x}{        header(\PYGZdq{}Content\PYGZhy{}Length: \PYGZdq{} . filesize(\PYGZdl{}zipFile));}

\PYG{x}{        readfile(\PYGZdl{}zipFile);}
\PYG{x}{        exit;}
\PYG{x}{    \PYGZcb{}}
\PYG{x}{\PYGZcb{}}
\end{sphinxVerbatim}


\subsection{Modification des jeux}
\label{\detokenize{organique:modification-des-jeux}}
\sphinxAtStartPar
Actuellement, il est possible de modifier le nom et la description d’un jeu.


\section{Description technique: API}
\label{\detokenize{organique:description-technique-api}}

\subsection{Téléchargement de jeu}
\label{\detokenize{organique:telechargement-de-jeu}}
\sphinxAtStartPar
Les jeux sont sous forme de fichier .iso dans le dossier “caimanWeb\textbackslash{}games\textbackslash{}”.Par contre, ce n’est pas tout pour simplifier, il faut bien séparer les jeux des différentes consoles. J’ai décidé de créer des dossiers par émulateurs. C’est pourquoi il va y avoir des sous\sphinxhyphen{}dossiers “GamecubeWii\textbackslash{}” et “Playstation2\textbackslash{}”. Chaque jeu est stocké avec le nom que l’administrateur lui aura donné quand il l’a ajouté depuis le site web.

\sphinxAtStartPar
Le chemin sur le serveur web pour accéder au fichier des jeux n’est pas public. Par conséquent, il a donc fallu que je trouve une solution pour que n’importe qui ne puisse pas télécharger un jeu. Pour ce faire, j’ai créé une route dans mon API qui prend en paramètres l’id du jeu et l’apiKey de l’utilisateur qui veut télécharger le jeu.

\sphinxAtStartPar
La fonction getURL(idGame,apiKey) permet de recevoir le lien de téléchargement pour un jeu. Avant de valider le téléchargement, l’API vérifie que l’apiKey qui lui a été donné est valide. Pour savoir si l’apiKey est valide, j’utilise la fonction DAOUser\sphinxhyphen{}>Find(apikey) qui me retourne l’utilisateur en lien avec cette apiKey. Si l’apiKey est valide, je vais donc devoir reconstituer le chemin vers le dossier ou est le jeu stocké.

\sphinxAtStartPar
Pour connaître le nom du dossier où se trouve le fichier, il faut savoir de quelle console est le jeu. Pour connaître cette information, je dois rechercher un jeu grâce à son id. Quand j’ai l’id de la console en lien avec le jeu, je dois encore faire une recherche pour savoir quel est son nom de dossier dans la base de données. Alors j’utilise la fonction DAOConsole\sphinxhyphen{}>find(idConsole) pour connaître son nom de dossier.

\sphinxAtStartPar
Désormais que nous avons le dossier dans lequel se trouve le fichier du jeu, il faut maintenant connaître son nom de fichier. Pour cela, j’utilise la fonction DAOFile\sphinxhyphen{}>find(file). l’id du fichier est connu grâce à l’appel la fonction DAOGame\sphinxhyphen{}>find().

\sphinxAtStartPar
 présent que nous avons toutes les parties du chemin, il est possible de construire le chemin complet en concaténant toutes les parties. Ensuite pour renvoyer le jeu à l’utilisateur, j’utilise les fonctions fopen(path,’rb’) et fpassthru(fopen).

\begin{sphinxVerbatim}[commandchars=\\\{\}]
\PYG{x}{public function getURL(int \PYGZdl{}idGame, string \PYGZdl{}apikey)}
\PYG{x}{    \PYGZob{}}

\PYG{x}{        \PYGZdl{}user = \PYGZdl{}this\PYGZhy{}\PYGZgt{}DAOUser\PYGZhy{}\PYGZgt{}find(\PYGZdl{}apikey);}
\PYG{x}{        \PYGZdl{}fullpath = \PYGZdq{}\PYGZdq{};}
\PYG{x}{        if (is\PYGZus{}null(\PYGZdl{}user)) \PYGZob{}}
\PYG{x}{            return ResponseController::notFoundResponse();}
\PYG{x}{        \PYGZcb{} else \PYGZob{}}
\PYG{x}{            \PYGZdl{}game = \PYGZdl{}this\PYGZhy{}\PYGZgt{}DAOGame\PYGZhy{}\PYGZgt{}find(\PYGZdl{}idGame);}
\PYG{x}{            \PYGZdl{}file = \PYGZdl{}this\PYGZhy{}\PYGZgt{}DAOFile\PYGZhy{}\PYGZgt{}find(\PYGZdl{}game\PYGZhy{}\PYGZgt{}idFile);}
\PYG{x}{            \PYGZdl{}console = \PYGZdl{}this\PYGZhy{}\PYGZgt{}DAOConsole\PYGZhy{}\PYGZgt{}find(\PYGZdl{}game\PYGZhy{}\PYGZgt{}idConsole);}
\PYG{x}{            \PYGZdl{}fullpath = \PYGZdq{}../../../../caimanWeb/games/\PYGZdq{} . \PYGZdl{}console\PYGZhy{}\PYGZgt{}folderName . \PYGZdq{}/\PYGZdq{} . \PYGZdl{}file\PYGZhy{}\PYGZgt{}filename;}


\PYG{x}{            if (file\PYGZus{}exists(\PYGZdl{}fullpath)) \PYGZob{}}
\PYG{x}{                header(\PYGZsq{}Content\PYGZhy{}Description: File Transfer\PYGZsq{});}
\PYG{x}{                header(\PYGZsq{}Content\PYGZhy{}Type: application/octet\PYGZhy{}stream\PYGZsq{});}
\PYG{x}{                header(\PYGZsq{}Content\PYGZhy{}Disposition: attachment; filename=\PYGZsq{} . basename(\PYGZdl{}fullpath));}
\PYG{x}{                header(\PYGZsq{}Content\PYGZhy{}Transfer\PYGZhy{}Encoding: binary\PYGZsq{});}
\PYG{x}{                header(\PYGZsq{}Expires: 0\PYGZsq{});}
\PYG{x}{                header(\PYGZsq{}Cache\PYGZhy{}Control: must\PYGZhy{}revalidate, post\PYGZhy{}check=0, pre\PYGZhy{}check=0\PYGZsq{});}
\PYG{x}{                header(\PYGZsq{}Pragma: public\PYGZsq{});}
\PYG{x}{                header(\PYGZsq{}Content\PYGZhy{}Length: \PYGZsq{} . filesize(\PYGZdl{}fullpath));}
\PYG{x}{                \PYGZdl{}fp = fopen(\PYGZdl{}fullpath, \PYGZsq{}rb\PYGZsq{});}
\PYG{x}{                fpassthru(\PYGZdl{}fp);}
\PYG{x}{                exit;}
\PYG{x}{            \PYGZcb{}}
\PYG{x}{        \PYGZcb{}}

\PYG{x}{        return ResponseController::successfulRequest(\PYGZdl{}fullpath);}
\PYG{x}{    \PYGZcb{}}
\end{sphinxVerbatim}


\subsection{Téléchargement de sauvegarde}
\label{\detokenize{organique:telechargement-de-sauvegarde}}
\sphinxAtStartPar
Les sauvegardes des utilisateurs sont stockées sous forme de fichiers .zip. L’intérêt d’utiliser des fichiers .zip est la taille et le fait qu’un fichier puisse contenir toutes les sauvegardes instantanément. Les fichiers de sauvegardes se trouvent dans le dossier “\textbackslash{}CaimanWeb\textbackslash{}saves\textbackslash{}”.

\sphinxAtStartPar
Le nom attribué au fichier est décidé au premier envoie de sauvegardes, le nom est le md5 du microtime de l’heure d’envoie.

\sphinxAtStartPar
Pour recevoir le fichier, l’API possède une route qui prend les paramètres suivants:
\begin{itemize}
\item {} 
\sphinxAtStartPar
idEmulator

\item {} 
\sphinxAtStartPar
iduser

\item {} 
\sphinxAtStartPar
apikey

\end{itemize}

\sphinxAtStartPar
Cette route va envoyer le fichier demandé. Comme pour les fichiers des jeux, les sauvegardes sont dans un dossier privé du serveur. La fonction getURLsave( idEmulator, Iduser,apikey) se charge de renvoyer les fichiers à l’utilisateur. Pour faire ça il faut connaître le nom du fichier qui doit être envoyé. Il faut donc utiliser la fonction DAOFileSave\sphinxhyphen{}>find( idemulator, idUser) pour connaître les informations du fichier.

\sphinxAtStartPar
Désormais que le nom du fichier est connu, il est possible de retourner le fichier grâce à la fonction fopen(path,’rb’) et fpassthru(fopen) pour renvoyer le fichier à l’utilisateur.

\begin{sphinxVerbatim}[commandchars=\\\{\}]
\PYG{x}{public function getURLSave(int \PYGZdl{}idEmulator, int \PYGZdl{}idUser, string \PYGZdl{}apikey)}
\PYG{x}{    \PYGZob{}}

\PYG{x}{        \PYGZdl{}user = \PYGZdl{}this\PYGZhy{}\PYGZgt{}DAOUser\PYGZhy{}\PYGZgt{}find(\PYGZdl{}apikey);}
\PYG{x}{        \PYGZdl{}fullpath = \PYGZdq{}\PYGZdq{};}
\PYG{x}{        if (is\PYGZus{}null(\PYGZdl{}user)) \PYGZob{}}
\PYG{x}{            return ResponseController::notFoundResponse();}
\PYG{x}{        \PYGZcb{} else \PYGZob{}}
\PYG{x}{            \PYGZdl{}fileSave = \PYGZdl{}this\PYGZhy{}\PYGZgt{}DAOFileSave\PYGZhy{}\PYGZgt{}find(\PYGZdl{}idEmulator, \PYGZdl{}idUser);}

\PYG{x}{            if (is\PYGZus{}null(\PYGZdl{}fileSave)) \PYGZob{}}
\PYG{x}{                return ResponseController::notFoundResponse();}
\PYG{x}{                exit;}
\PYG{x}{            \PYGZcb{}}
\PYG{x}{            \PYGZdl{}file = \PYGZdl{}this\PYGZhy{}\PYGZgt{}DAOFile\PYGZhy{}\PYGZgt{}find(\PYGZdl{}fileSave\PYGZhy{}\PYGZgt{}idFile);}
\PYG{x}{            \PYGZdl{}fullpath = \PYGZdq{}../../../../caimanWeb/saves/\PYGZdq{} . \PYGZdl{}file\PYGZhy{}\PYGZgt{}filename;}


\PYG{x}{            if (file\PYGZus{}exists(\PYGZdl{}fullpath)) \PYGZob{}}
\PYG{x}{                header(\PYGZsq{}Content\PYGZhy{}Description: File Transfer\PYGZsq{});}
\PYG{x}{                header(\PYGZsq{}Content\PYGZhy{}Type: application/zip\PYGZsq{});}
\PYG{x}{                header(\PYGZsq{}Content\PYGZhy{}Disposition: attachment; filename=\PYGZsq{} . basename(\PYGZdl{}fullpath));}
\PYG{x}{                header(\PYGZsq{}Content\PYGZhy{}Transfer\PYGZhy{}Encoding: binary\PYGZsq{});}
\PYG{x}{                header(\PYGZsq{}Expires: 0\PYGZsq{});}
\PYG{x}{                header(\PYGZsq{}Cache\PYGZhy{}Control: must\PYGZhy{}revalidate, post\PYGZhy{}check=0, pre\PYGZhy{}check=0\PYGZsq{});}
\PYG{x}{                header(\PYGZsq{}Pragma: public\PYGZsq{});}
\PYG{x}{                header(\PYGZsq{}Content\PYGZhy{}Length: \PYGZsq{} . filesize(\PYGZdl{}fullpath));}
\PYG{x}{                \PYGZdl{}fp = fopen(\PYGZdl{}fullpath, \PYGZsq{}rb\PYGZsq{});}
\PYG{x}{                fpassthru(\PYGZdl{}fp);}
\PYG{x}{                exit;}
\PYG{x}{            \PYGZcb{}}
\PYG{x}{        \PYGZcb{}}

\PYG{x}{        return ResponseController::successfulRequest();}
\PYG{x}{    \PYGZcb{}}
\end{sphinxVerbatim}

\sphinxAtStartPar
Dans l’état actuel de l’applicationCaiman, il est impossible de télécharger seulement la sauvegarde d’un seul jeu, la structure des sauvegardes des émulateurs ne laisse pas vraiment le choix.


\subsection{Upload de sauvegarde}
\label{\detokenize{organique:upload-de-sauvegarde}}
\sphinxAtStartPar
Pour enregistrer les sauvegardes, il y a une route dans l’API qui prend en paramètres:
\begin{itemize}
\item {} 
\sphinxAtStartPar
idEmulator

\item {} 
\sphinxAtStartPar
iduser

\item {} 
\sphinxAtStartPar
apikey

\item {} 
\sphinxAtStartPar
file

\end{itemize}

\sphinxAtStartPar
Les fichiers de sauvegarde concernent un émulateur àchaque fois, donc il faut le spécifier à l’envoi. Il faut aussi spécifier l’utilisateur à qui appartiennent ces sauvegardes, il faut aussi transmettre une apiKey valide et finalement le fichier à envoyer.

\sphinxAtStartPar
Quand l’API reçoit ces informations, il faut déjà vérifier si l’apiKey reçu est bien valide. Si elle est valide, il faut ensuite vérifier si l’utilisateur a déjà créé une sauvegarde pour cet émulateur. Pour effectuer cette vérification, il faut utiliser la fonction DAOFileSave\sphinxhyphen{}>find(idEmulator,idUser) qui retourne les informations du fichier s’il existe.

\begin{sphinxVerbatim}[commandchars=\\\{\}]
\PYG{x}{public function AddSave(\PYGZdl{}idEmulator, \PYGZdl{}idUser, \PYGZdl{}apiKey, \PYGZdl{}file)}
\PYG{x}{    \PYGZob{}}

\PYG{x}{        if (is\PYGZus{}null(\PYGZdl{}idEmulator) || is\PYGZus{}null(\PYGZdl{}idUser) || is\PYGZus{}null(\PYGZdl{}apiKey)) \PYGZob{}}
\PYG{x}{            return ResponseController::notFoundResponse();}
\PYG{x}{            exit;}
\PYG{x}{        \PYGZcb{}}
\PYG{x}{        \PYGZdl{}user = \PYGZdl{}this\PYGZhy{}\PYGZgt{}DAOUser\PYGZhy{}\PYGZgt{}find(\PYGZdl{}apiKey);}

\PYG{x}{        if (\PYGZdl{}user == null) \PYGZob{}}
\PYG{x}{            return ResponseController::notFoundResponse();}
\PYG{x}{            exit;}
\PYG{x}{        \PYGZcb{}}
\PYG{x}{        \PYGZdl{}isNewFile = false;}
\PYG{x}{        \PYGZdl{}newfilename = \PYGZdl{}this\PYGZhy{}\PYGZgt{}DAOFileSave\PYGZhy{}\PYGZgt{}FindFileName(\PYGZdl{}idEmulator, \PYGZdl{}idUser);}

\PYG{x}{        if (\PYGZdl{}newfilename == null) \PYGZob{}}
\PYG{x}{            \PYGZdl{}newfilename = md5(microtime());}
\PYG{x}{            \PYGZdl{}newfilename = \PYGZdl{}newfilename . \PYGZdq{}.zip\PYGZdq{};}
\PYG{x}{            \PYGZdl{}isNewFile = true;}
\PYG{x}{        \PYGZcb{}}
\PYG{x}{        \PYGZdl{}target\PYGZus{}dir = \PYGZdq{}../../../../caimanWeb/saves/\PYGZdq{};}

\PYG{x}{        if (move\PYGZus{}uploaded\PYGZus{}file(\PYGZdl{}file, \PYGZdl{}target\PYGZus{}dir . \PYGZdl{}newfilename)) \PYGZob{}}
\PYG{x}{            if (\PYGZdl{}isNewFile) \PYGZob{}}
\PYG{x}{                \PYGZdl{}this\PYGZhy{}\PYGZgt{}DAOFileSave\PYGZhy{}\PYGZgt{}AddFileSave(\PYGZdl{}idEmulator, \PYGZdl{}idUser, \PYGZdl{}newfilename);}
\PYG{x}{            \PYGZcb{}}
\PYG{x}{        \PYGZcb{} else \PYGZob{}}
\PYG{x}{            return ResponseController::uploadFailed();}
\PYG{x}{            exit;}
\PYG{x}{        \PYGZcb{}}

\PYG{x}{        return ResponseController::successfulRequest();}
\PYG{x}{    \PYGZcb{}}
\end{sphinxVerbatim}


\subsubsection{Sauvegarde déjà présente}
\label{\detokenize{organique:sauvegarde-deja-presente}}
\sphinxAtStartPar
Si la sauvegarde d’un émulateur est déjà présente, il faut utiliser la fonction move\_uploaded\_file(file,target\_dir). Tout d’abord il faut connaître le nom du fichier qui est attribué à la sauvegarde. Pour faire cela, il faut appeler la fonction DAOFileSave\sphinxhyphen{}>findFilename(idEmulator,idUser) celle\sphinxhyphen{}ci va renvoyer le nom du fichier.Il suffit maintenant d’écraser le fichier présent sur le serveur par le nouveau.

\begin{sphinxVerbatim}[commandchars=\\\{\}]
\PYG{x}{    public function findFileName(int \PYGZdl{}idEmulator, int \PYGZdl{}idUser)}
\PYG{x}{    \PYGZob{}}

\PYG{x}{        \PYGZdl{}statement = \PYGZdq{}}
\PYG{x}{    SELECT f.filename filesave FROM `filesave` as fs}
\PYG{x}{                LEFT JOIN file as f}
\PYG{x}{                ON fs.idFile = f.id}
\PYG{x}{                WHERE idUser = :ID\PYGZus{}user AND idEmulator = :ID\PYGZus{}emulator;\PYGZdq{};}
\PYG{x}{        try \PYGZob{}}
\PYG{x}{            \PYGZdl{}statement = \PYGZdl{}this\PYGZhy{}\PYGZgt{}db\PYGZhy{}\PYGZgt{}prepare(\PYGZdl{}statement);}
\PYG{x}{            \PYGZdl{}statement\PYGZhy{}\PYGZgt{}bindParam(\PYGZsq{}:ID\PYGZus{}user\PYGZsq{}, \PYGZdl{}idUser, \PYGZbs{}PDO::PARAM\PYGZus{}INT);}
\PYG{x}{            \PYGZdl{}statement\PYGZhy{}\PYGZgt{}bindParam(\PYGZsq{}:ID\PYGZus{}emulator\PYGZsq{}, \PYGZdl{}idEmulator, \PYGZbs{}PDO::PARAM\PYGZus{}INT);}
\PYG{x}{            \PYGZdl{}statement\PYGZhy{}\PYGZgt{}execute();}

\PYG{x}{            \PYGZdl{}file = new File();}

\PYG{x}{            if (\PYGZdl{}statement\PYGZhy{}\PYGZgt{}rowCount() == 1) \PYGZob{}}
\PYG{x}{                \PYGZdl{}result = \PYGZdl{}statement\PYGZhy{}\PYGZgt{}fetch(\PYGZbs{}PDO::FETCH\PYGZus{}ASSOC);}
\PYG{x}{                \PYGZdl{}file\PYGZhy{}\PYGZgt{}filename = \PYGZdl{}result[\PYGZsq{}filesave\PYGZsq{}];}
\PYG{x}{            \PYGZcb{}}
\PYG{x}{            return \PYGZdl{}file\PYGZhy{}\PYGZgt{}filename;}
\PYG{x}{        \PYGZcb{} catch (\PYGZbs{}PDOException \PYGZdl{}e) \PYGZob{}}
\PYG{x}{            exit(\PYGZdl{}e\PYGZhy{}\PYGZgt{}getMessage());}
\PYG{x}{        \PYGZcb{}}
\PYG{x}{    \PYGZcb{}}
\end{sphinxVerbatim}


\subsubsection{Sauvegarde pas encore présente}
\label{\detokenize{organique:sauvegarde-pas-encore-presente}}
\sphinxAtStartPar
S’il n’y a aucune sauvegarde pour l’émulateur en question et l’utilisateur en question. Il faut commencer par donner un nom au fichier. Ce nom est constitué du MD5 du microtime() et de l’extension “.zip”. Quand le nom pour le fichier aété créé, il faut uploader le fichier dans le dossier de sauvegarde grâce à la fonction move\_uploaded\_file(file,path).

\sphinxAtStartPar
Quand l’upload est validé, un ajout de fichier de sauvegarde va se faire grâce à la fonction DAOFileSave\sphinxhyphen{}>AddFileSave(idEmulator,idUser,newfilename). Cette fonction va créer une entrée dans la base de données pour le nouveau fichier.

\begin{sphinxVerbatim}[commandchars=\\\{\}]
\PYG{x}{public function AddFileSave(\PYGZdl{}idEmulator, \PYGZdl{}idUser, \PYGZdl{}newFileName)}
\PYG{x}{    \PYGZob{}}
\PYG{x}{        \PYGZdl{}statementFile = \PYGZdq{}}
\PYG{x}{        INSERT INTO file}
\PYG{x}{        (filename, dateUpdate)}
\PYG{x}{        VALUES}
\PYG{x}{        (:FILENAME, NOW())\PYGZdq{};}
\PYG{x}{        try \PYGZob{}}
\PYG{x}{            \PYGZdl{}statementFile = \PYGZdl{}this\PYGZhy{}\PYGZgt{}db\PYGZhy{}\PYGZgt{}prepare(\PYGZdl{}statementFile);}
\PYG{x}{            \PYGZdl{}statementFile\PYGZhy{}\PYGZgt{}bindParam(\PYGZsq{}:FILENAME\PYGZsq{}, \PYGZdl{}newFileName, \PYGZbs{}PDO::PARAM\PYGZus{}STR);}
\PYG{x}{            \PYGZdl{}statementFile\PYGZhy{}\PYGZgt{}execute();}
\PYG{x}{            //when create file is done}
\PYG{x}{            \PYGZdl{}statementFileSave = \PYGZdq{}}
\PYG{x}{            INSERT INTO filesave}
\PYG{x}{            (idUser, idEmulator,idFile)}
\PYG{x}{            VALUES}
\PYG{x}{            (:ID\PYGZus{}user,:ID\PYGZus{}emulator, :ID\PYGZus{}file)\PYGZdq{};}
\PYG{x}{            try \PYGZob{}}
\PYG{x}{                \PYGZdl{}statementFileSave = \PYGZdl{}this\PYGZhy{}\PYGZgt{}db\PYGZhy{}\PYGZgt{}prepare(\PYGZdl{}statementFileSave);}
\PYG{x}{                \PYGZdl{}statementFileSave\PYGZhy{}\PYGZgt{}bindParam(\PYGZsq{}:ID\PYGZus{}user\PYGZsq{}, \PYGZdl{}idUser, \PYGZbs{}PDO::PARAM\PYGZus{}INT);}
\PYG{x}{                \PYGZdl{}statementFileSave\PYGZhy{}\PYGZgt{}bindParam(\PYGZsq{}:ID\PYGZus{}emulator\PYGZsq{}, \PYGZdl{}idEmulator, \PYGZbs{}PDO::PARAM\PYGZus{}INT);}
\PYG{x}{                \PYGZdl{}lastInsertId = \PYGZdl{}this\PYGZhy{}\PYGZgt{}db\PYGZhy{}\PYGZgt{}lastInsertId();}
\PYG{x}{                \PYGZdl{}statementFileSave\PYGZhy{}\PYGZgt{}bindParam(\PYGZsq{}:ID\PYGZus{}file\PYGZsq{}, \PYGZdl{}lastInsertId, \PYGZbs{}PDO::PARAM\PYGZus{}INT);}
\PYG{x}{                \PYGZdl{}statementFileSave\PYGZhy{}\PYGZgt{}execute();}
\PYG{x}{            \PYGZcb{} catch (\PYGZbs{}PDOException \PYGZdl{}e) \PYGZob{}}
\PYG{x}{                exit(\PYGZdl{}e\PYGZhy{}\PYGZgt{}getMessage());}
\PYG{x}{            \PYGZcb{}}
\PYG{x}{        \PYGZcb{} catch (\PYGZbs{}PDOException \PYGZdl{}e) \PYGZob{}}
\PYG{x}{            exit(\PYGZdl{}e\PYGZhy{}\PYGZgt{}getMessage());}
\PYG{x}{        \PYGZcb{}}
\PYG{x}{    \PYGZcb{}}
\end{sphinxVerbatim}


\subsection{Mise à jour du caimanToken}
\label{\detokenize{organique:mise-a-jour-du-caimantoken}}
\sphinxAtStartPar
Le caimanToken sert à pouvoir se connecter sans mot de passe à l’application Caiman. Se token doit être changée à chaque fois que l’utilisateur se connecte, que ce soit en se connectant avec un mot de passe ou par le caimanToken justement.

\sphinxAtStartPar
Pour ce faire, j’ai créé une function DAOUser\sphinxhyphen{}>updateCaimanToken(apiToken) qui sert à modifier le caimanToken d’un utilisateur dans la base de données. Le caimanToken est un md5 du microtime actuel. Par conséquent, il est presque impossible que deux utilisateurs aient le même.

\begin{sphinxVerbatim}[commandchars=\\\{\}]
\PYG{x}{public function updateCaimanToken(string \PYGZdl{}apitocken)}
\PYG{x}{    \PYGZob{}}
\PYG{x}{        \PYGZdl{}statement = \PYGZdq{}}
\PYG{x}{        UPDATE user}
\PYG{x}{        SET caimanToken = :CAIMAN\PYGZus{}TOKEN}
\PYG{x}{        WHERE apitocken = :API\PYGZus{}TOCKEN;\PYGZdq{};}

\PYG{x}{        try \PYGZob{}}
\PYG{x}{            \PYGZdl{}statement = \PYGZdl{}this\PYGZhy{}\PYGZgt{}db\PYGZhy{}\PYGZgt{}prepare(\PYGZdl{}statement);}
\PYG{x}{            \PYGZdl{}statement\PYGZhy{}\PYGZgt{}bindParam(\PYGZsq{}:API\PYGZus{}TOCKEN\PYGZsq{}, \PYGZdl{}apitocken);}
\PYG{x}{            \PYGZdl{}caimanTocken = md5(microtime());}
\PYG{x}{            \PYGZdl{}statement\PYGZhy{}\PYGZgt{}bindParam(\PYGZsq{}:CAIMAN\PYGZus{}TOKEN\PYGZsq{}, \PYGZdl{}caimanTocken);}
\PYG{x}{            \PYGZdl{}statement\PYGZhy{}\PYGZgt{}execute();}
\PYG{x}{        \PYGZcb{} catch (\PYGZbs{}PDOException \PYGZdl{}e) \PYGZob{}}
\PYG{x}{            exit(\PYGZdl{}e\PYGZhy{}\PYGZgt{}getMessage());}
\PYG{x}{        \PYGZcb{}}
\PYG{x}{    \PYGZcb{}}
\end{sphinxVerbatim}


\section{Description technique: Application Caiman}
\label{\detokenize{organique:description-technique-application-caiman}}
\sphinxAtStartPar
Diagramme de classe

\sphinxAtStartPar
\sphinxincludegraphics{{DiagramLogique}.png}


\subsection{Création de dossier}
\label{\detokenize{organique:creation-de-dossier}}

\subsubsection{Structure des dossiers}
\label{\detokenize{organique:structure-des-dossiers}}
\sphinxAtStartPar
Caiman nécessite de créer certains dossiers et fichiers pour pouvoir fonctionner correctement. Ces dossiers se trouvent dans le dossier “\%appdata\%\textbackslash{}Roaming\textbackslash{}Caiman”

\sphinxAtStartPar
\sphinxincludegraphics{{dossier_appdata_caiman}.png}

\sphinxAtStartPar
\sphinxstylestrong{Caiman}

\sphinxAtStartPar
Ce dossier contient les fichiers de configuration graphique, la liste des jeux téléchargés et le dernier CaimanToken reçu.

\sphinxAtStartPar
\sphinxstylestrong{img}

\sphinxAtStartPar
Ce dossier contient les différentes images des jeux, ces images sont téléchargées sur le site internet de caiman.

\sphinxAtStartPar
\sphinxstylestrong{users}

\sphinxAtStartPar
Ce dossier contient les dossiers personnels de chaque utilisateur de l’application qui s’est connecté sur le pc. Il possède les fichiers de sauvegarde de l’utilisateur en question.


\subsubsection{Création des dossiers}
\label{\detokenize{organique:creation-des-dossiers}}
\sphinxAtStartPar
\sphinxstylestrong{Dossier de base}

\sphinxAtStartPar
Les dossiers pour caiman doivent obligatoirement être créés. Donc au début du lancement de l’application Caiman, il y a toujours une vérification pour savoir si les dossiers sont bien présents. Si c’est le premier lancement de Caiman ou si les dossiers ont été supprimés, ils vont être créés.

\begin{sphinxVerbatim}[commandchars=\\\{\}]
\PYG{k}{private} \PYG{k}{void} \PYG{n+nf}{CreateAppDataFolder}\PYG{p}{(}\PYG{p}{)}
        \PYG{p}{\PYGZob{}}
            \PYG{k+kt}{var} \PYG{n}{appDataPath} \PYG{p}{=} \PYG{n}{Environment}\PYG{p}{.}\PYG{n}{GetFolderPath}\PYG{p}{(}\PYG{n}{Environment}\PYG{p}{.}\PYG{n}{SpecialFolder}\PYG{p}{.}\PYG{n}{ApplicationData}\PYG{p}{)}\PYG{p}{;}
            \PYG{k+kt}{var} \PYG{n}{basePath} \PYG{p}{=} \PYG{n}{Path}\PYG{p}{.}\PYG{n}{Combine}\PYG{p}{(}\PYG{n}{appDataPath}\PYG{p}{,} \PYG{l+s}{@\PYGZdq{}Caiman\PYGZbs{}\PYGZdq{}}\PYG{p}{)}\PYG{p}{;}
            \PYG{k+kt}{var} \PYG{n}{caimanConfigPath} \PYG{p}{=} \PYG{n}{Path}\PYG{p}{.}\PYG{n}{Combine}\PYG{p}{(}\PYG{n}{appDataPath}\PYG{p}{,} \PYG{l+s}{@\PYGZdq{}Caiman\PYGZbs{}Caiman\PYGZbs{}\PYGZdq{}}\PYG{p}{)}\PYG{p}{;}
            \PYG{k+kt}{var} \PYG{n}{imgPath} \PYG{p}{=} \PYG{n}{Path}\PYG{p}{.}\PYG{n}{Combine}\PYG{p}{(}\PYG{n}{appDataPath}\PYG{p}{,} \PYG{l+s}{@\PYGZdq{}Caiman\PYGZbs{}img\PYGZbs{}\PYGZdq{}}\PYG{p}{)}\PYG{p}{;}
            \PYG{k+kt}{var} \PYG{n}{gamesPath} \PYG{p}{=} \PYG{n}{Path}\PYG{p}{.}\PYG{n}{Combine}\PYG{p}{(}\PYG{n}{appDataPath}\PYG{p}{,} \PYG{l+s}{@\PYGZdq{}Caiman\PYGZbs{}Caiman\PYGZbs{}games.ini\PYGZdq{}}\PYG{p}{)}\PYG{p}{;}
            \PYG{k+kt}{var} \PYG{n}{configPath} \PYG{p}{=} \PYG{n}{Path}\PYG{p}{.}\PYG{n}{Combine}\PYG{p}{(}\PYG{n}{appDataPath}\PYG{p}{,} \PYG{l+s}{@\PYGZdq{}Caiman\PYGZbs{}Caiman\PYGZbs{}config.ini\PYGZdq{}}\PYG{p}{)}\PYG{p}{;}
            \PYG{k+kt}{var} \PYG{n}{loginPath} \PYG{p}{=} \PYG{n}{Path}\PYG{p}{.}\PYG{n}{Combine}\PYG{p}{(}\PYG{n}{appDataPath}\PYG{p}{,} \PYG{l+s}{@\PYGZdq{}Caiman\PYGZbs{}Caiman\PYGZbs{}login.ini\PYGZdq{}}\PYG{p}{)}\PYG{p}{;}

            \PYG{k}{if} \PYG{p}{(}\PYG{p}{!}\PYG{n}{Directory}\PYG{p}{.}\PYG{n}{Exists}\PYG{p}{(}\PYG{n}{basePath}\PYG{p}{)}\PYG{p}{)}
            \PYG{p}{\PYGZob{}}
                \PYG{n}{Directory}\PYG{p}{.}\PYG{n}{CreateDirectory}\PYG{p}{(}\PYG{n}{basePath}\PYG{p}{)}\PYG{p}{;}
            \PYG{p}{\PYGZcb{}}
            \PYG{k}{if} \PYG{p}{(}\PYG{p}{!}\PYG{n}{Directory}\PYG{p}{.}\PYG{n}{Exists}\PYG{p}{(}\PYG{n}{imgPath}\PYG{p}{)}\PYG{p}{)}
            \PYG{p}{\PYGZob{}}
                \PYG{n}{Directory}\PYG{p}{.}\PYG{n}{CreateDirectory}\PYG{p}{(}\PYG{n}{imgPath}\PYG{p}{)}\PYG{p}{;}
            \PYG{p}{\PYGZcb{}}
            \PYG{k}{if} \PYG{p}{(}\PYG{p}{!}\PYG{n}{Directory}\PYG{p}{.}\PYG{n}{Exists}\PYG{p}{(}\PYG{n}{caimanConfigPath}\PYG{p}{)}\PYG{p}{)}
            \PYG{p}{\PYGZob{}}
                \PYG{n}{Directory}\PYG{p}{.}\PYG{n}{CreateDirectory}\PYG{p}{(}\PYG{n}{caimanConfigPath}\PYG{p}{)}\PYG{p}{;}
            \PYG{p}{\PYGZcb{}}
            \PYG{k}{if} \PYG{p}{(}\PYG{p}{!}\PYG{n}{Directory}\PYG{p}{.}\PYG{n}{Exists}\PYG{p}{(}\PYG{l+s}{@\PYGZdq{}C:\PYGZbs{}Caiman\PYGZbs{}Playstation2\PYGZdq{}}\PYG{p}{)}\PYG{p}{)}
            \PYG{p}{\PYGZob{}}
                \PYG{n}{Directory}\PYG{p}{.}\PYG{n}{CreateDirectory}\PYG{p}{(}\PYG{l+s}{@\PYGZdq{}C:\PYGZbs{}Caiman\PYGZbs{}Playstation2\PYGZbs{}\PYGZdq{}}\PYG{p}{)}\PYG{p}{;}
            \PYG{p}{\PYGZcb{}}
            \PYG{k}{if} \PYG{p}{(}\PYG{p}{!}\PYG{n}{Directory}\PYG{p}{.}\PYG{n}{Exists}\PYG{p}{(}\PYG{l+s}{@\PYGZdq{}C:\PYGZbs{}Caiman\PYGZbs{}GamecubeWii\PYGZbs{}\PYGZdq{}}\PYG{p}{)}\PYG{p}{)}
            \PYG{p}{\PYGZob{}}
                \PYG{n}{Directory}\PYG{p}{.}\PYG{n}{CreateDirectory}\PYG{p}{(}\PYG{l+s}{@\PYGZdq{}C:\PYGZbs{}Caiman\PYGZbs{}GamecubeWii\PYGZbs{}\PYGZdq{}}\PYG{p}{)}\PYG{p}{;}
            \PYG{p}{\PYGZcb{}}
            \PYG{k}{if} \PYG{p}{(}\PYG{p}{!}\PYG{n}{File}\PYG{p}{.}\PYG{n}{Exists}\PYG{p}{(}\PYG{n}{loginPath}\PYG{p}{)}\PYG{p}{)}
            \PYG{p}{\PYGZob{}}
                \PYG{k}{using} \PYG{p}{(}\PYG{n}{StreamWriter} \PYG{n}{sw} \PYG{p}{=} \PYG{n}{File}\PYG{p}{.}\PYG{n}{CreateText}\PYG{p}{(}\PYG{n}{loginPath}\PYG{p}{)}\PYG{p}{)}
                \PYG{p}{\PYGZob{}}
                    \PYG{n}{sw}\PYG{p}{.}\PYG{n}{WriteLine}\PYG{p}{(}\PYG{l+s}{\PYGZdq{}token = 0\PYGZdq{}}\PYG{p}{)}\PYG{p}{;}
                \PYG{p}{\PYGZcb{}}
                \PYG{n}{loginFile} \PYG{p}{=} \PYG{k}{new} \PYG{n}{ConfigFileEditor}\PYG{p}{(}\PYG{n}{caimanConfigPath}\PYG{p}{,} \PYG{l+s}{\PYGZdq{}login.ini\PYGZdq{}}\PYG{p}{)}\PYG{p}{;}
            \PYG{p}{\PYGZcb{}}
            \PYG{k}{if} \PYG{p}{(}\PYG{p}{!}\PYG{n}{File}\PYG{p}{.}\PYG{n}{Exists}\PYG{p}{(}\PYG{n}{gamesPath}\PYG{p}{)}\PYG{p}{)}
            \PYG{p}{\PYGZob{}}
                \PYG{k}{using} \PYG{p}{(}\PYG{n}{StreamWriter} \PYG{n}{sw} \PYG{p}{=} \PYG{n}{File}\PYG{p}{.}\PYG{n}{CreateText}\PYG{p}{(}\PYG{n}{gamesPath}\PYG{p}{)}\PYG{p}{)}
                \PYG{p}{\PYGZob{}}

                \PYG{p}{\PYGZcb{}}
                \PYG{n}{configFile} \PYG{p}{=} \PYG{k}{new} \PYG{n}{ConfigFileEditor}\PYG{p}{(}\PYG{n}{caimanConfigPath}\PYG{p}{,} \PYG{l+s}{\PYGZdq{}config.ini\PYGZdq{}}\PYG{p}{)}\PYG{p}{;}
            \PYG{p}{\PYGZcb{}}
            \PYG{k}{if} \PYG{p}{(}\PYG{p}{!}\PYG{n}{File}\PYG{p}{.}\PYG{n}{Exists}\PYG{p}{(}\PYG{n}{configPath}\PYG{p}{)}\PYG{p}{)}
            \PYG{p}{\PYGZob{}}
                \PYG{c+c1}{//if the file do not exist create it with with the default emulators value}
                \PYG{k}{using} \PYG{p}{(}\PYG{n}{StreamWriter} \PYG{n}{sw} \PYG{p}{=} \PYG{n}{File}\PYG{p}{.}\PYG{n}{CreateText}\PYG{p}{(}\PYG{n}{configPath}\PYG{p}{)}\PYG{p}{)}
                \PYG{p}{\PYGZob{}}
                    \PYG{n}{sw}\PYG{p}{.}\PYG{n}{WriteLine}\PYG{p}{(}\PYG{l+s}{\PYGZdq{}configuration = 1080\PYGZdq{}}\PYG{p}{)}\PYG{p}{;}
                    \PYG{n}{sw}\PYG{p}{.}\PYG{n}{WriteLine}\PYG{p}{(}\PYG{l+s}{\PYGZdq{}fullscreen = true\PYGZdq{}}\PYG{p}{)}\PYG{p}{;}
                    \PYG{n}{sw}\PYG{p}{.}\PYG{n}{WriteLine}\PYG{p}{(}\PYG{l+s}{\PYGZdq{}definition = 3\PYGZdq{}}\PYG{p}{)}\PYG{p}{;}
                    \PYG{n}{sw}\PYG{p}{.}\PYG{n}{WriteLine}\PYG{p}{(}\PYG{l+s}{\PYGZdq{}formatSeizeNeuvieme = true\PYGZdq{}}\PYG{p}{)}\PYG{p}{;}
                    \PYG{n}{sw}\PYG{p}{.}\PYG{n}{WriteLine}\PYG{p}{(}\PYG{l+s}{\PYGZdq{}filtrageAnioscopique = 3\PYGZdq{}}\PYG{p}{)}\PYG{p}{;}
                \PYG{p}{\PYGZcb{}}
                \PYG{n}{configFile} \PYG{p}{=} \PYG{k}{new} \PYG{n}{ConfigFileEditor}\PYG{p}{(}\PYG{n}{caimanConfigPath}\PYG{p}{,} \PYG{l+s}{\PYGZdq{}config.ini\PYGZdq{}}\PYG{p}{)}\PYG{p}{;}

            \PYG{p}{\PYGZcb{}}


        \PYG{p}{\PYGZcb{}}
\end{sphinxVerbatim}

\sphinxAtStartPar
\sphinxstylestrong{Dossier des utilisateurs}

\sphinxAtStartPar
La création des dossiers des utilisateurs pour gérer correctement les sauvegardes. Ces dossiers sont créés à la première connexion de l’utilisateur.

\begin{sphinxVerbatim}[commandchars=\\\{\}]
 \PYG{k}{public} \PYG{k}{void} \PYG{n+nf}{CreateUserFolder}\PYG{p}{(}\PYG{p}{)}
        \PYG{p}{\PYGZob{}}
            \PYG{k+kt}{var} \PYG{n}{appDataPath} \PYG{p}{=} \PYG{n}{Environment}\PYG{p}{.}\PYG{n}{GetFolderPath}\PYG{p}{(}\PYG{n}{Environment}\PYG{p}{.}\PYG{n}{SpecialFolder}\PYG{p}{.}\PYG{n}{ApplicationData}\PYG{p}{)}\PYG{p}{;}
            \PYG{k+kt}{var} \PYG{n}{userPath} \PYG{p}{=} \PYG{n}{Path}\PYG{p}{.}\PYG{n}{Combine}\PYG{p}{(}\PYG{n}{appDataPath}\PYG{p}{,} \PYG{l+s}{@\PYGZdq{}Caiman\PYGZbs{}users\PYGZbs{}\PYGZdq{}} \PYG{p}{+} \PYG{n}{username} \PYG{p}{+} \PYG{l+s}{@\PYGZdq{}\PYGZbs{}\PYGZdq{}}\PYG{p}{)}\PYG{p}{;}
            \PYG{k+kt}{var} \PYG{n}{savePath} \PYG{p}{=} \PYG{n}{Path}\PYG{p}{.}\PYG{n}{Combine}\PYG{p}{(}\PYG{n}{userPath}\PYG{p}{,} \PYG{l+s}{@\PYGZdq{}Save\PYGZbs{}\PYGZdq{}}\PYG{p}{)}\PYG{p}{;}
            \PYG{k+kt}{var} \PYG{n}{savePathPlaystation} \PYG{p}{=} \PYG{n}{Path}\PYG{p}{.}\PYG{n}{Combine}\PYG{p}{(}\PYG{n}{savePath}\PYG{p}{,} \PYG{l+s}{@\PYGZdq{}Playstation2\PYGZbs{}\PYGZdq{}}\PYG{p}{)}\PYG{p}{;}
            \PYG{k+kt}{var} \PYG{n}{savePathGamecubeWii} \PYG{p}{=} \PYG{n}{Path}\PYG{p}{.}\PYG{n}{Combine}\PYG{p}{(}\PYG{n}{savePath}\PYG{p}{,} \PYG{l+s}{@\PYGZdq{}GamecubeWii\PYGZbs{}\PYGZdq{}}\PYG{p}{)}\PYG{p}{;}

            \PYG{k}{if} \PYG{p}{(}\PYG{p}{!}\PYG{n}{Directory}\PYG{p}{.}\PYG{n}{Exists}\PYG{p}{(}\PYG{n}{userPath}\PYG{p}{)}\PYG{p}{)}
            \PYG{p}{\PYGZob{}}
                \PYG{n}{Directory}\PYG{p}{.}\PYG{n}{CreateDirectory}\PYG{p}{(}\PYG{n}{userPath}\PYG{p}{)}\PYG{p}{;}
            \PYG{p}{\PYGZcb{}}
            \PYG{k}{if} \PYG{p}{(}\PYG{p}{!}\PYG{n}{Directory}\PYG{p}{.}\PYG{n}{Exists}\PYG{p}{(}\PYG{n}{savePath}\PYG{p}{)}\PYG{p}{)}
            \PYG{p}{\PYGZob{}}
                \PYG{n}{Directory}\PYG{p}{.}\PYG{n}{CreateDirectory}\PYG{p}{(}\PYG{n}{savePath}\PYG{p}{)}\PYG{p}{;}
            \PYG{p}{\PYGZcb{}}
            \PYG{k}{if} \PYG{p}{(}\PYG{p}{!}\PYG{n}{Directory}\PYG{p}{.}\PYG{n}{Exists}\PYG{p}{(}\PYG{n}{savePathPlaystation}\PYG{p}{)}\PYG{p}{)}
            \PYG{p}{\PYGZob{}}
                \PYG{n}{Directory}\PYG{p}{.}\PYG{n}{CreateDirectory}\PYG{p}{(}\PYG{n}{savePathPlaystation}\PYG{p}{)}\PYG{p}{;}
            \PYG{p}{\PYGZcb{}}
            \PYG{k}{if} \PYG{p}{(}\PYG{p}{!}\PYG{n}{Directory}\PYG{p}{.}\PYG{n}{Exists}\PYG{p}{(}\PYG{n}{savePathGamecubeWii}\PYG{p}{)}\PYG{p}{)}
            \PYG{p}{\PYGZob{}}
                \PYG{n}{Directory}\PYG{p}{.}\PYG{n}{CreateDirectory}\PYG{p}{(}\PYG{n}{savePathGamecubeWii}\PYG{p}{)}\PYG{p}{;}
            \PYG{p}{\PYGZcb{}}

        \PYG{p}{\PYGZcb{}}
\end{sphinxVerbatim}


\subsection{Verification des jeux présent sur le disque}
\label{\detokenize{organique:verification-des-jeux-present-sur-le-disque}}
\sphinxAtStartPar
Pour connaître les jeux qui ont été téléchargés, il existe un fichier comprenant les ids des jeux. Ce fichier se trouve dans le dossier “appdata” de Caiman. Il est donc commun à tous les utilisateurs. Ce fichier est mis à jour quand un utilisateur a fini de télécharger un jeu ou quand il décide d’en supprimer un.

\begin{sphinxVerbatim}[commandchars=\\\{\}]
\PYG{k}{private} \PYG{k}{void} \PYG{n+nf}{CheckIfGameFileIsPresentOnDisk}\PYG{p}{(}\PYG{p}{)}
        \PYG{p}{\PYGZob{}}
            \PYG{n}{List}\PYG{p}{\PYGZlt{}}\PYG{k+kt}{string}\PYG{p}{\PYGZgt{}} \PYG{n}{lst\PYGZus{}idGames} \PYG{p}{=} \PYG{n}{gamesListConfigFile}\PYG{p}{.}\PYG{n}{GetAllValueInList}\PYG{p}{(}\PYG{p}{)}\PYG{p}{;}

            \PYG{k}{foreach} \PYG{p}{(}\PYG{k+kt}{var} \PYG{n}{idGameString} \PYG{k}{in} \PYG{n}{lst\PYGZus{}idGames}\PYG{p}{)}
            \PYG{p}{\PYGZob{}}
                \PYG{k}{if} \PYG{p}{(}\PYG{n}{idGameString} \PYG{p}{!}\PYG{p}{=} \PYG{l+s}{\PYGZdq{}\PYGZdq{}}\PYG{p}{)}
                \PYG{p}{\PYGZob{}}
                    \PYG{k+kt}{int} \PYG{n}{idGame} \PYG{p}{=} \PYG{n}{Convert}\PYG{p}{.}\PYG{n}{ToInt32}\PYG{p}{(}\PYG{n}{idGameString}\PYG{p}{)}\PYG{p}{;}
                    \PYG{k}{if} \PYG{p}{(}\PYG{p}{!}\PYG{n}{File}\PYG{p}{.}\PYG{n}{Exists}\PYG{p}{(}\PYG{l+s}{@\PYGZdq{}C:\PYGZbs{}Caiman\PYGZbs{}\PYGZdq{}} \PYG{p}{+} \PYG{n}{callAPI}\PYG{p}{.}\PYG{n}{CallFolderNameGame}\PYG{p}{(}\PYG{n}{idGame}\PYG{p}{)} \PYG{p}{+} \PYG{l+s}{@\PYGZdq{}\PYGZbs{}\PYGZdq{}} \PYG{p}{+} \PYG{n}{callAPI}\PYG{p}{.}\PYG{n}{CallFileNameGame}\PYG{p}{(}\PYG{n}{idGame}\PYG{p}{)}\PYG{p}{)}\PYG{p}{)}
                    \PYG{p}{\PYGZob{}}
                        \PYG{n}{gamesListConfigFile}\PYG{p}{.}\PYG{n}{DeleteValue}\PYG{p}{(}\PYG{n}{idGameString}\PYG{p}{)}\PYG{p}{;}
                    \PYG{p}{\PYGZcb{}}
                \PYG{p}{\PYGZcb{}}

            \PYG{p}{\PYGZcb{}}
        \PYG{p}{\PYGZcb{}}
\end{sphinxVerbatim}

\sphinxAtStartPar
Si l’utilisateur décide de supprimer un fichier sans passer par caiman il pourrait y avoir un souci alors pour pallier à ce problème une vérification est faite. Au lancement de l’application, si un fichier est manquant, alors le fichier qui contient les id des jeux sera mis à jour.


\subsection{Paramètres graphique}
\label{\detokenize{organique:parametres-graphique}}

\subsubsection{Liste des paramètres graphique}
\label{\detokenize{organique:liste-des-parametres-graphique}}
\sphinxAtStartPar
L’utilisateur a la possibilité de modifier plusieurs paramètres:

\sphinxAtStartPar
La configuration global de l’émulateur
\begin{itemize}
\item {} 
\sphinxAtStartPar
Original

\item {} 
\sphinxAtStartPar
1080p

\item {} 
\sphinxAtStartPar
4K

\end{itemize}

\sphinxAtStartPar
Si le jeu doit se lancer en mode plein écran
\begin{itemize}
\item {} 
\sphinxAtStartPar
true

\item {} 
\sphinxAtStartPar
false

\end{itemize}

\sphinxAtStartPar
Si le jeu doit se lancer en 16/9
\begin{itemize}
\item {} 
\sphinxAtStartPar
true

\item {} 
\sphinxAtStartPar
false

\end{itemize}

\sphinxAtStartPar
Les différents modes graphiques modifient ces paramètres:
\begin{itemize}
\item {} 
\sphinxAtStartPar
le filtrage anisotropique

\item {} 
\sphinxAtStartPar
l’upscale de la définition

\end{itemize}

\sphinxAtStartPar
Le filtrage anisotropique permet entre autres de diminuer l’effet de crénelage, ainsi qu’à améliorer l’affichage des textures vue depuis un angle de vue extrême.

\sphinxAtStartPar
L’upscale de la définition permet d’augmenter la définition native du rendu. Par exemple, la définition de base de la Playstation 2 sur le jeu Kingdom Heart( 512x448). Cette définition est donc la définition que l’utilisateur va avoir en “Original”, mais s’il opte pour le paramètre 1080p, il va avoir une définition native de 1536x1344. En mode 4K le rendu sera en définition 4096x3584.


\subsubsection{Fichiers de configuration PCSX2}
\label{\detokenize{organique:fichiers-de-configuration-pcsx2}}
\sphinxAtStartPar
Pour l’émulateur PCSX2, il y a deux fichiers de configuration qui doivent être modifiés.

\sphinxAtStartPar
Les fichiers se trouvent dans le dossier “PCSX2\textbackslash{}inis\textbackslash{}”.

\sphinxAtStartPar
Le fichier “GSDX.ini” permet de modifier les paramètres de d’upscale et de filtrage anisotropique.

\sphinxAtStartPar
Le fichier “PCSX2\_ui.ini” permet de modifier l’affichage en plein écran et le format d’affichage.

\sphinxAtStartPar
Le paramètre de format d’écran dans le fichier de configuration de PCSX n’est pas un booléen mais du texte je dois avant l’écriture dans le fichier le convertir.

\sphinxAtStartPar
Pareil pour le paramètre de mode plein écran

\begin{sphinxVerbatim}[commandchars=\\\{\}]
\PYG{k}{public} \PYG{k}{override} \PYG{k}{void} \PYG{n+nf}{UpdateConfigurationFile}\PYG{p}{(}\PYG{p}{)}
        \PYG{p}{\PYGZob{}}
            \PYG{n}{configFileGSdx}\PYG{p}{.}\PYG{n}{UpdateProperties}\PYG{p}{(}\PYG{l+s}{\PYGZdq{}upscale\PYGZus{}multiplier\PYGZdq{}}\PYG{p}{,} \PYG{n}{definition}\PYG{p}{.}\PYG{n}{ToString}\PYG{p}{(}\PYG{p}{)}\PYG{p}{)}\PYG{p}{;}
            \PYG{n}{configFileGSdx}\PYG{p}{.}\PYG{n}{UpdateProperties}\PYG{p}{(}\PYG{l+s}{\PYGZdq{}MaxAnisotropy\PYGZdq{}}\PYG{p}{,} \PYG{n}{filtrageAnioscopique}\PYG{p}{.}\PYG{n}{ToString}\PYG{p}{(}\PYG{p}{)}\PYG{p}{)}\PYG{p}{;}

            \PYG{c+c1}{// the format is not true or false so i have to format it}
            \PYG{k}{if} \PYG{p}{(}\PYG{n}{fullScreen}\PYG{p}{)}
            \PYG{p}{\PYGZob{}}
                \PYG{n}{configFilePCSX2\PYGZus{}ui}\PYG{p}{.}\PYG{n}{UpdateProperties}\PYG{p}{(}\PYG{l+s}{\PYGZdq{}DefaultToFullscreen\PYGZdq{}}\PYG{p}{,} \PYG{l+s}{\PYGZdq{}enabled\PYGZdq{}}\PYG{p}{)}\PYG{p}{;}
            \PYG{p}{\PYGZcb{}}
            \PYG{k}{else}
            \PYG{p}{\PYGZob{}}
                \PYG{n}{configFilePCSX2\PYGZus{}ui}\PYG{p}{.}\PYG{n}{UpdateProperties}\PYG{p}{(}\PYG{l+s}{\PYGZdq{}DefaultToFullscreen\PYGZdq{}}\PYG{p}{,} \PYG{l+s}{\PYGZdq{}disabled\PYGZdq{}}\PYG{p}{)}\PYG{p}{;}
            \PYG{p}{\PYGZcb{}}

            \PYG{c+c1}{// the format is not true or false so i have to format it }
            \PYG{k}{if} \PYG{p}{(}\PYG{n}{formatSeizeNeuvieme}\PYG{p}{)}
            \PYG{p}{\PYGZob{}}
                \PYG{n}{configFilePCSX2\PYGZus{}ui}\PYG{p}{.}\PYG{n}{UpdateProperties}\PYG{p}{(}\PYG{l+s}{\PYGZdq{}AspectRatio\PYGZdq{}}\PYG{p}{,} \PYG{l+s}{\PYGZdq{}16:9\PYGZdq{}}\PYG{p}{)}\PYG{p}{;}
            \PYG{p}{\PYGZcb{}}
            \PYG{k}{else}
            \PYG{p}{\PYGZob{}}
                \PYG{n}{configFilePCSX2\PYGZus{}ui}\PYG{p}{.}\PYG{n}{UpdateProperties}\PYG{p}{(}\PYG{l+s}{\PYGZdq{}AspectRatio\PYGZdq{}}\PYG{p}{,} \PYG{l+s}{\PYGZdq{}4:3\PYGZdq{}}\PYG{p}{)}\PYG{p}{;}
            \PYG{p}{\PYGZcb{}}
        \PYG{p}{\PYGZcb{}}
\end{sphinxVerbatim}


\subsubsection{Fichiers de configuration Dolphin}
\label{\detokenize{organique:fichiers-de-configuration-dolphin}}
\sphinxAtStartPar
Pour l’émulateur Dolphin, il y a deux fichiers de configuration qui doivent être modifiés.

\sphinxAtStartPar
Les fichiers se trouvent dans le dossier “Dolphin\textbackslash{}User\textbackslash{}Config\textbackslash{}”.

\sphinxAtStartPar
Le fichier “GFX.ini” permet de modifier les paramètres de d’upscale et de filtrage anisotropique.

\sphinxAtStartPar
Le fichier “Dolphin.ini” permet de modifier l’affichage en plein écran et le format d’affichage.

\sphinxAtStartPar
Le paramètre de format d’écran dans le fichier de configuration de Dolphin n’est pas un booléen mais un nombre. Donc avant l’écriture dans le fichier je dois le convertir.

\begin{sphinxVerbatim}[commandchars=\\\{\}]
\PYG{k}{public} \PYG{k}{override} \PYG{k}{void} \PYG{n+nf}{UpdateConfigurationFile}\PYG{p}{(}\PYG{p}{)}
        \PYG{p}{\PYGZob{}}
            \PYG{n}{configFileDolphin}\PYG{p}{.}\PYG{n}{UpdateProperties}\PYG{p}{(}\PYG{l+s}{\PYGZdq{}Fullscreen\PYGZdq{}}\PYG{p}{,} \PYG{n}{fullScreen}\PYG{p}{.}\PYG{n}{ToString}\PYG{p}{(}\PYG{p}{)}\PYG{p}{)}\PYG{p}{;}
            \PYG{n}{configFileGFX}\PYG{p}{.}\PYG{n}{UpdateProperties}\PYG{p}{(}\PYG{l+s}{\PYGZdq{}InternalResolution\PYGZdq{}}\PYG{p}{,} \PYG{n}{definition}\PYG{p}{.}\PYG{n}{ToString}\PYG{p}{(}\PYG{p}{)}\PYG{p}{)}\PYG{p}{;}
            \PYG{n}{configFileGFX}\PYG{p}{.}\PYG{n}{UpdateProperties}\PYG{p}{(}\PYG{l+s}{\PYGZdq{}MaxAnisotropy\PYGZdq{}}\PYG{p}{,} \PYG{n}{filtrageAnioscopique}\PYG{p}{.}\PYG{n}{ToString}\PYG{p}{(}\PYG{p}{)}\PYG{p}{)}\PYG{p}{;}

            \PYG{k}{if} \PYG{p}{(}\PYG{n}{formatSeizeNeuvieme}\PYG{p}{)}
            \PYG{p}{\PYGZob{}}
                \PYG{n}{configFileGFX}\PYG{p}{.}\PYG{n}{UpdateProperties}\PYG{p}{(}\PYG{l+s}{\PYGZdq{}AspectRatio\PYGZdq{}}\PYG{p}{,} \PYG{l+s}{\PYGZdq{}1\PYGZdq{}}\PYG{p}{)}\PYG{p}{;}
            \PYG{p}{\PYGZcb{}}
            \PYG{k}{else}
            \PYG{p}{\PYGZob{}}
                \PYG{n}{configFileGFX}\PYG{p}{.}\PYG{n}{UpdateProperties}\PYG{p}{(}\PYG{l+s}{\PYGZdq{}AspectRatio\PYGZdq{}}\PYG{p}{,} \PYG{l+s}{\PYGZdq{}2\PYGZdq{}}\PYG{p}{)}\PYG{p}{;}
            \PYG{p}{\PYGZcb{}}
            
        \PYG{p}{\PYGZcb{}}
\end{sphinxVerbatim}


\subsubsection{Sauvegarde des paramètres graphique}
\label{\detokenize{organique:sauvegarde-des-parametres-graphique}}
\sphinxAtStartPar
Les paramètres graphiques définis par l’utilisateur se trouvent dans le dossier “appdata\textbackslash{}Caiman\textbackslash{}Caiman\textbackslash{}config.ini”.

\sphinxAtStartPar
Les paramètres sont mis à jour grâce à la classe ConfigFileEditor, cette classe permet la manipulation de fichier “.ini” que ce soit la lecture,la modification, la suppression, la création de propriétés.

\begin{sphinxVerbatim}[commandchars=\\\{\}]
\PYG{k}{public} \PYG{k}{void} \PYG{n+nf}{ApplyGlobalConfiguration}\PYG{p}{(}\PYG{k+kt}{string} \PYG{n}{configuration}\PYG{p}{)}
        \PYG{p}{\PYGZob{}}
            \PYG{k}{switch} \PYG{p}{(}\PYG{n}{configuration}\PYG{p}{)}
            \PYG{p}{\PYGZob{}}
                \PYG{k}{case} \PYG{l+s}{\PYGZdq{}original\PYGZdq{}}\PYG{p}{:}
                    \PYG{n}{configFile}\PYG{p}{.}\PYG{n}{UpdateProperties}\PYG{p}{(}\PYG{l+s}{\PYGZdq{}definition\PYGZdq{}}\PYG{p}{,} \PYG{l+s}{\PYGZdq{}1\PYGZdq{}}\PYG{p}{)}\PYG{p}{;}
                    \PYG{n}{configFile}\PYG{p}{.}\PYG{n}{UpdateProperties}\PYG{p}{(}\PYG{l+s}{\PYGZdq{}filtrageAnioscopique\PYGZdq{}}\PYG{p}{,} \PYG{l+s}{\PYGZdq{}1\PYGZdq{}}\PYG{p}{)}\PYG{p}{;}
                    \PYG{n}{configFile}\PYG{p}{.}\PYG{n}{UpdateProperties}\PYG{p}{(}\PYG{l+s}{\PYGZdq{}configuration\PYGZdq{}}\PYG{p}{,} \PYG{l+s}{\PYGZdq{}original\PYGZdq{}}\PYG{p}{)}\PYG{p}{;}
                    \PYG{n}{ScanConfiguration}\PYG{p}{(}\PYG{p}{)}\PYG{p}{;}
                    \PYG{k}{break}\PYG{p}{;}
                \PYG{k}{case} \PYG{l+s}{\PYGZdq{}1080\PYGZdq{}}\PYG{p}{:}
                    \PYG{n}{configFile}\PYG{p}{.}\PYG{n}{UpdateProperties}\PYG{p}{(}\PYG{l+s}{\PYGZdq{}definition\PYGZdq{}}\PYG{p}{,} \PYG{l+s}{\PYGZdq{}3\PYGZdq{}}\PYG{p}{)}\PYG{p}{;}
                    \PYG{n}{configFile}\PYG{p}{.}\PYG{n}{UpdateProperties}\PYG{p}{(}\PYG{l+s}{\PYGZdq{}filtrageAnioscopique\PYGZdq{}}\PYG{p}{,} \PYG{l+s}{\PYGZdq{}4\PYGZdq{}}\PYG{p}{)}\PYG{p}{;}
                    \PYG{n}{configFile}\PYG{p}{.}\PYG{n}{UpdateProperties}\PYG{p}{(}\PYG{l+s}{\PYGZdq{}configuration\PYGZdq{}}\PYG{p}{,} \PYG{l+s}{\PYGZdq{}1080p\PYGZdq{}}\PYG{p}{)}\PYG{p}{;}
                    \PYG{n}{ScanConfiguration}\PYG{p}{(}\PYG{p}{)}\PYG{p}{;}
                    \PYG{k}{break}\PYG{p}{;}
                \PYG{k}{case} \PYG{l+s}{\PYGZdq{}4K\PYGZdq{}}\PYG{p}{:}
                    \PYG{n}{configFile}\PYG{p}{.}\PYG{n}{UpdateProperties}\PYG{p}{(}\PYG{l+s}{\PYGZdq{}definition\PYGZdq{}}\PYG{p}{,} \PYG{l+s}{\PYGZdq{}8\PYGZdq{}}\PYG{p}{)}\PYG{p}{;}
                    \PYG{n}{configFile}\PYG{p}{.}\PYG{n}{UpdateProperties}\PYG{p}{(}\PYG{l+s}{\PYGZdq{}filtrageAnioscopique\PYGZdq{}}\PYG{p}{,} \PYG{l+s}{\PYGZdq{}4\PYGZdq{}}\PYG{p}{)}\PYG{p}{;}
                    \PYG{n}{configFile}\PYG{p}{.}\PYG{n}{UpdateProperties}\PYG{p}{(}\PYG{l+s}{\PYGZdq{}configuration\PYGZdq{}}\PYG{p}{,} \PYG{l+s}{\PYGZdq{}4K\PYGZdq{}}\PYG{p}{)}\PYG{p}{;}
                    \PYG{n}{ScanConfiguration}\PYG{p}{(}\PYG{p}{)}\PYG{p}{;}
                    \PYG{k}{break}\PYG{p}{;}
                \PYG{k}{default}\PYG{p}{:}
                    \PYG{k}{break}\PYG{p}{;}
            \PYG{p}{\PYGZcb{}}
        \PYG{p}{\PYGZcb{}}
\end{sphinxVerbatim}


\subsection{Téléchargement d’images}
\label{\detokenize{organique:telechargement-dimages}}
\sphinxAtStartPar
Le téléchargement des images se fait à la création d’un objet de type Game. Le téléchargement se fait à partir du site web de Caiman “caiman.cfpt.info”. Les fichiers sont téléchargés dans le dossier appdata. L’intérêt de stocker les images dans ce dossier est de créer un système de cache qui sera accessible à tous les utilisateurs.  Le téléchargement se fait grâce à un WebClient. Avant tout téléchargement de fichier, Caiman vérifie si le fichier n’est pas déjà existant, ce qui permet d’éviter de télécharger plusieurs fois la même image.

\begin{sphinxVerbatim}[commandchars=\\\{\}]
\PYG{k}{private} \PYG{k}{void} \PYG{n+nf}{DownloadImage}\PYG{p}{(}\PYG{p}{)}
        \PYG{p}{\PYGZob{}}

            \PYG{k}{if} \PYG{p}{(}\PYG{p}{!}\PYG{n}{File}\PYG{p}{.}\PYG{n}{Exists}\PYG{p}{(}\PYG{n}{imgPath} \PYG{p}{+} \PYG{n}{imageName}\PYG{p}{)}\PYG{p}{)}
            \PYG{p}{\PYGZob{}}
                \PYG{k}{using} \PYG{p}{(}\PYG{n}{WebClient} \PYG{n}{client} \PYG{p}{=} \PYG{k}{new} \PYG{n}{WebClient}\PYG{p}{(}\PYG{p}{)}\PYG{p}{)}
                \PYG{p}{\PYGZob{}}
                    \PYG{n}{client}\PYG{p}{.}\PYG{n}{DownloadFile}\PYG{p}{(}\PYG{k}{new} \PYG{n}{Uri}\PYG{p}{(}\PYG{n}{URL\PYGZus{}IMAGES\PYGZus{}CAIMAN} \PYG{p}{+} \PYG{n}{imageName}\PYG{p}{)}\PYG{p}{,} \PYG{p}{(}\PYG{n}{imgPath} \PYG{p}{+} \PYG{n}{imageName}\PYG{p}{)}\PYG{p}{)}\PYG{p}{;}
                \PYG{p}{\PYGZcb{}}
            \PYG{p}{\PYGZcb{}}
        \PYG{p}{\PYGZcb{}}
\end{sphinxVerbatim}


\subsection{Exécution de jeu}
\label{\detokenize{organique:execution-de-jeu}}

\subsubsection{Choix du jeu à lancer}
\label{\detokenize{organique:choix-du-jeu-a-lancer}}
\sphinxAtStartPar
Le choix du jeu se fait à partir de la page de visualisation des détails d’un jeu. Quand l’utilisateur clique sur le bouton “Play” l’id du jeu qui doit être exécuté est envoyé à la fonction EmulatorManager.StartGame(idGame).

\sphinxAtStartPar
La fonction va ensuite faire un appel à l’API pour connaître de quelle console est le jeu qui doit être exécuté et selon la console un émulateur différent sera utilisé.


\subsubsection{Emulateur embarqué avec Caiman}
\label{\detokenize{organique:emulateur-embarque-avec-caiman}}
\sphinxAtStartPar
Caiman embarque deux émulateurs PCSX2 et Dolphin ces émulateurs servent respectivement à l’exécution de jeux de Playstation 2 et de Gamecube/Wii. L’exécution n’est pas la même selon l’émulateur, alors je vais détailler pour chacun. Avant l’exécution d’un jeu, l’application des paramètres graphiques est faite mais je ne vais pas le détailler ici.


\subsubsection{Exécution avec PCSX2}
\label{\detokenize{organique:execution-avec-pcsx2}}
\sphinxAtStartPar
En premier lieu, je vais recréer le chemin qui mène au fichier qui doit être exécuté par exemple “C:\textbackslash{}Caiman\textbackslash{}Playstation2\textbackslash{}DRAGON\_QUEST\_VIII.iso”. Par la suite, je vais créer une variable pour les différents paramètres, dans la version de caiman actuel on ne peut pas changer ces paramètres se sera donc toujours “ –nogui –portable”.

\sphinxAtStartPar
Le premier paramètre fait en sorte que l’émulateur n’affiche pas d’interface graphique, seule la fenêtre d’exécution du jeu est visible. Le deuxième paramètre sert à dire à PCSX2 que les fichiers de configurations qui doivent être utilisés sont ceux du dossier de l’émulateur, et non ceux du dossier créé pour chaque utilisateur windows par PCSX2.

\sphinxAtStartPar
Pour lancer l’exécution, je vais lancer le processus de PCSX2.exe avec en premier paramètre le chemin de l’iso à exécuter et après les paramètres cités précédemment.

\begin{sphinxVerbatim}[commandchars=\\\{\}]
\PYG{k}{public} \PYG{k}{override} \PYG{k}{void} \PYG{n+nf}{Execute}\PYG{p}{(}\PYG{k+kt}{int} \PYG{n}{idGame}\PYG{p}{)}
        \PYG{p}{\PYGZob{}}
            \PYG{k+kt}{string} \PYG{n}{path} \PYG{p}{=} \PYG{l+s}{@\PYGZdq{}C:\PYGZbs{}Caiman\PYGZbs{}\PYGZdq{}} \PYG{p}{+} \PYG{n}{callAPI}\PYG{p}{.}\PYG{n}{CallFolderNameGame}\PYG{p}{(}\PYG{n}{idGame}\PYG{p}{)} \PYG{p}{+} \PYG{l+s}{@\PYGZdq{}\PYGZbs{}\PYGZdq{}}\PYG{p}{;}
            \PYG{k+kt}{string} \PYG{n}{filename} \PYG{p}{=} \PYG{n}{callAPI}\PYG{p}{.}\PYG{n}{CallFileNameGame}\PYG{p}{(}\PYG{n}{idGame}\PYG{p}{)}\PYG{p}{;}
            \PYG{n}{UpdateConfigurationFile}\PYG{p}{(}\PYG{p}{)}\PYG{p}{;}
            \PYG{k+kt}{int} \PYG{n}{process} \PYG{p}{=} \PYG{n}{Process}\PYG{p}{.}\PYG{n}{GetProcessesByName}\PYG{p}{(}\PYG{n}{PROCESS\PYGZus{}NAME}\PYG{p}{)}\PYG{p}{.}\PYG{n}{Length}\PYG{p}{;}

            \PYG{k}{if} \PYG{p}{(}\PYG{n}{process} \PYG{p}{=}\PYG{p}{=} \PYG{l+m}{0}\PYG{p}{)}
            \PYG{p}{\PYGZob{}}
                \PYG{c+c1}{//param pour ne pas afficher l\PYGZsq{}interface graphique \PYGZhy{}\PYGZhy{}portable \PYGZhy{}\PYGZhy{}nogui}
                \PYG{k+kt}{string} \PYG{n}{param} \PYG{p}{=} \PYG{l+s}{\PYGZdq{} \PYGZhy{}\PYGZhy{}nogui \PYGZhy{}\PYGZhy{}portable\PYGZdq{}}\PYG{p}{;}

                \PYG{n}{processEmulator} \PYG{p}{=} \PYG{n}{Process}\PYG{p}{.}\PYG{n}{Start}\PYG{p}{(}\PYG{n}{PCSX2Folder} \PYG{p}{+} \PYG{n}{EXE\PYGZus{}NAME}\PYG{p}{,} \PYG{n}{path} \PYG{p}{+} \PYG{n}{filename} \PYG{p}{+} \PYG{n}{param}\PYG{p}{)}\PYG{p}{;}

            \PYG{p}{\PYGZcb{}}
            \PYG{k}{else}
            \PYG{p}{\PYGZob{}}
                \PYG{n}{Close}\PYG{p}{(}\PYG{p}{)}\PYG{p}{;}
                \PYG{n}{Execute}\PYG{p}{(}\PYG{n}{idGame}\PYG{p}{)}\PYG{p}{;}
            \PYG{p}{\PYGZcb{}}
        \PYG{p}{\PYGZcb{}}
\end{sphinxVerbatim}


\subsubsection{Exécution avec Dolphin}
\label{\detokenize{organique:execution-avec-dolphin}}
\sphinxAtStartPar
Ensuite, je vais recréer le chemin qui mène au fichier qui doit être exécuté par exemple “C:\textbackslash{}Caiman\textbackslash{}GamecubeWii\textbackslash{}METROID\_PRIME.iso”. Je vais également créer une variable pour les différents paramètres, dans la version de caiman actuel on ne peut pas changer ces paramètres, il sera toujours “ –batch”.

\sphinxAtStartPar
Le paramètre “–batch” fait en sorte de ne pas afficher l’interface de Dolphin, et donc d’avoir seulement la fenêtre d’exécution du jeu.

\sphinxAtStartPar
Pour lancer l’exécution, je vais lancer le processus de PCSX2.exe avec en premier paramètre le chemin de l’iso a exécuter et après le paramètre cité précédemment.

\begin{sphinxVerbatim}[commandchars=\\\{\}]
\PYG{k}{public} \PYG{k}{override} \PYG{k}{void} \PYG{n+nf}{Execute}\PYG{p}{(}\PYG{k+kt}{int} \PYG{n}{idGame}\PYG{p}{)}
        \PYG{p}{\PYGZob{}}
            \PYG{k+kt}{string} \PYG{n}{path} \PYG{p}{=} \PYG{l+s}{@\PYGZdq{}C:\PYGZbs{}Caiman\PYGZbs{}\PYGZdq{}} \PYG{p}{+} \PYG{n}{callAPI}\PYG{p}{.}\PYG{n}{CallFolderNameGame}\PYG{p}{(}\PYG{n}{idGame}\PYG{p}{)} \PYG{p}{+} \PYG{l+s}{@\PYGZdq{}\PYGZbs{}\PYGZdq{}}\PYG{p}{;}
            \PYG{k+kt}{string} \PYG{n}{filename} \PYG{p}{=} \PYG{n}{callAPI}\PYG{p}{.}\PYG{n}{CallFileNameGame}\PYG{p}{(}\PYG{n}{idGame}\PYG{p}{)}\PYG{p}{;}
            
            \PYG{n}{UpdateConfigurationFile}\PYG{p}{(}\PYG{p}{)}\PYG{p}{;}
            \PYG{k+kt}{int} \PYG{n}{process} \PYG{p}{=} \PYG{n}{Process}\PYG{p}{.}\PYG{n}{GetProcessesByName}\PYG{p}{(}\PYG{n}{PROCESS\PYGZus{}NAME}\PYG{p}{)}\PYG{p}{.}\PYG{n}{Length}\PYG{p}{;}

            \PYG{k}{if} \PYG{p}{(}\PYG{n}{process} \PYG{p}{=}\PYG{p}{=} \PYG{l+m}{0}\PYG{p}{)}
            \PYG{p}{\PYGZob{}}
                \PYG{c+c1}{//param pour ne pas mettre de gui et en fullscreen \PYGZhy{}\PYGZhy{}portable}
                \PYG{k+kt}{string} \PYG{n}{param} \PYG{p}{=} \PYG{l+s}{\PYGZdq{} \PYGZhy{}\PYGZhy{}batch\PYGZdq{}}\PYG{p}{;}

                \PYG{n}{processEmulator} \PYG{p}{=} \PYG{n}{Process}\PYG{p}{.}\PYG{n}{Start}\PYG{p}{(}\PYG{n}{dolphinFolder} \PYG{p}{+} \PYG{n}{EXE\PYGZus{}NAME}\PYG{p}{,} \PYG{n}{param} \PYG{p}{+} \PYG{l+s}{\PYGZdq{} \PYGZhy{}\PYGZhy{}exec \PYGZbs{}\PYGZdq{}\PYGZdq{}} \PYG{p}{+} \PYG{n}{path} \PYG{p}{+} \PYG{n}{filename}\PYG{p}{)}\PYG{p}{;}

            \PYG{p}{\PYGZcb{}}
            \PYG{k}{else}
            \PYG{p}{\PYGZob{}}
                \PYG{n}{Close}\PYG{p}{(}\PYG{p}{)}\PYG{p}{;}
                \PYG{n}{Execute}\PYG{p}{(}\PYG{n}{idGame}\PYG{p}{)}\PYG{p}{;}
            \PYG{p}{\PYGZcb{}}
        \PYG{p}{\PYGZcb{}}
\end{sphinxVerbatim}


\subsubsection{Information complémentaire}
\label{\detokenize{organique:information-complementaire}}
\sphinxAtStartPar
Il n’est pas possible de lancer un jeu alors qu” un autre est toujours en cours.


\subsection{Affichage et calcul du temps de jeu}
\label{\detokenize{organique:affichage-et-calcul-du-temps-de-jeu}}
\sphinxAtStartPar
Le temps de jeu est comptabilisé dans la base de données à la minute près.


\subsubsection{Affichage dans le détail d’un jeu}
\label{\detokenize{organique:affichage-dans-le-detail-dun-jeu}}
\sphinxAtStartPar
Quand un utilisateur de Caiman se rend sur la page de détails d’un jeu, il verra son nombre d’heures et de minutes de jeu. Pour récupérer cette information, je fais un appel à la base de données .CallTimeInGameUser(idGame,idUser) cette appel va me rendre un objet TimeInGame qui va permettre de formater la réponse de l’API et afficher les heures et le minutes sous le format “10h10”. Si l’utilisateur n’a pas jouer au jeu, il verra afficher “Time played: 00h00”.

\begin{sphinxVerbatim}[commandchars=\\\{\}]
\PYG{k}{public} \PYG{k+kt}{string} \PYG{n}{TimeHoursMinutes}
        \PYG{p}{\PYGZob{}}
            \PYG{k}{get}
            \PYG{p}{\PYGZob{}}
                \PYG{k+kt}{string} \PYG{n}{time} \PYG{p}{=} \PYG{l+s}{\PYGZdq{}\PYGZdq{}}\PYG{p}{;}
                \PYG{k+kt}{int} \PYG{n}{hours} \PYG{p}{=} \PYG{k}{this}\PYG{p}{.}\PYG{n}{minutes} \PYG{p}{/} \PYG{l+m}{6}\PYG{l+m}{0}\PYG{p}{;}
                \PYG{k+kt}{int} \PYG{n}{minutesInt} \PYG{p}{=} \PYG{k}{this}\PYG{p}{.}\PYG{n}{minutes} \PYG{p}{\PYGZpc{}} \PYG{l+m}{6}\PYG{l+m}{0}\PYG{p}{;}
                \PYG{k}{if} \PYG{p}{(}\PYG{n}{minutesInt} \PYG{p}{=}\PYG{p}{=} \PYG{l+m}{6}\PYG{l+m}{0}\PYG{p}{)}
                \PYG{p}{\PYGZob{}}
                    \PYG{n}{hours}\PYG{p}{+}\PYG{p}{+}\PYG{p}{;}
                    \PYG{n}{minutesInt} \PYG{p}{=} \PYG{l+m}{0}\PYG{p}{;}
                \PYG{p}{\PYGZcb{}}
                \PYG{k+kt}{string} \PYG{n}{minutesString} \PYG{p}{=} \PYG{n}{minutesInt}\PYG{p}{.}\PYG{n}{ToString}\PYG{p}{(}\PYG{p}{)}\PYG{p}{;}
                \PYG{k+kt}{string} \PYG{n}{hoursString} \PYG{p}{=} \PYG{l+s}{\PYGZdq{}\PYGZdq{}}\PYG{p}{;}
                \PYG{k}{if} \PYG{p}{(}\PYG{n}{minutesInt} \PYG{p}{\PYGZlt{}} \PYG{l+m}{1}\PYG{l+m}{0}\PYG{p}{)}
                \PYG{p}{\PYGZob{}}
                    \PYG{n}{minutesString} \PYG{p}{=} \PYG{l+s}{\PYGZdq{}0\PYGZdq{}} \PYG{p}{+} \PYG{n}{minutesInt}\PYG{p}{;}
                \PYG{p}{\PYGZcb{}}
                \PYG{k}{if} \PYG{p}{(}\PYG{n}{hours} \PYG{p}{\PYGZlt{}} \PYG{l+m}{1}\PYG{l+m}{0}\PYG{p}{)}
                \PYG{p}{\PYGZob{}}
                    \PYG{n}{hoursString} \PYG{p}{=} \PYG{l+s}{\PYGZdq{}0\PYGZdq{}} \PYG{p}{+} \PYG{n}{hours}\PYG{p}{;}
                \PYG{p}{\PYGZcb{}}
                \PYG{n}{time} \PYG{p}{=} \PYG{n}{hoursString} \PYG{p}{+} \PYG{l+s}{\PYGZdq{} h \PYGZdq{}} \PYG{p}{+} \PYG{n}{minutesString}\PYG{p}{;}
                \PYG{k}{return} \PYG{n}{time}\PYG{p}{;}
            \PYG{p}{\PYGZcb{}}
        \PYG{p}{\PYGZcb{}}
\end{sphinxVerbatim}


\subsubsection{Affichage du temps de jeu actuel}
\label{\detokenize{organique:affichage-du-temps-de-jeu-actuel}}
\sphinxAtStartPar
Quand un utilisateur lance un jeu, un objet GameTimer va être créé, cet objet sert à compter le temps de jeu de la session actuelle et à mettre à jour la base de données.

\sphinxAtStartPar
La classe GameTimer va initialiser in timer pour rafraîchir l’affichage de la navbar avec la valeur adéquate. Ce timer se rafraîchit toutes les secondes. Celui\sphinxhyphen{}ci va également appeler la fonction UpdateTimer, UpdateTimer va incrémenter le nombre de secondes de jeu et toutes les 60 secondes, il va faire un appel à l’API pour incrémenter le temps de jeu de l’utilisateur dans la base de données. L’incrémentation se fait donc toutes les minutes, cela permet d’être assez précis dans le décompte de temps de jeu.

\begin{sphinxVerbatim}[commandchars=\\\{\}]
\PYG{k}{public} \PYG{k}{void} \PYG{n+nf}{InitTimer}\PYG{p}{(}\PYG{p}{)}
        \PYG{p}{\PYGZob{}}
            \PYG{n}{timer} \PYG{p}{=} \PYG{k}{new} \PYG{n}{Timer}\PYG{p}{(}\PYG{p}{)}\PYG{p}{;}
            \PYG{n}{timer}\PYG{p}{.}\PYG{n}{Tick} \PYG{p}{+}\PYG{p}{=} \PYG{k}{new} \PYG{n}{EventHandler}\PYG{p}{(}\PYG{n}{UpdateTimer}\PYG{p}{)}\PYG{p}{;}

            \PYG{n}{timer}\PYG{p}{.}\PYG{n}{Interval} \PYG{p}{=} \PYG{l+m}{1}\PYG{l+m}{0}\PYG{l+m}{0}\PYG{l+m}{0}\PYG{p}{;}
            \PYG{n}{timer}\PYG{p}{.}\PYG{n}{Start}\PYG{p}{(}\PYG{p}{)}\PYG{p}{;}
        \PYG{p}{\PYGZcb{}}

\PYG{k}{public} \PYG{k}{void} \PYG{n+nf}{UpdateTimer}\PYG{p}{(}\PYG{k+kt}{object} \PYG{n}{sender}\PYG{p}{,} \PYG{n}{EventArgs} \PYG{n}{e}\PYG{p}{)}
        \PYG{p}{\PYGZob{}}
            \PYG{n}{counter}\PYG{p}{+}\PYG{p}{+}\PYG{p}{;}

            \PYG{k}{if} \PYG{p}{(}\PYG{n}{counter} \PYG{p}{=}\PYG{p}{=} \PYG{l+m}{6}\PYG{l+m}{0}\PYG{p}{)}
            \PYG{p}{\PYGZob{}}
                \PYG{n}{minutes}\PYG{p}{+}\PYG{p}{+}\PYG{p}{;}
                \PYG{n}{counter} \PYG{p}{=} \PYG{l+m}{0}\PYG{p}{;}

                \PYG{n}{callAPI}\PYG{p}{.}\PYG{n}{AddOneMinuteToGame}\PYG{p}{(}\PYG{n}{game}\PYG{p}{.}\PYG{n}{id}\PYG{p}{,} \PYG{n}{emulatorsManager}\PYG{p}{.}\PYG{n}{user}\PYG{p}{.}\PYG{n}{id}\PYG{p}{)}\PYG{p}{;}
            \PYG{p}{\PYGZcb{}}

        \PYG{p}{\PYGZcb{}}
\end{sphinxVerbatim}

\sphinxAtStartPar
L’affichage de temps de jeu se fait dans la navbar en haut à gauche, Les informations affichées sont les suivantes:
\begin{itemize}
\item {} 
\sphinxAtStartPar
Nom du jeu en cours

\item {} 
\sphinxAtStartPar
heures et minutes de jeu de la session actuel sous le format 10m50

\end{itemize}


\subsection{Téléchargement de jeux}
\label{\detokenize{organique:telechargement-de-jeux}}

\subsubsection{Type de fichiers}
\label{\detokenize{organique:type-de-fichiers}}
\sphinxAtStartPar
Les fichiers utilisés par les émulateurs sont des fichiers .iso, que ce soit par PCSX2 ou Dolphin.


\subsubsection{Choix du fichier à télécharger}
\label{\detokenize{organique:choix-du-fichier-a-telecharger}}
\sphinxAtStartPar
Le fichier à télécharger se fait sur la page de détails d’un jeu, si le jeu n’a pas déjà été téléchargé. Un bouton “download” sera présent, ce bouton va envoyer l’id du jeu qui doit être télécharger.

\begin{sphinxVerbatim}[commandchars=\\\{\}]
\PYG{k}{if} \PYG{p}{(}\PYG{p}{!}\PYG{n}{File}\PYG{p}{.}\PYG{n}{Exists}\PYG{p}{(}\PYG{n}{gamePath}\PYG{p}{)}\PYG{p}{)}
                \PYG{p}{\PYGZob{}}
                    \PYG{n}{XboxButton} \PYG{n}{btn\PYGZus{}download} \PYG{p}{=} \PYG{k}{new} \PYG{n}{XboxButton}\PYG{p}{(}\PYG{l+s}{\PYGZdq{}download\PYGZdq{}}\PYG{p}{,} \PYG{n}{game}\PYG{p}{.}\PYG{n}{id}\PYG{p}{,} \PYG{l+m}{0}\PYG{p}{,} \PYG{l+m}{0}\PYG{p}{)}\PYG{p}{;}
                    \PYG{n}{btn\PYGZus{}download}\PYG{p}{.}\PYG{n}{Text} \PYG{p}{=} \PYG{l+s}{\PYGZdq{}Download: \PYGZdq{}} \PYG{p}{+} \PYG{n}{game}\PYG{p}{.}\PYG{n}{name}\PYG{p}{;}
                    \PYG{n}{btn\PYGZus{}download}\PYG{p}{.}\PYG{n}{Location} \PYG{p}{=} \PYG{k}{new} \PYG{n}{System}\PYG{p}{.}\PYG{n}{Drawing}\PYG{p}{.}\PYG{n}{Point}\PYG{p}{(}\PYG{l+m}{5}\PYG{l+m}{0}\PYG{l+m}{0}\PYG{p}{,} \PYG{l+m}{6}\PYG{l+m}{5}\PYG{l+m}{0}\PYG{p}{)}\PYG{p}{;}
                    \PYG{n}{btn\PYGZus{}download}\PYG{p}{.}\PYG{n}{Click} \PYG{p}{+}\PYG{p}{=} \PYG{k}{new} \PYG{n}{System}\PYG{p}{.}\PYG{n}{EventHandler}\PYG{p}{(}\PYG{n}{bouton\PYGZus{}Click}\PYG{p}{)}\PYG{p}{;}
                    \PYG{n}{lstControls}\PYG{p}{[}\PYG{n}{rowCounter}\PYG{p}{]}\PYG{p}{.}\PYG{n}{Add}\PYG{p}{(}\PYG{n}{btn\PYGZus{}download}\PYG{p}{)}\PYG{p}{;}
                    \PYG{n}{Controls}\PYG{p}{.}\PYG{n}{Add}\PYG{p}{(}\PYG{n}{btn\PYGZus{}download}\PYG{p}{)}\PYG{p}{;}
                    \PYG{n}{rowCounter}\PYG{p}{+}\PYG{p}{+}\PYG{p}{;}
                    \PYG{n}{lstControls}\PYG{p}{.}\PYG{n}{Add}\PYG{p}{(}\PYG{k}{new} \PYG{n}{List}\PYG{p}{\PYGZlt{}}\PYG{n}{Control}\PYG{p}{\PYGZgt{}}\PYG{p}{(}\PYG{p}{)}\PYG{p}{)}\PYG{p}{;}
                \PYG{p}{\PYGZcb{}}
\end{sphinxVerbatim}


\subsubsection{Ajout d’un jeu à la liste de téléchargement}
\label{\detokenize{organique:ajout-dun-jeu-a-la-liste-de-telechargement}}
\sphinxAtStartPar
Les téléchargement des jeux sont gérés par la classe “DownloadManager”, cette classe contient 3 listes de téléchargement différentes:
\begin{itemize}
\item {} 
\sphinxAtStartPar
lst\_download

\item {} 
\sphinxAtStartPar
lst\_activeDownload

\item {} 
\sphinxAtStartPar
lst\_finishDownload

\end{itemize}

\sphinxAtStartPar
La première contient les téléchargement en attente, la deuxième le téléchargement en cours et la troisième la liste des téléchargements terminés.

\sphinxAtStartPar
Quand un téléchargement est créé, il est automatiquement ajouté à la liste de téléchargement en attente.


\subsubsection{Lancement d’un téléchargement}
\label{\detokenize{organique:lancement-dun-telechargement}}
\sphinxAtStartPar
À l’ajout d’un téléchargement, la fonction StartDownload() est appelée. Cette fonction vérifie si dans la liste de téléchargement il sera lancé. Si aucun téléchargement n’est en cours, la fonction NextDownload() va être lancée.

\sphinxAtStartPar
La fonction NextDownload() sert à savoir si un téléchargement est en cours,Si un téléchargement est en cours, il sera déplacé dans la liste des téléchargements finis, si aucun téléchargement n’est en cours le premier téléchargement de la liste en attente est lancé


\subsubsection{Téléchargement d’un fichier}
\label{\detokenize{organique:telechargement-dun-fichier}}
\sphinxAtStartPar
Quand la fonction download.StartDownload() est lancée, le WebClient se lance également. Ce WebClient télécharge le fichier en appelant l’API de caiman en passant en paramètres l’id du jeu qui doit être télécharger et l’apiKey de l’utilisateur.

\begin{sphinxVerbatim}[commandchars=\\\{\}]
\PYG{k}{public} \PYG{k}{void} \PYG{n+nf}{StartDownload}\PYG{p}{(}\PYG{p}{)}
        \PYG{p}{\PYGZob{}}
            
            \PYG{k}{if} \PYG{p}{(}\PYG{p}{!}\PYG{n}{CheckIfFileIsPresent}\PYG{p}{(}\PYG{p}{)}\PYG{p}{)}
            \PYG{p}{\PYGZob{}}
                \PYG{n}{webClient} \PYG{p}{=} \PYG{k}{new} \PYG{n}{WebClient}\PYG{p}{(}\PYG{p}{)}\PYG{p}{;}
                \PYG{n}{Uri} \PYG{n}{uri} \PYG{p}{=} \PYG{k}{new} \PYG{n}{Uri}\PYG{p}{(}\PYG{n}{URL\PYGZus{}TO\PYGZus{}GAMES} \PYG{p}{+} \PYG{l+s}{\PYGZdq{}?idGame=\PYGZdq{}}\PYG{p}{+}\PYG{n}{idGame}\PYG{p}{+}\PYG{l+s}{\PYGZdq{}\PYGZam{}apiKey=\PYGZdq{}}\PYG{p}{+}\PYG{n}{apiKey}\PYG{p}{)}\PYG{p}{;}
                \PYG{n}{webClient}\PYG{p}{.}\PYG{n}{DownloadProgressChanged} \PYG{p}{+}\PYG{p}{=} \PYG{n}{wc\PYGZus{}DownloadProgressChanged}\PYG{p}{;}
                \PYG{n}{webClient}\PYG{p}{.}\PYG{n}{DownloadFileAsync}\PYG{p}{(}\PYG{n}{uri}\PYG{p}{,}\PYG{n}{pathToFolder}\PYG{p}{+}\PYG{l+s}{\PYGZdq{}temp.\PYGZdq{}}\PYG{p}{+}\PYG{n}{filename}\PYG{p}{)}\PYG{p}{;}
                \PYG{n}{active} \PYG{p}{=} \PYG{k}{true}\PYG{p}{;}
            \PYG{p}{\PYGZcb{}}
        \PYG{p}{\PYGZcb{}}
\end{sphinxVerbatim}

\sphinxAtStartPar
Le fichier sera téléchargé dans le dossier spécifique à l’émulateur, par exemple si un jeu pour l’émulateur PCSX2 est téléchargé alors le jeu sera télécharger dans le dossier

\sphinxAtStartPar
“C:\textbackslash{}Caiman\textbackslash{}Playstation2”. Par contre, le fichier n’est pas téléchargé avec le nom final, le préfix “temp.” est ajouté.Le préfixe est ajouté pour éviter que si un utilisateur quitte Caiman pendant un téléchargement, il ne se retrouve pas avec un fichier incomplet au prochain lancement de l’application.

\sphinxAtStartPar
A la fin du téléchargement d’un fichier, la fonction DownloadManager.NextDownload() est appelée. l’id du jeu est ajouté à la liste des jeux téléchargés, et le fichier est renommé avec le nom correct.

\begin{sphinxVerbatim}[commandchars=\\\{\}]
\PYG{k}{public} \PYG{k}{void} \PYG{n+nf}{NextDownload}\PYG{p}{(}\PYG{p}{)}
        \PYG{p}{\PYGZob{}}
            \PYG{k}{if} \PYG{p}{(}\PYG{n}{lst\PYGZus{}activeDownload}\PYG{p}{.}\PYG{n}{Count} \PYG{p}{\PYGZgt{}}\PYG{l+m}{0}\PYG{p}{)}
            \PYG{p}{\PYGZob{}}
                \PYG{n}{lst\PYGZus{}finishDownload}\PYG{p}{.}\PYG{n}{Add}\PYG{p}{(}\PYG{n}{lst\PYGZus{}activeDownload}\PYG{p}{[}\PYG{l+m}{0}\PYG{p}{]}\PYG{p}{)}\PYG{p}{;}
                \PYG{n}{lst\PYGZus{}activeDownload}\PYG{p}{.}\PYG{n}{RemoveAt}\PYG{p}{(}\PYG{l+m}{0}\PYG{p}{)}\PYG{p}{;}
            \PYG{p}{\PYGZcb{}}

            \PYG{k}{if} \PYG{p}{(}\PYG{n}{lst\PYGZus{}download}\PYG{p}{.}\PYG{n}{Count}\PYG{p}{(}\PYG{p}{)} \PYG{p}{=}\PYG{p}{=} \PYG{l+m}{1}\PYG{p}{)}
            \PYG{p}{\PYGZob{}}
                \PYG{n}{lst\PYGZus{}activeDownload}\PYG{p}{.}\PYG{n}{Add}\PYG{p}{(}\PYG{n}{lst\PYGZus{}download}\PYG{p}{[}\PYG{l+m}{0}\PYG{p}{]}\PYG{p}{)}\PYG{p}{;}
                \PYG{n}{lst\PYGZus{}download}\PYG{p}{.}\PYG{n}{RemoveAt}\PYG{p}{(}\PYG{l+m}{0}\PYG{p}{)}\PYG{p}{;}
                \PYG{n}{StartDownload}\PYG{p}{(}\PYG{p}{)}\PYG{p}{;}
            \PYG{p}{\PYGZcb{}}
            
            
        \PYG{p}{\PYGZcb{}}
\end{sphinxVerbatim}


\subsection{Synchronisation des sauvegardes}
\label{\detokenize{organique:synchronisation-des-sauvegardes}}

\subsubsection{Structure des sauvegardes}
\label{\detokenize{organique:structure-des-sauvegardes}}
\sphinxAtStartPar
Les fichiers de sauvegardes des émulateurs sont faits ainsi:

\sphinxAtStartPar
PCSX2:

\sphinxAtStartPar
Les fichiers de sauvegardes pour PCSX2 sont 2 fichiers de 8MB, comme  les memory cards étaient à l’époque de la Playstation 2. Ces deux fichiers contiennent les différentes sauvegardes des jeux.

\sphinxAtStartPar
\sphinxincludegraphics{{dossier_memcards}.png}

\sphinxAtStartPar
Dolphin:

\sphinxAtStartPar
Les fichiers de sauvegardes de l’émulateur Dolphin ne concernent qu’un seul jeu à la fois.

\sphinxAtStartPar
\sphinxincludegraphics{{dossier_card_a}.png}

\sphinxAtStartPar
La solution pour structurer les sauvegardes des différents utilisateurs est celle\sphinxhyphen{}ci:

\sphinxAtStartPar
Chaque utilisateur possède un dossier personnel dans le dossier appdata de Caiman. Ce dossier contient le fichier de configuration graphique de l’application et un dossier ou ces sauvegardes sont présentes.

\sphinxAtStartPar
Le dossier de l’utilisateur lorenzo1227:

\sphinxAtStartPar
\sphinxincludegraphics{{dossier_lorenzo1227}.png}

\sphinxAtStartPar
Le dossier contenant les sauvegardes de l’utilisateur lorenzo1227:

\sphinxAtStartPar
\sphinxincludegraphics{{save_lorenzo1227}.png}


\subsubsection{Upload des sauvegardes}
\label{\detokenize{organique:upload-des-sauvegardes}}
\sphinxAtStartPar
Pour pouvoir synchroniser les sauvegardes entre les différents PCs d’un utilisateur, j’ai commencé par savoir si l’un des fichiers présents dans les dossiers des différents émulateurs a été mis à jour. Pour ce faire, j’ai un timer qui va faire une vérification sur la dernière date de modification du fichier. S’il s’avère qu’une modification a été faite dans l’un des dossiers de sauvegardes, je copie l’intégralité du dossier de l’émulateur vers le dossier appdata de l’utilisateur.

\begin{sphinxVerbatim}[commandchars=\\\{\}]
\PYG{k}{private} \PYG{k}{void} \PYG{n+nf}{ScanDateFile}\PYG{p}{(}\PYG{p}{)}
        \PYG{p}{\PYGZob{}}
            \PYG{k+kt}{int} \PYG{n}{counter} \PYG{p}{=} \PYG{l+m}{0}\PYG{p}{;}
            \PYG{n}{lst\PYGZus{}saveTimeOld}\PYG{p}{.}\PYG{n}{Clear}\PYG{p}{(}\PYG{p}{)}\PYG{p}{;}
            \PYG{n}{lst\PYGZus{}saveTimeOld}\PYG{p}{.}\PYG{n}{AddRange}\PYG{p}{(}\PYG{n}{lst\PYGZus{}saveTimeNow}\PYG{p}{)}\PYG{p}{;}
            \PYG{n}{lst\PYGZus{}saveTimeNow}\PYG{p}{.}\PYG{n}{Clear}\PYG{p}{(}\PYG{p}{)}\PYG{p}{;}
            \PYG{c+c1}{//if a file has been added or delete sync the folder}
            \PYG{k}{if} \PYG{p}{(}\PYG{n}{initialCounterFile} \PYG{p}{!}\PYG{p}{=} \PYG{n}{CountFileInFolder}\PYG{p}{(}\PYG{p}{)}\PYG{p}{)}
            \PYG{p}{\PYGZob{}}
                \PYG{k}{if} \PYG{p}{(}\PYG{n}{isLocalFile}\PYG{p}{)}
                \PYG{p}{\PYGZob{}}
                    \PYG{n}{initialCounterFile} \PYG{p}{=} \PYG{n}{CountFileInFolder}\PYG{p}{(}\PYG{p}{)}\PYG{p}{;}
                    \PYG{n}{MoveAllFileToUserFolder}\PYG{p}{(}\PYG{p}{)}\PYG{p}{;}
                \PYG{p}{\PYGZcb{}}
            \PYG{p}{\PYGZcb{}}
            \PYG{k}{foreach} \PYG{p}{(}\PYG{n}{FileInfo} \PYG{n}{save} \PYG{k}{in} \PYG{n}{lst\PYGZus{}save}\PYG{p}{)}
            \PYG{p}{\PYGZob{}}

                \PYG{n}{save}\PYG{p}{.}\PYG{n}{Refresh}\PYG{p}{(}\PYG{p}{)}\PYG{p}{;}

                \PYG{n}{lst\PYGZus{}saveTimeNow}\PYG{p}{.}\PYG{n}{Add}\PYG{p}{(}\PYG{n}{save}\PYG{p}{.}\PYG{n}{LastWriteTime}\PYG{p}{.}\PYG{n}{Ticks}\PYG{p}{.}\PYG{n}{ToString}\PYG{p}{(}\PYG{p}{)}\PYG{p}{)}\PYG{p}{;}
                \PYG{k}{if} \PYG{p}{(}\PYG{n}{lst\PYGZus{}saveTimeOld}\PYG{p}{.}\PYG{n}{Count}\PYG{p}{(}\PYG{p}{)} \PYG{p}{\PYGZgt{}} \PYG{l+m}{0}\PYG{p}{)}
                \PYG{p}{\PYGZob{}}
                    \PYG{k}{try}
                    \PYG{p}{\PYGZob{}}
                        \PYG{k}{if} \PYG{p}{(}\PYG{n}{lst\PYGZus{}saveTimeOld}\PYG{p}{[}\PYG{n}{counter}\PYG{p}{]} \PYG{p}{!}\PYG{p}{=} \PYG{n}{lst\PYGZus{}saveTimeNow}\PYG{p}{[}\PYG{n}{counter}\PYG{p}{]}\PYG{p}{)}
                        \PYG{p}{\PYGZob{}}
                            \PYG{c+c1}{//if the file has benn modified since the last time move or sync the folder}
                            \PYG{k}{if} \PYG{p}{(}\PYG{n}{lst\PYGZus{}saveTimeOld}\PYG{p}{[}\PYG{n}{counter}\PYG{p}{]} \PYG{p}{!}\PYG{p}{=} \PYG{l+s}{\PYGZdq{}\PYGZdq{}}\PYG{p}{)}
                            \PYG{p}{\PYGZob{}}
                                \PYG{k}{if} \PYG{p}{(}\PYG{n}{isLocalFile}\PYG{p}{)}
                                \PYG{p}{\PYGZob{}}
                                    \PYG{n}{MoveFileToUserFolder}\PYG{p}{(}\PYG{n}{save}\PYG{p}{)}\PYG{p}{;}
                                \PYG{p}{\PYGZcb{}}
                                \PYG{k}{else}
                                \PYG{p}{\PYGZob{}}
                                    \PYG{n}{UploadSave}\PYG{p}{(}\PYG{p}{)}\PYG{p}{;}
                                \PYG{p}{\PYGZcb{}}
                            \PYG{p}{\PYGZcb{}}

                        \PYG{p}{\PYGZcb{}}
                        \PYG{n}{counter}\PYG{p}{+}\PYG{p}{+}\PYG{p}{;}
                    \PYG{p}{\PYGZcb{}}
                    \PYG{k}{catch} \PYG{p}{\PYGZob{}} \PYG{p}{\PYGZcb{}}
                \PYG{p}{\PYGZcb{}}


            \PYG{p}{\PYGZcb{}}
        \PYG{p}{\PYGZcb{}}

        \PYG{k}{public} \PYG{k}{void} \PYG{n+nf}{MoveAllFileToUserFolder}\PYG{p}{(}\PYG{p}{)}
        \PYG{p}{\PYGZob{}}
            \PYG{n}{DirectoryInfo} \PYG{n}{d} \PYG{p}{=} \PYG{k}{new} \PYG{n}{DirectoryInfo}\PYG{p}{(}\PYG{n}{savePath}\PYG{p}{)}\PYG{p}{;}
            \PYG{n}{FileInfo}\PYG{p}{[}\PYG{p}{]} \PYG{n}{Files} \PYG{p}{=} \PYG{n}{d}\PYG{p}{.}\PYG{n}{GetFiles}\PYG{p}{(}\PYG{p}{)}\PYG{p}{;} 
            \PYG{k}{foreach} \PYG{p}{(}\PYG{n}{FileInfo} \PYG{n}{file} \PYG{k}{in} \PYG{n}{Files}\PYG{p}{)}
            \PYG{p}{\PYGZob{}}
                \PYG{k}{try}
                \PYG{p}{\PYGZob{}}

                    \PYG{n}{File}\PYG{p}{.}\PYG{n}{Copy}\PYG{p}{(}\PYG{n}{file}\PYG{p}{.}\PYG{n}{FullName}\PYG{p}{,} \PYG{n}{Path}\PYG{p}{.}\PYG{n}{Combine}\PYG{p}{(}\PYG{n}{destinationPath}\PYG{p}{,} \PYG{n}{file}\PYG{p}{.}\PYG{n}{Name}\PYG{p}{)}\PYG{p}{,} \PYG{k}{true}\PYG{p}{)}\PYG{p}{;}
                \PYG{p}{\PYGZcb{}}
                \PYG{k}{catch} \PYG{p}{(}\PYG{n}{Exception}\PYG{p}{)}
                \PYG{p}{\PYGZob{}}
                \PYG{p}{\PYGZcb{}}
            \PYG{p}{\PYGZcb{}}
        \PYG{p}{\PYGZcb{}}
\end{sphinxVerbatim}

\sphinxAtStartPar
Si je trouve une différence alors cela veut dire que le fichier a été modifié. Quand un fichier a été modifié je crée une copie de ce fichier dans le dossier “username/save/nom\_de\_emulateur/”.

\sphinxAtStartPar
Quand un fichier a été mis à jour, supprimé ou créé dans le dossier de sauvegarde de l’utilisateur, je zip les fichiers présents dans le dossier pour les envoyer sur le serveur de Caiman.

\begin{sphinxVerbatim}[commandchars=\\\{\}]
\PYG{k}{public} \PYG{k}{void} \PYG{n+nf}{UploadSave}\PYG{p}{(}\PYG{p}{)}
        \PYG{p}{\PYGZob{}}

            \PYG{k+kt}{var} \PYG{n}{appDataPath} \PYG{p}{=} \PYG{n}{Environment}\PYG{p}{.}\PYG{n}{GetFolderPath}\PYG{p}{(}\PYG{n}{Environment}\PYG{p}{.}\PYG{n}{SpecialFolder}\PYG{p}{.}\PYG{n}{ApplicationData}\PYG{p}{)}\PYG{p}{;}
            \PYG{k+kt}{var} \PYG{n}{savePathZipDolpin} \PYG{p}{=} \PYG{l+s}{\PYGZdq{}\PYGZdq{}}\PYG{p}{;}
            \PYG{k+kt}{var} \PYG{n}{savePathZipPCSX2} \PYG{p}{=} \PYG{l+s}{\PYGZdq{}\PYGZdq{}}\PYG{p}{;}
            \PYG{k+kt}{var} \PYG{n}{savePath} \PYG{p}{=} \PYG{n}{Path}\PYG{p}{.}\PYG{n}{Combine}\PYG{p}{(}\PYG{n}{appDataPath}\PYG{p}{,} \PYG{l+s}{@\PYGZdq{}Caiman\PYGZbs{}users\PYGZbs{}\PYGZdq{}} \PYG{p}{+} \PYG{n}{emulatorsManager}\PYG{p}{.}\PYG{n}{user}\PYG{p}{.}\PYG{n}{username} \PYG{p}{+} \PYG{l+s}{@\PYGZdq{}\PYGZbs{}Save\PYGZbs{}\PYGZdq{}}\PYG{p}{)}\PYG{p}{;}

            \PYG{n}{savePathZipDolpin} \PYG{p}{=} \PYG{n}{Path}\PYG{p}{.}\PYG{n}{Combine}\PYG{p}{(}\PYG{n}{appDataPath}\PYG{p}{,} \PYG{l+s}{@\PYGZdq{}Caiman\PYGZbs{}users\PYGZbs{}\PYGZdq{}} \PYG{p}{+}  \PYG{n}{emulatorsManager}\PYG{p}{.}\PYG{n}{user}\PYG{p}{.}\PYG{n}{username}\PYG{p}{+} \PYG{l+s}{@\PYGZdq{}\PYGZbs{}Save\PYGZbs{}GamecubeWii\PYGZbs{}\PYGZdq{}}\PYG{p}{)}\PYG{p}{;}
            \PYG{n}{savePathZipPCSX2} \PYG{p}{=} \PYG{n}{Path}\PYG{p}{.}\PYG{n}{Combine}\PYG{p}{(}\PYG{n}{appDataPath}\PYG{p}{,} \PYG{l+s}{@\PYGZdq{}Caiman\PYGZbs{}users\PYGZbs{}\PYGZdq{}} \PYG{p}{+} \PYG{n}{emulatorsManager}\PYG{p}{.}\PYG{n}{user}\PYG{p}{.}\PYG{n}{username} \PYG{p}{+} \PYG{l+s}{@\PYGZdq{}\PYGZbs{}Save\PYGZbs{}Playstation2\PYGZbs{}\PYGZdq{}}\PYG{p}{)}\PYG{p}{;}
            \PYG{c+c1}{//zip the save of the emulators}
            \PYG{n}{ZipFile}\PYG{p}{.}\PYG{n}{CreateFromDirectory}\PYG{p}{(}\PYG{n}{savePathZipPCSX2}\PYG{p}{,} \PYG{n}{savePath} \PYG{p}{+} \PYG{l+s}{\PYGZdq{}tempPCSX2.zip\PYGZdq{}}\PYG{p}{)}\PYG{p}{;}
            \PYG{n}{ZipFile}\PYG{p}{.}\PYG{n}{CreateFromDirectory}\PYG{p}{(}\PYG{n}{savePathZipDolpin}\PYG{p}{,} \PYG{n}{savePath} \PYG{p}{+} \PYG{l+s}{\PYGZdq{}tempDolphin.zip\PYGZdq{}}\PYG{p}{)}\PYG{p}{;}
            \PYG{c+c1}{//call the api to create a copy of the save to the Bunker}
            \PYG{n}{callAPI}\PYG{p}{.}\PYG{n}{UploadSave}\PYG{p}{(}\PYG{l+m}{1}\PYG{p}{,} \PYG{n}{emulatorsManager}\PYG{p}{.}\PYG{n}{user}\PYG{p}{.}\PYG{n}{id}\PYG{p}{,} \PYG{n}{emulatorsManager}\PYG{p}{.}\PYG{n}{user}\PYG{p}{.}\PYG{n}{apitoken}\PYG{p}{,} \PYG{n}{savePath} \PYG{p}{+} \PYG{l+s}{\PYGZdq{}tempDolphin.zip\PYGZdq{}}\PYG{p}{)}\PYG{p}{;}
            \PYG{n}{callAPI}\PYG{p}{.}\PYG{n}{UploadSave}\PYG{p}{(}\PYG{l+m}{2}\PYG{p}{,}\PYG{n}{emulatorsManager}\PYG{p}{.}\PYG{n}{user}\PYG{p}{.}\PYG{n}{id}\PYG{p}{,}\PYG{n}{emulatorsManager}\PYG{p}{.}\PYG{n}{user}\PYG{p}{.}\PYG{n}{apitoken}\PYG{p}{,} \PYG{n}{savePath} \PYG{p}{+} \PYG{l+s}{\PYGZdq{}tempPCSX2.zip\PYGZdq{}}\PYG{p}{)}\PYG{p}{;}
            \PYG{c+c1}{//delete the temporary zip file}
            \PYG{n}{File}\PYG{p}{.}\PYG{n}{Delete}\PYG{p}{(}\PYG{n}{savePath} \PYG{p}{+} \PYG{l+s}{\PYGZdq{}tempPCSX2.zip\PYGZdq{}}\PYG{p}{)}\PYG{p}{;}
            \PYG{n}{File}\PYG{p}{.}\PYG{n}{Delete}\PYG{p}{(}\PYG{n}{savePath} \PYG{p}{+} \PYG{l+s}{\PYGZdq{}tempDolphin.zip\PYGZdq{}}\PYG{p}{)}\PYG{p}{;}
            
        \PYG{p}{\PYGZcb{}}
\end{sphinxVerbatim}


\subsubsection{Download des sauvegardes présente sur le serveur de Caiman}
\label{\detokenize{organique:download-des-sauvegardes-presente-sur-le-serveur-de-caiman}}
\sphinxAtStartPar
Le lancement du téléchargement des sauvegardes est lancé à la connexion de l’utilisateur. Un appel à l’API est fait pour les deux émulateurs:

\begin{sphinxVerbatim}[commandchars=\\\{\}]
 \PYG{k}{public} \PYG{k}{void} \PYG{n+nf}{CreateDownload}\PYG{p}{(}\PYG{k+kt}{int} \PYG{n}{idEmulator}\PYG{p}{,} \PYG{k+kt}{string} \PYG{n}{apiKey}\PYG{p}{)}
        \PYG{p}{\PYGZob{}}

            \PYG{k+kt}{var} \PYG{n}{appDataPath} \PYG{p}{=} \PYG{n}{Environment}\PYG{p}{.}\PYG{n}{GetFolderPath}\PYG{p}{(}\PYG{n}{Environment}\PYG{p}{.}\PYG{n}{SpecialFolder}\PYG{p}{.}\PYG{n}{ApplicationData}\PYG{p}{)}\PYG{p}{;}
            \PYG{k+kt}{var} \PYG{n}{savePath} \PYG{p}{=} \PYG{n}{Path}\PYG{p}{.}\PYG{n}{Combine}\PYG{p}{(}\PYG{n}{appDataPath}\PYG{p}{,} \PYG{l+s}{@\PYGZdq{}Caiman\PYGZbs{}users\PYGZbs{}\PYGZdq{}} \PYG{p}{+} \PYG{n}{user}\PYG{p}{.}\PYG{n}{username} \PYG{p}{+} \PYG{l+s}{@\PYGZdq{}\PYGZbs{}Save\PYGZbs{}\PYGZdq{}}\PYG{p}{)}\PYG{p}{;}


            \PYG{k}{if} \PYG{p}{(}\PYG{n}{lst\PYGZus{}download} \PYG{p}{=}\PYG{p}{=} \PYG{k}{null}\PYG{p}{)}
            \PYG{p}{\PYGZob{}}
                \PYG{n}{lst\PYGZus{}download} \PYG{p}{=} \PYG{k}{new} \PYG{n}{List}\PYG{p}{\PYGZlt{}}\PYG{n}{DownloadSave}\PYG{p}{\PYGZgt{}}\PYG{p}{(}\PYG{p}{)}\PYG{p}{;}
            \PYG{p}{\PYGZcb{}}
            \PYG{n}{lst\PYGZus{}download}\PYG{p}{.}\PYG{n}{Add}\PYG{p}{(}\PYG{k}{new} \PYG{n}{DownloadSave}\PYG{p}{(}\PYG{n}{savePath}\PYG{p}{,} \PYG{n}{idEmulator}\PYG{p}{,}\PYG{n}{user}\PYG{p}{.}\PYG{n}{id}\PYG{p}{,} \PYG{n}{apiKey}\PYG{p}{,}\PYG{n}{user}\PYG{p}{.}\PYG{n}{username}\PYG{p}{,}\PYG{k}{this}\PYG{p}{)}\PYG{p}{)}\PYG{p}{;}
        \PYG{p}{\PYGZcb{}}
\end{sphinxVerbatim}

\sphinxAtStartPar
Ces appels à l’API vont télécharger des fichiers .zip contenant les sauvegardes des différents émulateurs. Ces deux fichiers vont être décompressé dans le dossier spécifique de chaque émulateur. Avant de pouvoir décompresser le dossier, je dois au préalable supprimer le contenu des dossiers de destination (Si je dois supprimer le contenu des dossiers c’est parce que la classe Zipfile de C\# sous la version 5.0 ne peut pas “override” le contenu d’un dossier).

\sphinxAtStartPar
Quand les fichiers zip sont décompressés le contenu est envoyé dans les dossiers des différents émulateurs.


\section{Description technique: Interface graphique}
\label{\detokenize{organique:description-technique-interface-graphique}}
\sphinxAtStartPar
Diagramme de classe

\sphinxAtStartPar
\sphinxincludegraphics{{DiagramVue}.png}


\subsection{Organisation des boutons}
\label{\detokenize{organique:organisation-des-boutons}}
\sphinxAtStartPar
Pour faire en sorte de pouvoir se déplacer de bouton en bouton, il m’a fallu trouver une stratégie pour organiser les boutons. Pour structurer la liste des boutons, j’ai finalement décidé de créer une liste de liste de boutons. La liste principale sert de rangé et la “sous liste” sert de colonne. Cela permet de connaître plus aisément l’emplacement des boutons qu’une simple liste à une seule dimension.


\subsection{Gestion des inputs}
\label{\detokenize{organique:gestion-des-inputs}}

\subsubsection{Interaction avec les manettes}
\label{\detokenize{organique:interaction-avec-les-manettes}}
\sphinxAtStartPar
L’application Caiman prend en charge les manettes reconnues nativement par windows (Xinput). Pour pouvoir communiquer avec la manette, j’ai utilisé le paquet NuGet “SharpDX”, il me permet de connaître les informations et les inputs des manette connectés au pc.

\sphinxAtStartPar
Pour simplifier l’interaction avec la manette de l’utilisateur, j’ai créé une classe “XboxController.cs”, Cette classe me permet de connaître les manettes connectées et de savoir à un instant T les inputs de chaque manette.


\subsubsection{Transformation d’inputs en événement.}
\label{\detokenize{organique:transformation-dinputs-en-evenement}}
\sphinxAtStartPar
Pour simplifier la navigation, seule la manette une a le droit de se déplacer dans l’application de Caiman. Pour connaître les inputs, je les rafraîchis toutes les 10ms, cela permet de ne pas avoir de latence ou de louper des inputs.


\subsubsection{Gestion des boutons}
\label{\detokenize{organique:gestion-des-boutons}}
\sphinxAtStartPar
Quand la fonction ScanInput reçoit un string contenant les inputs de la manette 1. Ce string est envoyé dans un switch qui va faire une comparaison entre l’input précédent de la manette et l’input actuel. Si l’input actuel est différent de l’ancien alors celaveut dire que l’utilisateur a pressé un bouton puis il l’a soit relâché soit fait une autre combinaison de touches.

\sphinxAtStartPar
Le code suivant n’est pas complet!

\begin{sphinxVerbatim}[commandchars=\\\{\}]
\PYG{k}{public} \PYG{k}{void} \PYG{n+nf}{ScanInput}\PYG{p}{(}\PYG{k+kt}{object} \PYG{n}{sender}\PYG{p}{,} \PYG{n}{EventArgs} \PYG{n}{e}\PYG{p}{)}
        \PYG{p}{\PYGZob{}}

            \PYG{c+c1}{// if the user 1 controller is connected}
            \PYG{k}{if} \PYG{p}{(}\PYG{n}{xboxController}\PYG{p}{.}\PYG{n}{lstController}\PYG{p}{[}\PYG{l+m}{0}\PYG{p}{]}\PYG{p}{.}\PYG{n}{IsConnected}\PYG{p}{)}
            \PYG{p}{\PYGZob{}}
                \PYG{k+kt}{string} \PYG{n}{input} \PYG{p}{=} \PYG{n}{xboxController}\PYG{p}{.}\PYG{n}{lstController}\PYG{p}{[}\PYG{l+m}{0}\PYG{p}{]}\PYG{p}{.}\PYG{n}{GetState}\PYG{p}{(}\PYG{p}{)}\PYG{p}{.}\PYG{n}{Gamepad}\PYG{p}{.}\PYG{n}{Buttons}\PYG{p}{.}\PYG{n}{ToString}\PYG{p}{(}\PYG{p}{)}\PYG{p}{;}
                \PYG{k+kt}{int} \PYG{n}{inputAnalogLeftX} \PYG{p}{=} \PYG{n}{xboxController}\PYG{p}{.}\PYG{n}{lstController}\PYG{p}{[}\PYG{l+m}{0}\PYG{p}{]}\PYG{p}{.}\PYG{n}{GetState}\PYG{p}{(}\PYG{p}{)}\PYG{p}{.}\PYG{n}{Gamepad}\PYG{p}{.}\PYG{n}{LeftThumbX}\PYG{p}{;}
                \PYG{k+kt}{int} \PYG{n}{inputAnalogLeftY} \PYG{p}{=} \PYG{n}{xboxController}\PYG{p}{.}\PYG{n}{lstController}\PYG{p}{[}\PYG{l+m}{0}\PYG{p}{]}\PYG{p}{.}\PYG{n}{GetState}\PYG{p}{(}\PYG{p}{)}\PYG{p}{.}\PYG{n}{Gamepad}\PYG{p}{.}\PYG{n}{LeftThumbY}\PYG{p}{;}
\end{sphinxVerbatim}

\begin{sphinxVerbatim}[commandchars=\\\{\}]
\PYG{k}{if} \PYG{p}{(}\PYG{n}{input} \PYG{p}{=}\PYG{p}{=} \PYG{l+s}{\PYGZdq{}A\PYGZdq{}} \PYG{p}{\PYGZam{}}\PYG{p}{\PYGZam{}} \PYG{n}{old\PYGZus{}input} \PYG{p}{!}\PYG{p}{=} \PYG{l+s}{\PYGZdq{}A\PYGZdq{}}\PYG{p}{)}
                    \PYG{p}{\PYGZob{}}
                        \PYG{n}{SendKeys}\PYG{p}{.}\PYG{n}{Send}\PYG{p}{(}\PYG{l+s}{\PYGZdq{}\PYGZob{}ENTER\PYGZcb{}\PYGZdq{}}\PYG{p}{)}\PYG{p}{;}
                    \PYG{p}{\PYGZcb{}}
\end{sphinxVerbatim}


\subsubsection{Gestion du joystick gauche}
\label{\detokenize{organique:gestion-du-joystick-gauche}}
\sphinxAtStartPar
La réception des inputs pour le joystick n’est pas comme pour les boutons, en sachant que les joysticks sont analogiques, je reçois une valeur entre \sphinxhyphen{}32000 et +32000. Je vais recevoir ces valeurs pour l’axe X et l’axe Y. Pour savoir si l’utilisateur a poussé le joystick dans une direction, j’ai défini une “deadzone” dans laquelle je considère qu” aucune direction n’est définie par l’utilisateur de la manette. J’ai défini cette zone à 20000, j’ai fait des tests pour savoir où je trouvais que la “deadzone” devait s’arrêter. Puis j’ai fait comme pour les boutons,  j’ai fait une comparaison avec la valeur précédente pour savoir si le joystick vient d’être poussé.

\begin{sphinxVerbatim}[commandchars=\\\{\}]
                \PYG{k+kt}{bool} \PYG{n}{leftAnalogUp} \PYG{p}{=} \PYG{k}{false}\PYG{p}{;}
                \PYG{k+kt}{bool} \PYG{n}{leftAnalogDown} \PYG{p}{=} \PYG{k}{false}\PYG{p}{;}
                \PYG{k+kt}{bool} \PYG{n}{leftAnalogLeft} \PYG{p}{=} \PYG{k}{false}\PYG{p}{;}
                \PYG{k+kt}{bool} \PYG{n}{leftAnalogRight} \PYG{p}{=} \PYG{k}{false}\PYG{p}{;}

                \PYG{k}{if} \PYG{p}{(}\PYG{n}{inputAnalogLeftX} \PYG{p}{\PYGZgt{}} \PYG{n}{DEAD\PYGZus{}ZONE\PYGZus{}JOYSTICK}\PYG{p}{)}
                \PYG{p}{\PYGZob{}}
                    \PYG{n}{leftAnalogRight} \PYG{p}{=} \PYG{k}{true}\PYG{p}{;}
                \PYG{p}{\PYGZcb{}}

                \PYG{k}{if} \PYG{p}{(}\PYG{n}{inputAnalogLeftX} \PYG{p}{\PYGZlt{}} \PYG{p}{\PYGZhy{}}\PYG{n}{DEAD\PYGZus{}ZONE\PYGZus{}JOYSTICK}\PYG{p}{)}
                \PYG{p}{\PYGZob{}}
                    \PYG{n}{leftAnalogLeft} \PYG{p}{=} \PYG{k}{true}\PYG{p}{;}
                \PYG{p}{\PYGZcb{}}

                \PYG{k}{if} \PYG{p}{(}\PYG{n}{inputAnalogLeftY} \PYG{p}{\PYGZgt{}} \PYG{n}{DEAD\PYGZus{}ZONE\PYGZus{}JOYSTICK}\PYG{p}{)}
                \PYG{p}{\PYGZob{}}
                    \PYG{n}{leftAnalogUp} \PYG{p}{=} \PYG{k}{true}\PYG{p}{;}
                \PYG{p}{\PYGZcb{}}

                \PYG{k}{if} \PYG{p}{(}\PYG{n}{inputAnalogLeftY} \PYG{p}{\PYGZlt{}} \PYG{p}{\PYGZhy{}}\PYG{n}{DEAD\PYGZus{}ZONE\PYGZus{}JOYSTICK}\PYG{p}{)}
                \PYG{p}{\PYGZob{}}
                    \PYG{n}{leftAnalogDown} \PYG{p}{=} \PYG{k}{true}\PYG{p}{;}
                \PYG{p}{\PYGZcb{}}
\end{sphinxVerbatim}


\subsection{Découpage de l’interface}
\label{\detokenize{organique:decoupage-de-linterface}}
\sphinxAtStartPar
L’interface de l’application Caiman est divisée en trois parties hormis pour la connexion.


\subsubsection{Navbar}
\label{\detokenize{organique:navbar}}
\sphinxAtStartPar
La première est la navbar qui contient les différents boutons de navigation et les informations sur le jeu en cours. Tous les boutons sont créés dynamiquement, le placement s’adapte à l’écran principal de l’utilisateur, mais s’il a un écran vraiment trop petit, les informations sur le jeu en cours et les boutons risquent de s’entremêler.

\sphinxAtStartPar
Pour afficher le jeu en cours, le navigateur est obligé de se rafraîchir et de se redessiner toutes les secondes pour afficher le temps de jeu. Pour savoir si un jeu est en cours,lal’a navbar regarde si le GameTimer est lancé. Si le GameTimer est lancé c’est qu’un jeu est en cours, dans ce cas il va afficher le nom et le temps de jeu.

\begin{sphinxVerbatim}[commandchars=\\\{\}]
\PYG{k}{private} \PYG{k}{void} \PYG{n+nf}{UpdateData}\PYG{p}{(}\PYG{k+kt}{object} \PYG{n}{sender}\PYG{p}{,} \PYG{n}{EventArgs} \PYG{n}{e}\PYG{p}{)}
        \PYG{p}{\PYGZob{}}
            \PYG{k}{if} \PYG{p}{(}\PYG{n}{xboxMainForm}\PYG{p}{.}\PYG{n}{emulatorsManager}\PYG{p}{.}\PYG{n}{gameTimer} \PYG{p}{!}\PYG{p}{=} \PYG{k}{null}\PYG{p}{)}
            \PYG{p}{\PYGZob{}}
                \PYG{k}{if} \PYG{p}{(}\PYG{n}{actualGameName} \PYG{p}{!}\PYG{p}{=} \PYG{l+s}{\PYGZdq{}\PYGZdq{}}\PYG{p}{)}
                \PYG{p}{\PYGZob{}}
                    \PYG{n}{lbl\PYGZus{}game\PYGZus{}actual}\PYG{p}{.}\PYG{n}{Text} \PYG{p}{=} \PYG{l+s}{\PYGZdq{} Now playing: \PYGZdq{}} \PYG{p}{+} \PYG{n}{actualGameName} \PYG{p}{+} \PYG{l+s}{\PYGZdq{}  \PYGZdq{}} \PYG{p}{+} \PYG{n}{xboxMainForm}\PYG{p}{.}\PYG{n}{emulatorsManager}\PYG{p}{.}\PYG{n}{gameTimer}\PYG{p}{.}\PYG{n}{ToString}\PYG{p}{(}\PYG{p}{)}\PYG{p}{;}
                \PYG{p}{\PYGZcb{}}
                \PYG{k}{else}
                \PYG{p}{\PYGZob{}}
                    \PYG{n}{lbl\PYGZus{}game\PYGZus{}actual}\PYG{p}{.}\PYG{n}{Text} \PYG{p}{=} \PYG{l+s}{\PYGZdq{}\PYGZdq{}}\PYG{p}{;}
                \PYG{p}{\PYGZcb{}}
            \PYG{p}{\PYGZcb{}}
            \PYG{k}{else}
            \PYG{p}{\PYGZob{}}
                \PYG{n}{lbl\PYGZus{}game\PYGZus{}actual}\PYG{p}{.}\PYG{n}{Text} \PYG{p}{=} \PYG{l+s}{\PYGZdq{}\PYGZdq{}}\PYG{p}{;}
            \PYG{p}{\PYGZcb{}}
        \PYG{p}{\PYGZcb{}}
\end{sphinxVerbatim}


\subsubsection{Sidebar}
\label{\detokenize{organique:sidebar}}
\sphinxAtStartPar
La sidebar contient des boutons qui permettent d’afficher différentes listes de jeux. Ces boutons sont en partie créés dynamiquement et en partie reçus de l’API. Les premiers boutons sont ceux qui concernent l’utilisateur, on va les avoir dans l’ordre:
\begin{itemize}
\item {} 
\sphinxAtStartPar
Downloaded games

\item {} 
\sphinxAtStartPar
Favorites games

\item {} 
\sphinxAtStartPar
All games

\end{itemize}

\sphinxAtStartPar
Ces catégories ne sont pas créées dynamiquement.

\sphinxAtStartPar
Par contre, les suivants sont reçus de l’API, ils concernent les différentes catégories disponibles pour Caiman. À l’appui de ces boutons, la liste des jeux va être chargée et le main panel va afficher la liste des jeux reçus par l’API.

\begin{sphinxVerbatim}[commandchars=\\\{\}]
\PYG{k}{public} \PYG{k}{void} \PYG{n+nf}{CreateListNavButton}\PYG{p}{(}\PYG{p}{)}
        \PYG{p}{\PYGZob{}}
            \PYG{n}{List}\PYG{p}{\PYGZlt{}}\PYG{k+kt}{string}\PYG{p}{\PYGZgt{}} \PYG{n}{lst\PYGZus{}navbar} \PYG{p}{=} \PYG{k}{new} \PYG{n}{List}\PYG{p}{\PYGZlt{}}\PYG{k+kt}{string}\PYG{p}{\PYGZgt{}}\PYG{p}{(}\PYG{p}{)}\PYG{p}{;}
            \PYG{n}{List}\PYG{p}{\PYGZlt{}}\PYG{n}{Category}\PYG{p}{\PYGZgt{}} \PYG{n}{lst\PYGZus{}category} \PYG{p}{=} \PYG{n}{xboxMainForm}\PYG{p}{.}\PYG{n}{callAPI}\PYG{p}{.}\PYG{n}{CallAllCategories}\PYG{p}{(}\PYG{p}{)}\PYG{p}{;}
            \PYG{n}{lst\PYGZus{}navbar}\PYG{p}{.}\PYG{n}{Add}\PYG{p}{(}\PYG{l+s}{\PYGZdq{}Downloaded games\PYGZdq{}}\PYG{p}{)}\PYG{p}{;}
            \PYG{n}{lst\PYGZus{}navbar}\PYG{p}{.}\PYG{n}{Add}\PYG{p}{(}\PYG{l+s}{\PYGZdq{}Favorites games\PYGZdq{}}\PYG{p}{)}\PYG{p}{;}
            \PYG{n}{lst\PYGZus{}navbar}\PYG{p}{.}\PYG{n}{Add}\PYG{p}{(}\PYG{l+s}{\PYGZdq{}All games\PYGZdq{}}\PYG{p}{)}\PYG{p}{;}

            \PYG{k}{foreach} \PYG{p}{(}\PYG{k+kt}{var} \PYG{n}{item} \PYG{k}{in} \PYG{n}{lst\PYGZus{}category}\PYG{p}{)}
            \PYG{p}{\PYGZob{}}
                \PYG{n}{lst\PYGZus{}navbar}\PYG{p}{.}\PYG{n}{Add}\PYG{p}{(}\PYG{n}{item}\PYG{p}{.}\PYG{n}{name}\PYG{p}{)}\PYG{p}{;}
            \PYG{p}{\PYGZcb{}}
            \PYG{k}{for} \PYG{p}{(}\PYG{k+kt}{int} \PYG{n}{i} \PYG{p}{=} \PYG{l+m}{0}\PYG{p}{;} \PYG{n}{i} \PYG{p}{\PYGZlt{}} \PYG{n}{lst\PYGZus{}navbar}\PYG{p}{.}\PYG{n}{Count}\PYG{p}{;} \PYG{n}{i}\PYG{p}{+}\PYG{p}{+}\PYG{p}{)}
            \PYG{p}{\PYGZob{}}
                \PYG{n}{lstControls}\PYG{p}{.}\PYG{n}{Add}\PYG{p}{(}\PYG{k}{new} \PYG{n}{List}\PYG{p}{\PYGZlt{}}\PYG{n}{Control}\PYG{p}{\PYGZgt{}}\PYG{p}{(}\PYG{p}{)}\PYG{p}{)}\PYG{p}{;}
                \PYG{n}{lstControls}\PYG{p}{[}\PYG{n}{i}\PYG{p}{]}\PYG{p}{.}\PYG{n}{Add}\PYG{p}{(}\PYG{k}{new} \PYG{n}{XboxButton}\PYG{p}{(}\PYG{p}{)}\PYG{p}{)}\PYG{p}{;}
            \PYG{p}{\PYGZcb{}}

            \PYG{c+c1}{//update the buttons infos}
            \PYG{k}{for} \PYG{p}{(}\PYG{k+kt}{int} \PYG{n}{a\PYGZus{}row} \PYG{p}{=} \PYG{l+m}{0}\PYG{p}{;} \PYG{n}{a\PYGZus{}row} \PYG{p}{\PYGZlt{}}\PYG{p}{=} \PYG{p}{(}\PYG{n}{lst\PYGZus{}navbar}\PYG{p}{.}\PYG{n}{Count} \PYG{p}{\PYGZhy{}}\PYG{l+m}{1}\PYG{p}{)}\PYG{p}{;} \PYG{n}{a\PYGZus{}row}\PYG{p}{+}\PYG{p}{+}\PYG{p}{)}
            \PYG{p}{\PYGZob{}}

                \PYG{n}{List}\PYG{p}{\PYGZlt{}}\PYG{k+kt}{string}\PYG{p}{\PYGZgt{}} \PYG{n}{lstString} \PYG{p}{=} \PYG{k}{new} \PYG{n}{List}\PYG{p}{\PYGZlt{}}\PYG{k+kt}{string}\PYG{p}{\PYGZgt{}}\PYG{p}{(}\PYG{p}{)}\PYG{p}{;}
                \PYG{n}{XboxButton} \PYG{n}{tempButton} \PYG{p}{=} \PYG{k}{new} \PYG{n}{XboxButton}\PYG{p}{(}\PYG{l+s}{\PYGZdq{}side\PYGZdq{}}\PYG{p}{,} \PYG{n}{a\PYGZus{}row}\PYG{p}{,} \PYG{n}{a\PYGZus{}row}\PYG{p}{,} \PYG{l+m}{0}\PYG{p}{)}\PYG{p}{;}
                \PYG{n}{lstControls}\PYG{p}{[}\PYG{n}{a\PYGZus{}row}\PYG{p}{]}\PYG{p}{[}\PYG{l+m}{0}\PYG{p}{]} \PYG{p}{=} \PYG{n}{tempButton}\PYG{p}{;}
                \PYG{n}{lstControls}\PYG{p}{[}\PYG{n}{a\PYGZus{}row}\PYG{p}{]}\PYG{p}{[}\PYG{l+m}{0}\PYG{p}{]}\PYG{p}{.}\PYG{n}{Text} \PYG{p}{=} \PYG{n}{lst\PYGZus{}navbar}\PYG{p}{[}\PYG{n}{a\PYGZus{}row}\PYG{p}{]}\PYG{p}{;}
                \PYG{n}{lstControls}\PYG{p}{[}\PYG{n}{a\PYGZus{}row}\PYG{p}{]}\PYG{p}{[}\PYG{l+m}{0}\PYG{p}{]}\PYG{p}{.}\PYG{n}{Location} \PYG{p}{=} \PYG{k}{new} \PYG{n}{System}\PYG{p}{.}\PYG{n}{Drawing}\PYG{p}{.}\PYG{n}{Point}\PYG{p}{(}\PYG{l+m}{0} \PYG{p}{*} \PYG{l+m}{1}\PYG{l+m}{0}\PYG{l+m}{0} \PYG{p}{+} \PYG{l+m}{2}\PYG{l+m}{0}\PYG{p}{,} \PYG{n}{a\PYGZus{}row} \PYG{p}{*} \PYG{l+m}{7}\PYG{l+m}{5} \PYG{p}{+} \PYG{l+m}{1}\PYG{l+m}{5}\PYG{p}{)}\PYG{p}{;}
                \PYG{n}{lstControls}\PYG{p}{[}\PYG{n}{a\PYGZus{}row}\PYG{p}{]}\PYG{p}{[}\PYG{l+m}{0}\PYG{p}{]}\PYG{p}{.}\PYG{n}{Width} \PYG{p}{=} \PYG{l+m}{2}\PYG{l+m}{0}\PYG{l+m}{0}\PYG{p}{;}
                \PYG{n}{lstControls}\PYG{p}{[}\PYG{n}{a\PYGZus{}row}\PYG{p}{]}\PYG{p}{[}\PYG{l+m}{0}\PYG{p}{]}\PYG{p}{.}\PYG{n}{Height} \PYG{p}{=} \PYG{l+m}{5}\PYG{l+m}{0}\PYG{p}{;}
                \PYG{n}{lstControls}\PYG{p}{[}\PYG{n}{a\PYGZus{}row}\PYG{p}{]}\PYG{p}{[}\PYG{l+m}{0}\PYG{p}{]}\PYG{p}{.}\PYG{n}{Name} \PYG{p}{=}  \PYG{l+s}{\PYGZdq{}btn\PYGZus{}\PYGZdq{}}\PYG{p}{+} \PYG{n}{lst\PYGZus{}navbar}\PYG{p}{[}\PYG{n}{a\PYGZus{}row}\PYG{p}{]}\PYG{p}{;}


                \PYG{n}{Controls}\PYG{p}{.}\PYG{n}{Add}\PYG{p}{(}\PYG{n}{lstControls}\PYG{p}{[}\PYG{n}{a\PYGZus{}row}\PYG{p}{]}\PYG{p}{[}\PYG{l+m}{0}\PYG{p}{]}\PYG{p}{)}\PYG{p}{;}
                \PYG{n}{lstControls}\PYG{p}{[}\PYG{n}{a\PYGZus{}row}\PYG{p}{]}\PYG{p}{[}\PYG{l+m}{0}\PYG{p}{]}\PYG{p}{.}\PYG{n}{Click} \PYG{p}{+}\PYG{p}{=} \PYG{k}{new} \PYG{n}{System}\PYG{p}{.}\PYG{n}{EventHandler}\PYG{p}{(}\PYG{n}{bouton\PYGZus{}Click}\PYG{p}{)}\PYG{p}{;}

            \PYG{p}{\PYGZcb{}}

            \PYG{c+c1}{//set the action of button}
            \PYG{n}{XboxButton} \PYG{n}{downloadedGames} \PYG{p}{=} \PYG{p}{(}\PYG{n}{XboxButton}\PYG{p}{)}\PYG{n}{lstControls}\PYG{p}{[}\PYG{l+m}{0}\PYG{p}{]}\PYG{p}{[}\PYG{l+m}{0}\PYG{p}{]}\PYG{p}{;}
            \PYG{n}{downloadedGames}\PYG{p}{.}\PYG{n}{contextInfos}\PYG{p}{.}\PYG{n}{contexte} \PYG{p}{=} \PYG{l+s}{\PYGZdq{}downloadedGames\PYGZdq{}}\PYG{p}{;}
            \PYG{n}{XboxButton} \PYG{n}{userFavoritesGames} \PYG{p}{=} \PYG{p}{(}\PYG{n}{XboxButton}\PYG{p}{)}\PYG{n}{lstControls}\PYG{p}{[}\PYG{l+m}{1}\PYG{p}{]}\PYG{p}{[}\PYG{l+m}{0}\PYG{p}{]}\PYG{p}{;}
            \PYG{n}{userFavoritesGames}\PYG{p}{.}\PYG{n}{contextInfos}\PYG{p}{.}\PYG{n}{contexte} \PYG{p}{=} \PYG{l+s}{\PYGZdq{}favorite\PYGZdq{}}\PYG{p}{;}
            \PYG{n}{XboxButton} \PYG{n}{allGames} \PYG{p}{=} \PYG{p}{(}\PYG{n}{XboxButton}\PYG{p}{)}\PYG{n}{lstControls}\PYG{p}{[}\PYG{l+m}{2}\PYG{p}{]}\PYG{p}{[}\PYG{l+m}{0}\PYG{p}{]}\PYG{p}{;}
            \PYG{n}{allGames}\PYG{p}{.}\PYG{n}{contextInfos}\PYG{p}{.}\PYG{n}{contexte} \PYG{p}{=} \PYG{l+s}{\PYGZdq{}home\PYGZdq{}}\PYG{p}{;}

            \PYG{k}{for} \PYG{p}{(}\PYG{k+kt}{int} \PYG{n}{i} \PYG{p}{=} \PYG{l+m}{0}\PYG{p}{;} \PYG{n}{i} \PYG{p}{\PYGZlt{}} \PYG{n}{lst\PYGZus{}category}\PYG{p}{.}\PYG{n}{Count}\PYG{p}{;} \PYG{n}{i}\PYG{p}{+}\PYG{p}{+}\PYG{p}{)}
            \PYG{p}{\PYGZob{}}
                \PYG{n}{XboxButton} \PYG{n}{tempButton} \PYG{p}{=} \PYG{p}{(}\PYG{n}{XboxButton}\PYG{p}{)}\PYG{n}{lstControls}\PYG{p}{[}\PYG{p}{(}\PYG{n}{i}\PYG{p}{+}\PYG{l+m}{3}\PYG{p}{)}\PYG{p}{]}\PYG{p}{[}\PYG{l+m}{0}\PYG{p}{]}\PYG{p}{;}
                \PYG{n}{tempButton}\PYG{p}{.}\PYG{n}{contextInfos}\PYG{p}{.}\PYG{n}{contexte} \PYG{p}{=} \PYG{l+s}{\PYGZdq{}category\PYGZdq{}}\PYG{p}{;}
                \PYG{n}{tempButton}\PYG{p}{.}\PYG{n}{contextInfos}\PYG{p}{.}\PYG{n}{id\PYGZus{}contexte} \PYG{p}{=} \PYG{p}{(}\PYG{n}{lst\PYGZus{}category}\PYG{p}{[}\PYG{n}{i}\PYG{p}{]}\PYG{p}{.}\PYG{n}{id}\PYG{p}{)}\PYG{p}{;}
            \PYG{p}{\PYGZcb{}}

        \PYG{p}{\PYGZcb{}}


    \PYG{p}{\PYGZcb{}}
\end{sphinxVerbatim}


\subsubsection{Main panel}
\label{\detokenize{organique:main-panel}}
\sphinxAtStartPar
Néanmoins,le main panel ne peut pas vraiment être décrit, c’est un panel contextuel qui va être défini par les besoins de l’utilisateur.


\subsection{Affichage d’une liste de jeu}
\label{\detokenize{organique:affichage-dune-liste-de-jeu}}
\sphinxAtStartPar
Pour l’affichage des listes des jeux, j’ai essayé de faire en sorte que le panel soit le plus flexible possible. le panel demande juste une liste de jeux, il n’y a pas de panel fait pour les jeux favoris ou les jeux déjà présents sur le disque.

\sphinxAtStartPar
Le panel utilise la fonction CreateListGames() pour dessiner les différents boutons de lien vers les détails des jeux. Pour que certaines images ne dépasse pas l’écran et qu’elles soient coupées, la fonction calcul le nombre d’images qu’elle peut afficher par ligne. Le calcul se fait selon la définition de l’écran de l’utilisateur. Les images affichées sont celles qui sont dans le dossier appdata de Caiman.

\begin{sphinxVerbatim}[commandchars=\\\{\}]
\PYG{k}{public} \PYG{k}{void} \PYG{n+nf}{CreateListGames}\PYG{p}{(}\PYG{p}{)}
        \PYG{p}{\PYGZob{}}

            \PYG{k+kt}{string} \PYG{n}{imgPath} \PYG{p}{=} \PYG{n}{Path}\PYG{p}{.}\PYG{n}{Combine}\PYG{p}{(}\PYG{n}{Environment}\PYG{p}{.}\PYG{n}{GetFolderPath}\PYG{p}{(}\PYG{n}{Environment}\PYG{p}{.}\PYG{n}{SpecialFolder}\PYG{p}{.}\PYG{n}{ApplicationData}\PYG{p}{)}\PYG{p}{,} \PYG{n}{PATH\PYGZus{}IMG\PYGZus{}CAIMAN}\PYG{p}{)}\PYG{p}{;}
            \PYG{n}{XboxImage} \PYG{n}{tempXboxImage} \PYG{p}{=} \PYG{k}{new} \PYG{n}{XboxImage}\PYG{p}{(}\PYG{p}{)}\PYG{p}{;}
            \PYG{k+kt}{int} \PYG{n}{max\PYGZus{}rank} \PYG{p}{=} \PYG{n}{Width} \PYG{p}{/} \PYG{p}{(}\PYG{l+m}{3}\PYG{l+m}{1}\PYG{l+m}{5}\PYG{p}{)}\PYG{p}{;}
            \PYG{k+kt}{int} \PYG{n}{tempPos\PYGZus{}x} \PYG{p}{=} \PYG{l+m}{0}\PYG{p}{;}
            \PYG{k+kt}{int} \PYG{n}{tempPos\PYGZus{}y} \PYG{p}{=} \PYG{l+m}{0}\PYG{p}{;}
            \PYG{n}{lstControls}\PYG{p}{.}\PYG{n}{Add}\PYG{p}{(}\PYG{k}{new} \PYG{n}{List}\PYG{p}{\PYGZlt{}}\PYG{n}{Control}\PYG{p}{\PYGZgt{}}\PYG{p}{(}\PYG{p}{)}\PYG{p}{)}\PYG{p}{;}


            \PYG{k}{foreach} \PYG{p}{(}\PYG{k+kt}{var} \PYG{n}{game} \PYG{k}{in} \PYG{n}{lst\PYGZus{}games}\PYG{p}{)}
            \PYG{p}{\PYGZob{}}
                \PYG{k}{if} \PYG{p}{(}\PYG{n}{tempPos\PYGZus{}x} \PYG{p}{=}\PYG{p}{=} \PYG{n}{max\PYGZus{}rank}\PYG{p}{)}
                \PYG{p}{\PYGZob{}}
                    \PYG{n}{lstControls}\PYG{p}{.}\PYG{n}{Add}\PYG{p}{(}\PYG{k}{new} \PYG{n}{List}\PYG{p}{\PYGZlt{}}\PYG{n}{Control}\PYG{p}{\PYGZgt{}}\PYG{p}{(}\PYG{p}{)}\PYG{p}{)}\PYG{p}{;}
                    \PYG{n}{tempPos\PYGZus{}y}\PYG{p}{+}\PYG{p}{+}\PYG{p}{;}
                    \PYG{n}{tempPos\PYGZus{}x} \PYG{p}{=} \PYG{l+m}{0}\PYG{p}{;}
                \PYG{p}{\PYGZcb{}}

                \PYG{n}{lstControls}\PYG{p}{[}\PYG{n}{tempPos\PYGZus{}y}\PYG{p}{]}\PYG{p}{.}\PYG{n}{Add}\PYG{p}{(}\PYG{k}{new} \PYG{n}{XboxImage}\PYG{p}{(}\PYG{p}{)}\PYG{p}{)}\PYG{p}{;}
                \PYG{n}{Image} \PYG{n}{img} \PYG{p}{=} \PYG{k}{new} \PYG{n}{Bitmap}\PYG{p}{(}\PYG{p}{(}\PYG{n}{imgPath}\PYG{p}{+} \PYG{n}{game}\PYG{p}{.}\PYG{n}{imageName}\PYG{p}{)}\PYG{p}{)}\PYG{p}{;}
                \PYG{n}{XboxImage} \PYG{n}{tempButton} \PYG{p}{=} \PYG{k}{new} \PYG{n}{XboxImage}\PYG{p}{(}\PYG{l+s}{\PYGZdq{}game\PYGZdq{}}\PYG{p}{,} \PYG{n}{img}\PYG{p}{,} \PYG{n}{game}\PYG{p}{.}\PYG{n}{id}\PYG{p}{,} \PYG{n}{tempPos\PYGZus{}x}\PYG{p}{,} \PYG{n}{tempPos\PYGZus{}y}\PYG{p}{)}\PYG{p}{;}
                \PYG{n}{lstControls}\PYG{p}{[}\PYG{n}{tempPos\PYGZus{}y}\PYG{p}{]}\PYG{p}{[}\PYG{n}{tempPos\PYGZus{}x}\PYG{p}{]} \PYG{p}{=} \PYG{n}{tempButton}\PYG{p}{;}
                \PYG{n}{lstControls}\PYG{p}{[}\PYG{n}{tempPos\PYGZus{}y}\PYG{p}{]}\PYG{p}{[}\PYG{n}{tempPos\PYGZus{}x}\PYG{p}{]}\PYG{p}{.}\PYG{n}{Location} \PYG{p}{=} \PYG{k}{new} \PYG{n}{System}\PYG{p}{.}\PYG{n}{Drawing}\PYG{p}{.}\PYG{n}{Point}\PYG{p}{(}\PYG{n}{tempPos\PYGZus{}x} \PYG{p}{*} \PYG{l+m}{3}\PYG{l+m}{0}\PYG{l+m}{0} \PYG{p}{+} \PYG{l+m}{1}\PYG{l+m}{5}\PYG{p}{,} \PYG{n}{tempPos\PYGZus{}y} \PYG{p}{*} \PYG{l+m}{4}\PYG{l+m}{3}\PYG{l+m}{0} \PYG{p}{+} \PYG{l+m}{1}\PYG{l+m}{5}\PYG{p}{)}\PYG{p}{;}

                \PYG{n}{Controls}\PYG{p}{.}\PYG{n}{Add}\PYG{p}{(}\PYG{n}{lstControls}\PYG{p}{[}\PYG{n}{tempPos\PYGZus{}y}\PYG{p}{]}\PYG{p}{[}\PYG{n}{tempPos\PYGZus{}x}\PYG{p}{]}\PYG{p}{)}\PYG{p}{;}
                \PYG{n}{tempButton}\PYG{p}{.}\PYG{n}{Click} \PYG{p}{+}\PYG{p}{=} \PYG{k}{new} \PYG{n}{System}\PYG{p}{.}\PYG{n}{EventHandler}\PYG{p}{(}\PYG{n}{bouton\PYGZus{}Click}\PYG{p}{)}\PYG{p}{;}



                \PYG{n}{tempPos\PYGZus{}x}\PYG{p}{+}\PYG{p}{+}\PYG{p}{;}
            \PYG{p}{\PYGZcb{}}
        \PYG{p}{\PYGZcb{}}
\end{sphinxVerbatim}


\subsection{Gestion des erreurs de déplacement}
\label{\detokenize{organique:gestion-des-erreurs-de-deplacement}}
\sphinxAtStartPar
Le déplacement à la manette se fait par déplacement haut,bas,gauche,droite. Donc, J’ai décidé de faire en sorte que si l’utilisateur se déplace dans une case qui n’existe pas, ils soient soit déplacés dans l’une des casses disponibles proche de la case demandée, autrement  rien ne se passe.

\sphinxAtStartPar
Pour cela, j’ai dû décider arbitrairement ce qui se passait. si un mouvement illégal était fait. J’ai essayé de reproduire les déplacements utilisés dans l’interface de la console Xbox. Il y a certains cas qui ne sont pas forcément bien gérés. Il serait possible de faire un déplacement en calculant la position actuelle et en se basant sur la position en pixel. Cette alternative pourrait être intéressante.

\begin{sphinxVerbatim}[commandchars=\\\{\}]
\PYG{k}{public} \PYG{k}{void} \PYG{n+nf}{MoveActivateControl}\PYG{p}{(}\PYG{k+kt}{string} \PYG{n}{destination} \PYG{p}{=} \PYG{l+s}{\PYGZdq{}\PYGZdq{}}\PYG{p}{)}
        \PYG{p}{\PYGZob{}}
            \PYG{c+c1}{//top = 1}
            \PYG{c+c1}{//right = 2}
            \PYG{c+c1}{//down = 3}
            \PYG{c+c1}{//left = 4}
            \PYG{k}{if} \PYG{p}{(}\PYG{n}{destination} \PYG{p}{=}\PYG{p}{=} \PYG{l+s}{\PYGZdq{}down\PYGZdq{}}\PYG{p}{)}
            \PYG{p}{\PYGZob{}}
                \PYG{k}{if} \PYG{p}{(}\PYG{n}{lstControls}\PYG{p}{[}\PYG{n}{position\PYGZus{}y}\PYG{p}{]}\PYG{p}{[}\PYG{n}{position\PYGZus{}x}\PYG{p}{]} \PYG{p}{=}\PYG{p}{=} \PYG{k}{null}\PYG{p}{)}
                \PYG{p}{\PYGZob{}}
                    \PYG{k+kt}{int} \PYG{n}{x} \PYG{p}{=} \PYG{n}{position\PYGZus{}x}\PYG{p}{;}
                    \PYG{k+kt}{int} \PYG{n}{y} \PYG{p}{=} \PYG{n}{position\PYGZus{}y}\PYG{p}{;}

                    \PYG{k+kt}{int} \PYG{n}{y\PYGZus{}right\PYGZus{}not\PYGZus{}disponible}\PYG{p}{;}
                    \PYG{k}{while} \PYG{p}{(}\PYG{n}{x} \PYG{p}{\PYGZlt{}} \PYG{n}{lstControls}\PYG{p}{[}\PYG{n}{position\PYGZus{}y}\PYG{p}{]}\PYG{p}{.}\PYG{n}{Count}\PYG{p}{(}\PYG{p}{)} \PYG{p}{\PYGZam{}}\PYG{p}{\PYGZam{}} \PYG{n}{lstControls}\PYG{p}{[}\PYG{n}{position\PYGZus{}y}\PYG{p}{]}\PYG{p}{[}\PYG{n}{x}\PYG{p}{]} \PYG{p}{=}\PYG{p}{=} \PYG{k}{null}\PYG{p}{)}
                    \PYG{p}{\PYGZob{}}
                        \PYG{k}{if} \PYG{p}{(}\PYG{n}{x} \PYG{p}{\PYGZgt{}} \PYG{n}{lstControls}\PYG{p}{[}\PYG{n}{position\PYGZus{}x}\PYG{p}{\PYGZhy{}}\PYG{l+m}{1}\PYG{p}{]}\PYG{p}{.}\PYG{n}{Count}\PYG{p}{(}\PYG{p}{)}\PYG{p}{)}
                        \PYG{p}{\PYGZob{}}
                            \PYG{k}{break}\PYG{p}{;}
                        \PYG{p}{\PYGZcb{}}
                        \PYG{n}{x}\PYG{p}{+}\PYG{p}{+}\PYG{p}{;}
                    \PYG{p}{\PYGZcb{}}
                    \PYG{k}{if} \PYG{p}{(}\PYG{n}{x} \PYG{p}{=}\PYG{p}{=} \PYG{n}{lstControls}\PYG{p}{[}\PYG{n}{position\PYGZus{}y}\PYG{p}{]}\PYG{p}{.}\PYG{n}{Count}\PYG{p}{(}\PYG{p}{)}\PYG{p}{)}
                    \PYG{p}{\PYGZob{}}
                        \PYG{n}{y\PYGZus{}right\PYGZus{}not\PYGZus{}disponible} \PYG{p}{=} \PYG{n}{y}\PYG{p}{;}

                        \PYG{k}{while} \PYG{p}{(}\PYG{n}{y\PYGZus{}right\PYGZus{}not\PYGZus{}disponible}\PYG{p}{\PYGZgt{}}\PYG{l+m}{0}\PYG{p}{)}
                        \PYG{p}{\PYGZob{}}
                            \PYG{n}{y\PYGZus{}right\PYGZus{}not\PYGZus{}disponible} \PYG{p}{\PYGZhy{}}\PYG{p}{\PYGZhy{}}\PYG{p}{;}
                            \PYG{k}{if} \PYG{p}{(}\PYG{n}{lstControls}\PYG{p}{[}\PYG{n}{y\PYGZus{}right\PYGZus{}not\PYGZus{}disponible}\PYG{p}{]}\PYG{p}{[}\PYG{n}{position\PYGZus{}x}\PYG{p}{]} \PYG{p}{!}\PYG{p}{=} \PYG{k}{null}\PYG{p}{)}
                            \PYG{p}{\PYGZob{}}
                                \PYG{n}{position\PYGZus{}y} \PYG{p}{=} \PYG{n}{y\PYGZus{}right\PYGZus{}not\PYGZus{}disponible}\PYG{p}{;}
                                \PYG{n}{lstControls}\PYG{p}{[}\PYG{p}{(}\PYG{n}{position\PYGZus{}y}\PYG{p}{)}\PYG{p}{]}\PYG{p}{[}\PYG{n}{position\PYGZus{}x}\PYG{p}{]}\PYG{p}{.}\PYG{n}{Focus}\PYG{p}{(}\PYG{p}{)}\PYG{p}{;}
                                \PYG{k}{return}\PYG{p}{;}
                            \PYG{p}{\PYGZcb{}}
                        \PYG{p}{\PYGZcb{}}
                    \PYG{p}{\PYGZcb{}}
                    \PYG{n}{lstControls}\PYG{p}{[}\PYG{p}{(}\PYG{n}{position\PYGZus{}y}\PYG{p}{)}\PYG{p}{]}\PYG{p}{[}\PYG{n}{x}\PYG{p}{]}\PYG{p}{.}\PYG{n}{Focus}\PYG{p}{(}\PYG{p}{)}\PYG{p}{;}
                    \PYG{n}{Position\PYGZus{}x} \PYG{p}{=} \PYG{n}{x}\PYG{p}{;}
                \PYG{p}{\PYGZcb{}}
                \PYG{k}{else}
                \PYG{p}{\PYGZob{}}
                    \PYG{n}{lstControls}\PYG{p}{[}\PYG{p}{(}\PYG{n}{position\PYGZus{}y}\PYG{p}{)}\PYG{p}{]}\PYG{p}{[}\PYG{n}{position\PYGZus{}x}\PYG{p}{]}\PYG{p}{.}\PYG{n}{Focus}\PYG{p}{(}\PYG{p}{)}\PYG{p}{;}
                \PYG{p}{\PYGZcb{}}
            \PYG{p}{\PYGZcb{}}
            \PYG{k}{else}
            \PYG{p}{\PYGZob{}}
                \PYG{k}{if} \PYG{p}{(}\PYG{n}{position\PYGZus{}x} \PYG{p}{\PYGZlt{}} \PYG{n}{lstControls}\PYG{p}{[}\PYG{n}{position\PYGZus{}y}\PYG{p}{]}\PYG{p}{.}\PYG{n}{Count}\PYG{p}{(}\PYG{p}{)}\PYG{p}{)}
                \PYG{p}{\PYGZob{}}


                    \PYG{k}{if} \PYG{p}{(}\PYG{n}{lstControls}\PYG{p}{[}\PYG{n}{position\PYGZus{}y}\PYG{p}{]}\PYG{p}{[}\PYG{n}{position\PYGZus{}x}\PYG{p}{]} \PYG{p}{=}\PYG{p}{=} \PYG{k}{null}\PYG{p}{)}
                    \PYG{p}{\PYGZob{}}
                        \PYG{k+kt}{int} \PYG{n}{x} \PYG{p}{=} \PYG{n}{position\PYGZus{}x} \PYG{p}{\PYGZhy{}} \PYG{l+m}{1}\PYG{p}{;}
                        \PYG{k+kt}{int} \PYG{n}{y} \PYG{p}{=} \PYG{n}{position\PYGZus{}y}\PYG{p}{;}
                        \PYG{k}{while} \PYG{p}{(}\PYG{n}{lstControls}\PYG{p}{[}\PYG{n}{position\PYGZus{}y}\PYG{p}{]}\PYG{p}{[}\PYG{n}{x}\PYG{p}{]} \PYG{p}{=}\PYG{p}{=} \PYG{k}{null}\PYG{p}{)}
                        \PYG{p}{\PYGZob{}}
                            \PYG{k}{if} \PYG{p}{(}\PYG{n}{x} \PYG{p}{\PYGZlt{}} \PYG{l+m}{0}\PYG{p}{)}
                            \PYG{p}{\PYGZob{}}

                                \PYG{k}{break}\PYG{p}{;}
                            \PYG{p}{\PYGZcb{}}
                            \PYG{n}{x}\PYG{p}{\PYGZhy{}}\PYG{p}{\PYGZhy{}}\PYG{p}{;}
                        \PYG{p}{\PYGZcb{}}
                        \PYG{n}{lstControls}\PYG{p}{[}\PYG{p}{(}\PYG{n}{position\PYGZus{}y}\PYG{p}{)}\PYG{p}{]}\PYG{p}{[}\PYG{n}{x}\PYG{p}{]}\PYG{p}{.}\PYG{n}{Focus}\PYG{p}{(}\PYG{p}{)}\PYG{p}{;}
                        \PYG{n}{Position\PYGZus{}x} \PYG{p}{=} \PYG{n}{x}\PYG{p}{;}
                    \PYG{p}{\PYGZcb{}}
                    \PYG{k}{else}
                    \PYG{p}{\PYGZob{}}
                        \PYG{n}{lstControls}\PYG{p}{[}\PYG{p}{(}\PYG{n}{position\PYGZus{}y}\PYG{p}{)}\PYG{p}{]}\PYG{p}{[}\PYG{n}{position\PYGZus{}x}\PYG{p}{]}\PYG{p}{.}\PYG{n}{Focus}\PYG{p}{(}\PYG{p}{)}\PYG{p}{;}
                    \PYG{p}{\PYGZcb{}}
                \PYG{p}{\PYGZcb{}}
                \PYG{k}{else} \PYG{p}{\PYGZob{}}
                    \PYG{n}{position\PYGZus{}y}\PYG{p}{\PYGZhy{}}\PYG{p}{\PYGZhy{}}\PYG{p}{;}
                \PYG{p}{\PYGZcb{}}
            \PYG{p}{\PYGZcb{}}
        \PYG{p}{\PYGZcb{}}
\end{sphinxVerbatim}


\subsection{Passage de la souris à la manette}
\label{\detokenize{organique:passage-de-la-souris-a-la-manette}}
\sphinxAtStartPar
Durant mes tests de l’application, je me suis rendu compte que le fait de passer de la manette à la souris et inversement, posait un problème. Quand je passais de l’un à l’autre la position du curseur n’était pas sauvegardée. Quand je passais de la souris à la manette, le curseur se retrouvait à une position erronée. Pour pouvoir pallier au problème, chaque bouton a une position définie et quand l’utilisateur clique dessus ,un événement se déclenche et notifie la forme que le bouton actuel a changé.

\begin{sphinxVerbatim}[commandchars=\\\{\}]
\PYG{k}{protected} \PYG{k}{override} \PYG{k}{void} \PYG{n+nf}{OnClick}\PYG{p}{(}\PYG{n}{EventArgs} \PYG{n}{e}\PYG{p}{)}
        \PYG{p}{\PYGZob{}}
            \PYG{k}{base}\PYG{p}{.}\PYG{n}{OnClick}\PYG{p}{(}\PYG{n}{e}\PYG{p}{)}\PYG{p}{;}

            \PYG{c+c1}{//tell to the topMainForm which control is active}
            \PYG{n}{XboxUserControl} \PYG{n}{xboxUserControl} \PYG{p}{=} \PYG{p}{(}\PYG{n}{XboxUserControl}\PYG{p}{)}\PYG{k}{this}\PYG{p}{.}\PYG{n}{Parent}\PYG{p}{;}
            \PYG{k}{if} \PYG{p}{(}\PYG{n}{xboxUserControl} \PYG{p}{!}\PYG{p}{=} \PYG{k}{null}\PYG{p}{)}
            \PYG{p}{\PYGZob{}}
            \PYG{n}{xboxUserControl}\PYG{p}{.}\PYG{n}{position\PYGZus{}x} \PYG{p}{=} \PYG{k}{this}\PYG{p}{.}\PYG{n}{contextInfos}\PYG{p}{.}\PYG{n}{position\PYGZus{}x}\PYG{p}{;}
            \PYG{n}{xboxUserControl}\PYG{p}{.}\PYG{n}{position\PYGZus{}y} \PYG{p}{=} \PYG{k}{this}\PYG{p}{.}\PYG{n}{contextInfos}\PYG{p}{.}\PYG{n}{position\PYGZus{}y}\PYG{p}{;}
            \PYG{p}{\PYGZcb{}}


            \PYG{k}{if} \PYG{p}{(}\PYG{p}{(}\PYG{n}{XboxMainForm}\PYG{p}{)}\PYG{k}{this}\PYG{p}{.}\PYG{n}{TopLevelControl} \PYG{p}{!}\PYG{p}{=} \PYG{k}{null}\PYG{p}{)}
            \PYG{p}{\PYGZob{}}
                \PYG{n}{XboxMainForm} \PYG{n}{topMainForm} \PYG{p}{=} \PYG{p}{(}\PYG{n}{XboxMainForm}\PYG{p}{)}\PYG{k}{this}\PYG{p}{.}\PYG{n}{TopLevelControl}\PYG{p}{;}
                \PYG{n}{topMainForm}\PYG{p}{.}\PYG{n}{ActiveControl1} \PYG{p}{=} \PYG{n}{xboxUserControl}\PYG{p}{;}
            \PYG{p}{\PYGZcb{}}
        \PYG{p}{\PYGZcb{}}
\end{sphinxVerbatim}


\chapter{Tests}
\label{\detokenize{test:tests}}\label{\detokenize{test::doc}}

\section{Test de l’interface graphique}
\label{\detokenize{test:test-de-linterface-graphique}}
\sphinxAtStartPar
Pour pouvoir tester mon interface graphique, j’ai créé un projet contenant des différentes vues pour pouvoir essayer le déplacement et le changement de vue.

\sphinxAtStartPar
\sphinxincludegraphics{{caiman_interface_1}.png}

\sphinxAtStartPar
\sphinxincludegraphics{{caiman_interface_2}.png}

\sphinxAtStartPar
Ce programme de test m’a permis de me rendre compte des différentes difficultés que j’ai pu avoir, ainsi que de pouvoir les corriger avant de passer à la partie développement de l’application.

\sphinxAtStartPar
Le but de ces tests étaient de ne pas avoir à me soucier de l’interface pendant que je devais développer la gestion des jeux et des émulateurs.


\section{Test de Caiman}
\label{\detokenize{test:test-de-caiman}}

\subsection{Test globaux}
\label{\detokenize{test:test-globaux}}
\sphinxAtStartPar
Pour pouvoir faire des tests, j’ai d’abord pensé à utiliser les tests de Visual Studio, malheureusement cela n’est pas vraiment applicable à Caiman. Alors, je n’ai donc testé que quelques classes de cette manière. Pour quand même essayer de trouver le plus de bogues possibles, j’ai demandé à plusieurs personnes de mon entourage d’essayer mon application pour me faire remonter les différents soucis qu’ils ont pu avoir.

\sphinxAtStartPar
Grâce aux différents retours, j’ai pu connaître un bon nombre de bogues liés à l’interface utilisable à la manettes et aux différents soucis d’ergonomie de Caiman. J’ai pu fournir à différents moments du développement des versions Alpha de Caiman, cela me permettait d’avoir des retours le plus régulièrement possible pour éviter d’avoir trop de contraintes à régler à la fin du développement.


\subsection{Test synchronisation des sauvegardes}
\label{\detokenize{test:test-synchronisation-des-sauvegardes}}
\sphinxAtStartPar
Pour tester la synchronisation des sauvegardes, j’ai créé un “protocole”. Ce protocole consiste à utiliser les sauvegardes du jeu “Metroid Prime” pour tester les sauvegardes des jeux de gamecube. Metroid Prime nous laisse créer un plusieurs fichiers de sauvegarde directement au lancement du jeu, je me sers donc de ces fichiers pour savoir si la sauvegarde est bien présente.

\sphinxAtStartPar
Pour tester les sauvegardes des jeux de Playstation 2, j’ai utilisé le jeu Dragon Quest: VIII, il a l’avantage de pouvoir créer un fichier de sauvegarde avec un nom spécifique. Donc, J’ai créé un fichier de sauvegarde avec mon prénom et un autre avec un nom complètement étrange. Si je ne trouve pas les sauvegardes quand je lance le jeu, alors je sais que la synchronisation n’a pas marché.


\chapter{Planning}
\label{\detokenize{planning:planning}}\label{\detokenize{planning::doc}}

\section{Planning prévisionnel}
\label{\detokenize{planning:planning-previsionnel}}
\sphinxAtStartPar
\sphinxincludegraphics{{planning_previsionnel-2}.png}

\sphinxAtStartPar
\sphinxincludegraphics{{planning_previsionnel-1}.png}


\section{Planning effectif}
\label{\detokenize{planning:planning-effectif}}
\sphinxAtStartPar
\sphinxincludegraphics{{planning_effectif_2}.png}

\sphinxAtStartPar
\sphinxincludegraphics{{planning_effectif_1}.png}


\chapter{Conclusion et perspectives}
\label{\detokenize{conclusion:conclusion-et-perspectives}}\label{\detokenize{conclusion::doc}}

\section{Problème rencontrés}
\label{\detokenize{conclusion:probleme-rencontres}}
\sphinxAtStartPar
Durant ce travail, j’ai dû surmonter plusieurs difficultés, que ce soit des choses que je ne connaissais pas, ou bien des inconnues techniques.

\sphinxAtStartPar
Je pense que la plus grosse contrainte que j’ai eu, est que j’ai dû créer une API que je n’avais pas prévu à la base. Quand j’ai commencé l’application Caiman, j’ai discuté avec M.Maréchal et M.Schmid, les deux m’ont conseillé de créer une API pour accéder à la base de données depuis Caiman. Donc, J’ai pris une semaine complète pour créer mon API.Originellement, ce temps passé sur l’API n’était pas prévu mais je pense que cela m’a fait gagner du temps finalement.

\sphinxAtStartPar
Au niveau du code tout s’est relativement bien passé. L’un des seuls points où j’ai eu des difficultés est le téléchargement de jeux. Que ce soit du côté serveur ou client, je n’avais jamais fais de l’upload / téléchargement de fichiers. J’ai été conseillé par M.Schmid, ce qui m’a bien aidé.

\sphinxAtStartPar
Un autre point qui a été compliqué c’est la façon dont j’allais distribuer mon application. L’outil inclus dans Visual Studio n’étant pas vraiment mis à jour par Microsoft, j’ai du chercher de mon côté comment créer un programme d’installation, mais je n’ai pas vraiment trouvé. Finalement, M.Schmid m’a aidé mais tout n’était pas réglé pour autant. Le problème était que quand j’installais Caiman, certains fichiers des émulateurs n’étaient plus là donc je ne pouvais plus exécuter de jeu.

\sphinxAtStartPar
Le plus gros défi a été de créer une interface graphique. Durant notre formation nous ne sommes pas formés pour cela. Alors, j’ai dû beaucoup me documenter pour le projet. Finalement, je pense avoir une interface agréable et de surcroît utilisable avec une manette.


\section{Expérience acquise durant le travail}
\label{\detokenize{conclusion:experience-acquise-durant-le-travail}}
\sphinxAtStartPar
J’ai énormément appris de choses durant ce travail, en particulier sur le développement en C\#, et sur le fonctionnement des émulateurs. Je n’avais jamais créé d’application aussi complexe. Pour faire fonctionner mon application, j’ai dû faire énormément de choses différentes. J’ai apprécié mettre toutes ces choses en relation du développement C\#, du développement de site web, la création d’une API et la configuration d’un serveur Debian. J’ai donc pu créer une application de A à Z (sans compter les émulateurs bien sur) qui au final fonctionnent bien.


\section{Améliorations envisageables}
\label{\detokenize{conclusion:ameliorations-envisageables}}

\subsection{Caiman C\#}
\label{\detokenize{conclusion:caiman-c}}
\sphinxAtStartPar
J’ai eu une grande frustration durant ce travail, le fait de ne pas avoir 2 mois de plus. Chaque jour sur lequel j’ai travaillé sur l’application Caiman, j’avais énormément d’idées qui me venaient en tête. Les principales étaient.:
\begin{itemize}
\item {} 
\sphinxAtStartPar
Ajouter d’autres émulateurs

\item {} 
\sphinxAtStartPar
Synchroniser les paramètres de caiman entre les différents pc de l’utilisateur

\item {} 
\sphinxAtStartPar
Améliorer l’interface

\item {} 
\sphinxAtStartPar
Ajouter un système d’amis pour voir plus facilement ce à quoi jouent les autres joueurs

\item {} 
\sphinxAtStartPar
Créer un meilleur tri dans les jeux

\item {} 
\sphinxAtStartPar
Faire la liste des jeux les plus téléchargés

\item {} 
\sphinxAtStartPar
Avoir une meilleur gestion des téléchargements

\item {} 
\sphinxAtStartPar
Prendre en compte plus de manettes (PS4,Switch,etc)

\end{itemize}

\sphinxAtStartPar
Vous l’avez compris, cette liste est sans fin. Mais je savais pertinemment avant même de commencer ce travail que je ne pourrais pas être vraiment satisfait du résultat final.

\sphinxAtStartPar
J’aurais aussi aimé pouvoir créer un installateur pour Caiman, mais quand j’ai essayé j’ai eu des soucis avec les fichiers des émulateurs. J’ai donc finalement dû y renoncer à cause d’un manque de temps.


\subsection{API}
\label{\detokenize{conclusion:api}}
\sphinxAtStartPar
L’API n’étant pas prévu à la base, elle a été une charge de travail supplémentaire mais qui finalement m’a fait gagner du temps. Des améliorations sont possibles, par exemple je pense que mes endpoints pourraient être améliorés et surtout plus nombreux. Par exemple, je pourrais créer un endpoint pour les téléchargements des jeux et des fichiers de sauvegarde.


\section{Conclusion Globale}
\label{\detokenize{conclusion:conclusion-globale}}
\sphinxAtStartPar
J’ai eu un réel plaisir à créer Caiman. Ce fut une expérience vraiment intéressante pour moi. Durant ma formation, je n’avais jamais pu créer de “vrai” projet en C\#, alors quand j’ai dû choisir le sujet de mon travail, j’ai tout de suite sauté sur l’occasion d’en faire.

\sphinxAtStartPar
J’ai eu la chance de pouvoir travailler sur ce projet en sachant qu’il était atypique. Je voulais vraiment pour mon diplôme travailler sur un sujet qui me plait.

\sphinxAtStartPar
Pour conclure, je pense que Caiman est une réussite. Mon but principal était qu’une personne qui ne connaisse rien aux jeux vidéo et en particulier aux émulateurs puisse jouer à un jeu en moins de 5 minutes, ce qui est le cas. Voilà pourquoi je pense que mon projet est réussi.


\chapter{Remerciement}
\label{\detokenize{remerciment:remerciement}}\label{\detokenize{remerciment::doc}}
\sphinxAtStartPar
Pour commencer, je tenais en particulier à remercier les personnes qui ont pris le temps de me faire des retours sur Caiman. Sans les retours que j’ai reçus de leur part, Caiman ne serait pas aussi bien peaufiné qu’il l’est actuellement. Ensuite, j’aimerais remercier M.Maréchal et M.Schmid pour leurs conseils et leurs retours tout le long de mon travail de diplôme. Pour finir, j’aimerais remercier les autres élèves de ma classe avec qui j’ai passé de très bon moment et qui m’ont aussi donné des conseils pertinents. Je tiens en particulier à remercier M.Borel\sphinxhyphen{}Jaquet qui m’a énormément aidé avec la structure de l’API.


\chapter{Sources}
\label{\detokenize{sources:sources}}\label{\detokenize{sources::doc}}
\sphinxAtStartPar
\sphinxurl{https://stackoverflow.com/questions/30013448/c-sharp-create-dir-in-appdata/30013494}

\sphinxAtStartPar
\sphinxurl{https://www.digitalocean.com/community/tutorials/how-to-install-the-apache-web-server-on-debian-10}

\sphinxAtStartPar
\sphinxurl{https://linuxize.com/post/how-to-install-php-on-debian-10/}

\sphinxAtStartPar
\sphinxurl{https://stackoverflow.com/questions/307688/how-to-download-a-file-from-a-url-in-c}

\sphinxAtStartPar
\sphinxurl{https://www.tutorialspoint.com/check-if-a-file-exists-in-chash}

\sphinxAtStartPar
\sphinxurl{https://stackoverflow.com/questions/29728829/cannot-download-file-using-fpassthru}

\sphinxAtStartPar
\sphinxurl{https://stackoverflow.com/questions/26010606/how-to-make-winform-scrollable-in-c-sharp}

\sphinxAtStartPar
\sphinxurl{https://stackoverflow.com/questions/8255533/how-to-add-new-line-into-txt-file}

\sphinxAtStartPar
\sphinxurl{https://stackoverflow.com/questions/837488/how-can-i-get-the-applications-path-in-a-net-console-application}

\sphinxAtStartPar
\sphinxurl{https://stackoverflow.com/questions/2104099/c-sharp-if-then-directives-for-debug-vs-release}

\sphinxAtStartPar
\sphinxurl{https://www.reddit.com/r/DolphinEmulator/comments/gtj0es/dolphin\_for\_windows\_no\_gui/}

\sphinxAtStartPar
(pour les images des jeux)

\sphinxAtStartPar
\sphinxurl{https://ggapp.io/}

\sphinxAtStartPar
\sphinxurl{https://stackoverflow.com/questions/8103860/move-uploaded-file-gives-failed-to-open-stream-permission-denied-error}

\sphinxAtStartPar
\sphinxurl{https://www.c-sharpcorner.com/UploadFile/dbeniwal321/how-to-delete-a-file-in-C-Sharp/}

\sphinxAtStartPar
\sphinxurl{https://stackoverflow.com/questions/1288718/how-to-delete-all-files-and-folders-in-a-directory}

\sphinxAtStartPar
\sphinxurl{https://www.c-sharpcorner.com/UploadFile/mahesh/create-a-text-file-in-C-Sharp/}

\sphinxAtStartPar
\sphinxurl{https://stackoverflow.com/questions/4580263/how-to-open-in-default-browser-in-c-sharp}

\sphinxAtStartPar
\sphinxurl{https://docs.microsoft.com/en-us/dotnet/csharp/programming-guide/file-system/how-to-copy-delete-and-move-files-and-folders}

\sphinxAtStartPar
\sphinxurl{https://docs.microsoft.com/en-us/dotnet/standard/io/how-to-compress-and-extract-files}

\sphinxAtStartPar
\sphinxurl{https://www.reddit.com/r/DolphinEmulator/comments/gtj0es/dolphin\_for\_windows\_no\_gui/}

\sphinxAtStartPar
\sphinxurl{http://institutions.ville-geneve.ch/fr/bm/interroge/archives-questions-reponses/detail/question/est-il-legal-de-telecharger-des-films-sur-internet-br/}

\sphinxAtStartPar
\sphinxurl{https://css.comonsoft.com/tutoriels/installation-certificat-lets-encrypt-ssl-debian.htm}


\chapter{Logbook}
\label{\detokenize{logbook:logbook}}\label{\detokenize{logbook::doc}}

\section{19.04.2021}
\label{\detokenize{logbook:id1}}

\subsection{8h05}
\label{\detokenize{logbook:h05}}
\sphinxAtStartPar
Entretiens avec M. Garcia


\subsection{9h05}
\label{\detokenize{logbook:id2}}
\sphinxAtStartPar
Copie de mon disque pour Gawen


\subsection{9h20}
\label{\detokenize{logbook:h20}}
\sphinxAtStartPar
Création du git


\subsection{9h30}
\label{\detokenize{logbook:h30}}
\sphinxAtStartPar
je réfléchi à ce par quoi je vais commencer j’hésite entre commencé entre le site web ou les téléchargement


\subsection{9h40}
\label{\detokenize{logbook:h40}}
\sphinxAtStartPar
Modélisation de la BDD

\sphinxAtStartPar
\sphinxincludegraphics{{bdd_tableau}.jpg}


\subsection{10h30}
\label{\detokenize{logbook:id3}}
\sphinxAtStartPar
Installation de laragon


\subsection{10h40}
\label{\detokenize{logbook:id4}}
\sphinxAtStartPar
Création de la base de données


\subsection{12h40}
\label{\detokenize{logbook:id5}}
\sphinxAtStartPar
Création de la structure du site web


\subsection{14h35}
\label{\detokenize{logbook:h35}}
\sphinxAtStartPar
Création des différente pages et mise en place de bootstrap


\subsection{résumé}
\label{\detokenize{logbook:resume}}
\sphinxAtStartPar
j’ai créer la base de donnée et le site web


\bigskip\hrule\bigskip



\section{20.04.2021}
\label{\detokenize{logbook:id6}}

\subsection{8h05}
\label{\detokenize{logbook:id7}}
\sphinxAtStartPar
je continue à créer le site web je crée les formulaires pour la connexion


\subsection{15h20}
\label{\detokenize{logbook:id8}}
\sphinxAtStartPar
J’ai impliqué la création de compte et la connexion, le mail doit être unique et le username aussi.

\sphinxAtStartPar
j’ai selon les indication de M. Schmid utilisé les fonctions

\sphinxAtStartPar
password\_hash et password\_verify de php.


\subsection{15h30}
\label{\detokenize{logbook:id9}}
\sphinxAtStartPar
j’ajoute des donnée a la main dans la bdd


\subsection{15h56}
\label{\detokenize{logbook:h56}}
\sphinxAtStartPar
aide de M.Schmid pour du sql


\bigskip\hrule\bigskip



\section{21.04.2021}
\label{\detokenize{logbook:id10}}

\subsection{8h10}
\label{\detokenize{logbook:h10}}
\sphinxAtStartPar
Modification de la structure du site et ajout de l’update de mots de passe


\subsection{13h00}
\label{\detokenize{logbook:h00}}
\sphinxAtStartPar
l’affichage des jeux est disponible ainsi qu’une recherche sur les jeux grâce à leurs noms.


\subsection{13h20}
\label{\detokenize{logbook:id11}}
\sphinxAtStartPar
Ajout de champs dans la table game
\begin{itemize}
\item {} 
\sphinxAtStartPar
description

\item {} 
\sphinxAtStartPar
imageName

\end{itemize}


\subsection{15h00}
\label{\detokenize{logbook:id12}}
\sphinxAtStartPar
l’affichage de la recherche et du détail d’un jeu fonctionne mais n’est pas beau.


\bigskip\hrule\bigskip



\section{22.04.2021}
\label{\detokenize{logbook:id13}}

\subsection{8h05}
\label{\detokenize{logbook:id14}}
\sphinxAtStartPar
Ajout de l’affichage des catégories de chaque jeu


\subsection{9h00}
\label{\detokenize{logbook:id15}}
\sphinxAtStartPar
Affichage des jeux qui appartiennent à une catégorie.


\subsection{10h20}
\label{\detokenize{logbook:id16}}
\sphinxAtStartPar
modification de l’interface de recherche


\subsection{notes personnelles}
\label{\detokenize{logbook:notes-personnelles}}\begin{itemize}
\item {} 
\sphinxAtStartPar
je dois ajouter une table pour savoir le nombre d’heure de jeu de chaque utilisateurs

\item {} 
\sphinxAtStartPar
je dois ajouter une gestion des messages d’erreurs.

\item {} 
\sphinxAtStartPar
je peux ajouter une photo de profil

\end{itemize}


\subsection{11h0}
\label{\detokenize{logbook:h0}}
\sphinxAtStartPar
modification de l’interface de connexion et de d’inscription


\subsection{11h50}
\label{\detokenize{logbook:h50}}
\sphinxAtStartPar
suppression de la page de création de compte


\subsection{12h40}
\label{\detokenize{logbook:id17}}
\sphinxAtStartPar
Ajout d’un jeux en favoris


\subsection{15h00}
\label{\detokenize{logbook:id18}}
\sphinxAtStartPar
suppression d’un jeu en favoris


\subsection{a faire demain}
\label{\detokenize{logbook:a-faire-demain}}\begin{itemize}
\item {} 
\sphinxAtStartPar
l’affichage des card dans le dashboard n’est pas bon

\end{itemize}


\bigskip\hrule\bigskip



\section{23.04.2021}
\label{\detokenize{logbook:id19}}

\subsection{8h05}
\label{\detokenize{logbook:id20}}
\sphinxAtStartPar
modification de la l’interface du Dashboard


\subsection{8h40}
\label{\detokenize{logbook:id21}}
\sphinxAtStartPar
Ajout d’un champ dans la table utilisateur pour spécifier si l’utilisateur est privé ou non
si l’utilisateur n’est pas privé tout le monde va pouvoir voir son profils


\subsection{9h30}
\label{\detokenize{logbook:id22}}
\sphinxAtStartPar
la modification du paramètre pour savoir si le compte est privé ou non


\subsection{12h20}
\label{\detokenize{logbook:id23}}
\sphinxAtStartPar
test de Git Hook


\subsection{13h30}
\label{\detokenize{logbook:id24}}
\sphinxAtStartPar
modification de l’interface theme blanc \sphinxhyphen{}> dark


\subsection{15h20}
\label{\detokenize{logbook:id25}}
\sphinxAtStartPar
Création de la page de téléchargement et mise en place de du téléchargement de Caiman depuis le site


\subsection{notes pour la prochaine fois}
\label{\detokenize{logbook:notes-pour-la-prochaine-fois}}\begin{itemize}
\item {} 
\sphinxAtStartPar
Je dois créer la partie Administrateur du site

\item {} 
\sphinxAtStartPar
Je dois créer un une fonctionnalité qui me permet de gérer les messages d’erreurs

\item {} 
\sphinxAtStartPar
Je dois sécuriser l’accès au pages

\item {} 
\sphinxAtStartPar
je dois sécuriser les différents formulaires

\item {} 
\sphinxAtStartPar
je dois je dois me renseigner comment uploader des gros fichier depuis un poste clients

\item {} 
\sphinxAtStartPar
je dois changer de navbar

\end{itemize}


\bigskip\hrule\bigskip



\section{26.04.2021}
\label{\detokenize{logbook:id26}}

\subsection{8h05}
\label{\detokenize{logbook:id27}}
\sphinxAtStartPar
notes personnelles:
\begin{itemize}
\item {} 
\sphinxAtStartPar
je dois ajouter la possibilité d’afficher la page d’un utilisateur

\item {} 
\sphinxAtStartPar
je dois corriger mon script d’export de base de données

\end{itemize}


\subsection{8h10}
\label{\detokenize{logbook:id28}}
\sphinxAtStartPar
Création de la page dédiée aux administrateurs.


\subsection{8h30}
\label{\detokenize{logbook:id29}}
\sphinxAtStartPar
Ajout de catégorie


\subsection{9h00}
\label{\detokenize{logbook:id30}}
\sphinxAtStartPar
Ajout de jeu


\subsection{notes personnels}
\label{\detokenize{logbook:notes-personnels}}
\sphinxAtStartPar
j’ai regardé plusieurs méthodes pour envoyer un fichier depuis un formulaire en php. Pour l’instant j’utilise les fonctions de base de php et elle fonctionne donc je vais faire des tests une fois le site uploadé sur le Bunker.


\subsection{11h00}
\label{\detokenize{logbook:id31}}
\sphinxAtStartPar
le fichier .iso est uploadé avec le bon nom mais pas encore dans la base de données


\subsection{15h00}
\label{\detokenize{logbook:id32}}
\sphinxAtStartPar
Le jeu est bien ajouté avec le bon nom ainsi que la bonne image.


\subsection{15h45}
\label{\detokenize{logbook:h45}}
\sphinxAtStartPar
il est maintenant possible de mettre à jour le nom, la description ou la console d’un jeu.


\subsection{15h50}
\label{\detokenize{logbook:id33}}
\sphinxAtStartPar
modification de la structure du git


\bigskip\hrule\bigskip



\section{27.04.2021}
\label{\detokenize{logbook:id34}}

\subsection{8h05}
\label{\detokenize{logbook:id35}}
\sphinxAtStartPar
ajout/ suppression de catégories a un jeu


\subsection{9h20}
\label{\detokenize{logbook:id36}}
\sphinxAtStartPar
modification mineur de l’interface


\subsection{10h05}
\label{\detokenize{logbook:id37}}
\sphinxAtStartPar
recherche d’un profil utilisateur


\subsection{12h15}
\label{\detokenize{logbook:h15}}
\sphinxAtStartPar
la recherche et l’affichage d’un profil utilisateur est fonctionnel


\subsection{13h00}
\label{\detokenize{logbook:id38}}
\sphinxAtStartPar
modification de l’interface pour que les jeux s’affiche correctement


\subsection{13h30}
\label{\detokenize{logbook:id39}}
\sphinxAtStartPar
Le site est fonctionnel mais il manque des détail comme les message d’erreur et les droit sur les pages


\subsection{notes personnelles}
\label{\detokenize{logbook:id40}}
\sphinxAtStartPar
pour finir le site il me reste les choses suivante a faire:
\begin{itemize}
\item {} 
\sphinxAtStartPar
Sécuriser les pages

\item {} 
\sphinxAtStartPar
afficher des messages d’erreur

\item {} 
\sphinxAtStartPar
suppression de catégories

\item {} 
\sphinxAtStartPar
suppression de jeu

\item {} 
\sphinxAtStartPar
mot de passe oublié

\item {} 
\sphinxAtStartPar
commenter mon code

\item {} 
\sphinxAtStartPar
pagination pour les recherches

\end{itemize}


\subsection{13h40}
\label{\detokenize{logbook:id41}}
\sphinxAtStartPar
documentation


\bigskip\hrule\bigskip



\section{28.04.2021}
\label{\detokenize{logbook:id42}}

\subsection{8h05}
\label{\detokenize{logbook:id43}}
\sphinxAtStartPar
documentation


\subsection{10h40}
\label{\detokenize{logbook:id44}}
\sphinxAtStartPar
correction d’un bogue sur le nombre d’heure de jeu et riage des jeux par heure de jeu


\subsection{11h00}
\label{\detokenize{logbook:id45}}
\sphinxAtStartPar
gestion des droit d’accès au page


\subsection{13h10}
\label{\detokenize{logbook:id46}}
\sphinxAtStartPar
modification de la navbar


\subsection{14h00}
\label{\detokenize{logbook:id47}}
\sphinxAtStartPar
gestion des erreurs de du login


\bigskip\hrule\bigskip



\section{29.04.2021}
\label{\detokenize{logbook:id48}}

\subsection{8h30}
\label{\detokenize{logbook:id49}}
\sphinxAtStartPar
Documentation


\subsection{10h20}
\label{\detokenize{logbook:id50}}
\sphinxAtStartPar
Mr. Garcia m’a aidé à mettre en place ma documentation doxygen


\subsection{11h00}
\label{\detokenize{logbook:id51}}
\sphinxAtStartPar
documentation


\bigskip\hrule\bigskip



\section{30.04.2021}
\label{\detokenize{logbook:id52}}

\subsection{8h05}
\label{\detokenize{logbook:id53}}
\sphinxAtStartPar
Correction du logbook


\subsection{8h20}
\label{\detokenize{logbook:id54}}
\sphinxAtStartPar
supression de code inutile sur le site et mis a jour de la doccumentation


\subsection{9h30}
\label{\detokenize{logbook:id55}}
\sphinxAtStartPar
ajout de contrainte dans la base de données


\subsection{10h45}
\label{\detokenize{logbook:id56}}
\sphinxAtStartPar
Création du planning effectif


\subsection{13h00}
\label{\detokenize{logbook:id57}}
\sphinxAtStartPar
Configuration de debian

\sphinxAtStartPar
user tfp : FTPdiplomant
password ftp : SuperCfpt@

\sphinxAtStartPar
J’ai pus installer apache, php, et mysql mais je n’arrive pas à me connecter en ftp. A la connexion je bloque sur l’erreur : “Impossible de récupérer le contenu du dossier”. Malgré l’aide de Mr.Schmidt je n’ai toujours pas réussi.


\subsection{15h20}
\label{\detokenize{logbook:id58}}
\sphinxAtStartPar
Je n’arrive toujours pas a me connecter au ftp, je ne sais pas si le probléme viens de ma configuration ou du firewall


\section{notes pour le premier rendu}
\label{\detokenize{logbook:notes-pour-le-premier-rendu}}
\sphinxAtStartPar
J’ai durant ces deux premières semaines, créé le site internet de Caiman. Le site n’est pas fini à 100\% mais les fonctionnalités de base sont toutes implémentées. Les fonctionnalités actuelles permettent de faire toutes les choses nécessaires au fonctionnement de l’application. J’ai pris plus de temps que prévu à réaliser le site mais durant la création de mon planning prévisionnel j’ai fait des erreurs, j’ai par exemple oublié de planifier la création des fonctionnalités d’administration ( Ajout de jeux, ajout de catégories, assignation de catégories à un jeu, upload de jeux, etc ).

\sphinxAtStartPar
La documentation du site n’est pas forcément touffue mais le site n’est pas particulièrement complexe, il m’a pris du temps dû au nombre de tables à gérer. Il reste à mettre en place la récupération de mot de passe mais j’ai décidé de passer à l’interface graphique de l’application C\# dès la semaine prochaine. Étant donné que le projet est la partie la plus importante de l’application, je ne vais pas configurer la récupération de mot de passe maintenant.


\bigskip\hrule\bigskip



\section{03.05.2021}
\label{\detokenize{logbook:id59}}

\subsection{8h10}
\label{\detokenize{logbook:id60}}
\sphinxAtStartPar
Réflexion sur l’interface graphique et création du projet


\subsection{8h15}
\label{\detokenize{logbook:id61}}
\sphinxAtStartPar
Importation de la classe XboxController.cs que j’ai créé précédemment.

\sphinxAtStartPar
\sphinxincludegraphics{{schema-form}.jpg}

\sphinxAtStartPar
La table MainForm contient un XboxController() cette classe permet de connaître les manettes connectés au pc et de recevoir leur inputs.

\sphinxAtStartPar
Elle contient aussi une liste de form, ces formes sont les différentes fenêtres de l’application (si possible j’aimerai faire que seul une fenêtre soit active).

\sphinxAtStartPar
Je fais des test avec les usercontrols


\subsection{10h00}
\label{\detokenize{logbook:id62}}
\sphinxAtStartPar
je n’arrive pas a afficher dynamiquement quand une action est faite.


\subsection{12h00}
\label{\detokenize{logbook:id63}}
\sphinxAtStartPar
Mon problème venait du fait que je n’initialisait pas l “usercontrol.


\subsection{13h00}
\label{\detokenize{logbook:id64}}
\sphinxAtStartPar
Je vais essayer de me baser sur la structure des div en html pour gérer le contenu de l’affichage.
\sphinxincludegraphics{{schema-div}.jpeg}


\subsection{14h00}
\label{\detokenize{logbook:id65}}
\sphinxAtStartPar
FInalisation de la configuration du serveur


\bigskip\hrule\bigskip



\section{04.05.2021}
\label{\detokenize{logbook:id66}}

\subsection{8h05}
\label{\detokenize{logbook:id67}}
\sphinxAtStartPar
Je continue à faire des test pour pouvoir bouger le focus d’une “div” a une autre


\subsection{11h00}
\label{\detokenize{logbook:id68}}
\sphinxAtStartPar
Je continue à faire des test mais j’ai apporté des modifications:

\sphinxAtStartPar
le MainForm ne contient pas une liste de list de control, ce n’est pas nécessaire sachant que le lien entre les control est seulement connu des sous contrôle

\sphinxAtStartPar
dorénavant chaque sous control possède une liste de listes de control pour pouvoir se déplacer.

\sphinxAtStartPar
quand on déplace la position maximum dans un sous control il y a deux possibilité.:
un control est disponible dans la direction souhaité
rien n’est disponible
Si un control est disponible alors la main form est informée qu’elle doit changer d’activeForm pour pouvoir se déplacer dans de bonne condition.


\subsection{12h05}
\label{\detokenize{logbook:id69}}
\sphinxAtStartPar
Le déplacement dans chaque form fonctionne et l’on peut passer d’une form à une autre.


\bigskip\hrule\bigskip



\section{05.05.2021}
\label{\detokenize{logbook:id70}}

\subsection{8h05}
\label{\detokenize{logbook:id71}}
\sphinxAtStartPar
Je continue à améliorer le fonctionnement de l’interface.

\sphinxAtStartPar
J’ai décidé de diviser l’interface en 3 “partie”
la navbar
la sidebar
le main contenu

\sphinxAtStartPar
\sphinxincludegraphics{{schema-home-page}.jpeg}


\subsection{10h50}
\label{\detokenize{logbook:id72}}
\sphinxAtStartPar
J’ai créé une nouvelle classe ButtonContext.cs, elle contient les paramètres qui doivent être passer a la forme pour quel sache l’action à exécuter.

\sphinxAtStartPar
Il est maintenant possible de cliquer sur un bouton et la manette va reprendre la ou l’utilisateur a cliqué.


\subsection{12h40}
\label{\detokenize{logbook:id73}}
\sphinxAtStartPar
Je permet le déplacement grâce au joystick gauche


\subsection{13h00}
\label{\detokenize{logbook:id74}}
\sphinxAtStartPar
J’essaie de faire en sorte que je puisse revenir en arrière dans les fenétre


\subsection{15h00}
\label{\detokenize{logbook:id75}}
\sphinxAtStartPar
J’ai essayé de sauver dans une liste les choses précédemment affichée mais j’ai des soucis avec les liens entre les différents panel.

\sphinxAtStartPar
Je vais essayer de recréer les anciens panel a chaque fois au lieu de les recharger.


\bigskip\hrule\bigskip



\section{06.05.2021}
\label{\detokenize{logbook:id76}}

\subsection{8h05}
\label{\detokenize{logbook:id77}}
\sphinxAtStartPar
Je continue a faire en sorte que je puisse retourner en arrière grâce à la touche “B”


\subsection{9h00}
\label{\detokenize{logbook:id78}}
\sphinxAtStartPar
Il est maintenant possible de revenir en arrière dans la navigation


\subsection{10h00}
\label{\detokenize{logbook:id79}}
\sphinxAtStartPar
si il y a des “trou” dans la navigation le curseur le contourne


\subsection{10h40}
\label{\detokenize{logbook:id80}}
\sphinxAtStartPar
J’essaie d’afficher des images dans l’application


\subsection{13h00}
\label{\detokenize{logbook:id81}}
\sphinxAtStartPar
J’ai discuté avec Mr Maréchal de mon git. Pour pouvoir appliquer correctement mon gitignore j’ai dus supprimer les fichier du git


\subsection{14h00}
\label{\detokenize{logbook:id82}}
\sphinxAtStartPar
J’ai fais des recherche sur la publication de projet et j’ai corrigé des bugs


\bigskip\hrule\bigskip



\section{07.05.2021}
\label{\detokenize{logbook:id83}}

\subsection{8h05}
\label{\detokenize{logbook:id84}}
\sphinxAtStartPar
J’ai essayé le paquet “Microsoft Visual Studio Installer Projects” que m’a conseillé M. Schmid. L’installation marche mais certains dossiers de PCSX2 et de Dolphin ne sont pas inclus dans l’installation.


\subsection{10h00}
\label{\detokenize{logbook:id85}}
\sphinxAtStartPar
Je commente les classe que j’ai créé et je supprimer les fonctions qui ne sont pas utilisé


\subsection{12h40}
\label{\detokenize{logbook:id86}}
\sphinxAtStartPar
documentation + Création d’une release pour pouvoir essayer l’interface

\sphinxAtStartPar
Je déplace la documentation du projet web au projet desktop la seul documentation qui reste dans le projet web est celle qui concern son propre code (doxygen)


\subsection{notes personnelles}
\label{\detokenize{logbook:id87}}
\sphinxAtStartPar
Je dois modifier la connexion a la base de données  pour passer des fonctions de cryptographie de PHP au mds. Le problème ce que je ne peux pas me connecter depuis le c\# avec les fonctions de cryptographie de php.


\subsection{15h30}
\label{\detokenize{logbook:id88}}
\sphinxAtStartPar
J’ai mis à jour ma documentation du projet.

\sphinxAtStartPar
J’ai essayé d’ajouter des tâches à mon git mais j’ai eu des soucis pour la signature du projet donc je remet ça à plus tard.


\bigskip\hrule\bigskip



\section{10.05.2021}
\label{\detokenize{logbook:id89}}

\subsection{8h05}
\label{\detokenize{logbook:id90}}
\sphinxAtStartPar
Je modifie la création de mot de passe et la connexion pour qu’un utilisateur puisse se connecter depuis le site web et Caiman.

\sphinxAtStartPar
J’utilisais les fonctions de php (password\_hash et password\_verify) mais je passe a du md5 + salt. La raison est que je ne pouvais pas utiliser ces fonctions pour me connecter depuis l’application c\#.


\subsection{10h10}
\label{\detokenize{logbook:id91}}
\sphinxAtStartPar
je vais essayer de créer un login en C\#

\sphinxAtStartPar
Pour ce faire, je commence par créer une classe “AccessDatabase.cs” pour communiquer avec la base. Pour stocker la la connexionString je l’ai mise dans les settings de l’application.


\subsection{10h45}
\label{\detokenize{logbook:id92}}
\sphinxAtStartPar
Pour pouvoir me connecter a la base de donnée je dois passer par le port 1433 mais il est fermé dans le firewall du coup je dois essayer de me connecter par le port 433


\subsection{13h30}
\label{\detokenize{logbook:id93}}
\sphinxAtStartPar
Je commence à créer la “vrai” interface je commence par les menu de configurations et le menu pour quitter l’application.


\subsection{14h00}
\label{\detokenize{logbook:id94}}
\sphinxAtStartPar
Après une discussion avec M. Scmid j’ai décidé de faire une api.


\subsection{14h30}
\label{\detokenize{logbook:id95}}
\sphinxAtStartPar
J’ai demandé à M.Borel de l’aide pour la structure de mon api, il a pu m’indiquer une structure correcte mais elle est très verbeuse donc je vais surement avoir pour minimum 3 jours a faire mon API.


\bigskip\hrule\bigskip



\section{11.05.2021}
\label{\detokenize{logbook:id96}}

\subsection{8h00}
\label{\detokenize{logbook:id97}}
\sphinxAtStartPar
Je continue mon api, je commence par essayer la structure de M.Borel sur une seule table pour essayer puis je vais faire les autres.


\subsection{11h40}
\label{\detokenize{logbook:id98}}
\sphinxAtStartPar
Je peux maintenant faire des requêtes mais j’ai un souci avec les headers.
Le hearder Autorization n’est pas correctement recu
Je pense que vu le temps que faire une API prend je fais faire en sorte que l’API ne soit que faite pour Caiman. Je vais donc seulement créer les requêtes nécessaires


\subsection{13h00}
\label{\detokenize{logbook:id99}}
\sphinxAtStartPar
Je vais lister les requêtes qui me seront potentiellement utile pour Caiman

\sphinxAtStartPar
liste des jeux
recherche de jeu
affichage des informations d’un jeu
recherche des informations d’un utilisateur
recherche des jeux avec un nombre qui ont été joué par un utilisateur particulier
recherche d’un jeu selon sa catégorie
jeux favoris d’un utilisateur
recherche des jeux qui ont été joué
connexion d’un utilisateur
Création de compte
réception d’un fichier de sauvegarde
réception d’un fichier de configuration


\subsection{13h40}
\label{\detokenize{logbook:id100}}
\sphinxAtStartPar
Je fais un test d’appel a l’API depuis Caiman


\subsection{14h20}
\label{\detokenize{logbook:id101}}
\sphinxAtStartPar
Pour pouvoir utiliser correctement les appels à l’api je dois utiliser des objets que je je rempli avec chaque appel.

\sphinxAtStartPar
Par exemple, si je veux recevoir les informations d’un jeu je créer un objet jeu a qui je vais attribuer les données que je viens de recevoir.

\sphinxAtStartPar
Il me faut donc créer une classe pour l’utilisateur et une classe pour les jeux.


\subsection{15h50}
\label{\detokenize{logbook:id102}}
\sphinxAtStartPar
je dois aussi pouvoir connaître la liste des catégories grâce à l’API


\subsection{16h00}
\label{\detokenize{logbook:id103}}
\sphinxAtStartPar
Il est maintenant possible de récupérer les catégories


\bigskip\hrule\bigskip



\section{12.05.2021}
\label{\detokenize{logbook:id104}}

\subsection{8h05}
\label{\detokenize{logbook:id105}}
\sphinxAtStartPar
Documentation et suppression de code inutile


\subsection{13h00}
\label{\detokenize{logbook:id106}}
\sphinxAtStartPar
J’ai discuté avec M.Smid il m’a donner des conseil sur pour mon api


\subsection{14h30}
\label{\detokenize{logbook:id107}}
\sphinxAtStartPar
J’ai changé les routes de mon api
exemple:
/games/userFavorites/8  => /games/?byUserFavorites=8


\subsection{15h30}
\label{\detokenize{logbook:id108}}
\sphinxAtStartPar
J’ai essayé de mettre l’api en ligne mais quand j’arrive sur une page j’ai une erreur 500.


\bigskip\hrule\bigskip



\section{13.05.2021}
\label{\detokenize{logbook:id109}}

\subsection{11h00}
\label{\detokenize{logbook:id110}}
\sphinxAtStartPar
L’erreur 500 que j’avais était liée à une version de PHP, le serveur avait une version de php 7.3 alors que mon API a besoin d’une version PHP minimum en 7.4.
J’ai donc mis à jour la version présente sur le serveur.


\subsection{12h00}
\label{\detokenize{logbook:id111}}
\sphinxAtStartPar
J’ai corrigé différentes erreurs liées à l’api par exemple quand on envoyait que le username et pas de password pour se connecter une erreur apparaissait.


\subsection{13h00}
\label{\detokenize{logbook:id112}}
\sphinxAtStartPar
J’ai une erreur sur le serveur quand je veux faire une requête d’utilisateur avec sa clé d’API j’ai une erreur 404.

\sphinxAtStartPar
Je vais peut etre passer la recherche d’api en POST et non en GET


\section{14.05.2021}
\label{\detokenize{logbook:id113}}

\subsection{8h00}
\label{\detokenize{logbook:id114}}
\sphinxAtStartPar
J’ai toujours une erreur quand je veux récupérer les informations d’un utilisateur grâce à son apitoken, l’erreur n’est présente que sur le caiman.cfpt.info.
Pour corriger l’erreur je vais passer par du post pour éviter de perdre trop de temps.


\subsection{09h22}
\label{\detokenize{logbook:h22}}
\sphinxAtStartPar
j’ai push sur le serveur une version de l’API qui fonctionne bien, je vais maintenant continuer ma documentation


\subsection{09h30}
\label{\detokenize{logbook:id115}}
\sphinxAtStartPar
je me rend compte qu j’ai une requête qui ne marche pas


\subsection{11h30}
\label{\detokenize{logbook:id116}}
\sphinxAtStartPar
J’ai dû modifier le .htaccess pour pouvoir appliquer les rewriteRules. Le fichier n’était pas pris en compte sur le serveur je l’ai donc mis dans le dossier ou point le virtual host.


\subsection{12h00}
\label{\detokenize{logbook:id117}}
\sphinxAtStartPar
documentation


\bigskip\hrule\bigskip



\section{17.05.2021}
\label{\detokenize{logbook:id118}}

\subsection{8h05}
\label{\detokenize{logbook:id119}}
\sphinxAtStartPar
Création d’une ébauche de diagram de classe sur le tableau.

\sphinxAtStartPar
J’ai repris en grande partie le diagramme de mon POC en y ajoutant une gestion des sauvegardes et des téléchagement.


\subsection{9h00}
\label{\detokenize{logbook:id120}}
\sphinxAtStartPar
Création du schéma sous UMLetino

\sphinxAtStartPar
Première version du diagramme de la logique de Caiman.


\subsection{13h20}
\label{\detokenize{logbook:id121}}
\sphinxAtStartPar
J’essaie de réfléchir à la manière de télécharger les sauvegardes et la façon dont je dois stocker les jeux que l’utilisateur a téléchargé.

\sphinxAtStartPar
Je pense que je vais faire en sorte de vérifier si les jeux qui devraient être présents sur le disque de l’utilisateur le sont réellement.

\sphinxAtStartPar
Je vais faire en sorte que l’utilisateur ait la possibilité de pouvoir télécharger les jeux dans un dossier spécifique. Il va devoir au premier lancement de l’application spécifier ou l’installation doit se faire. (c’est peut etre contraire a mon but de faire une application simple d’utilisation).

\sphinxAtStartPar
Je dois aussi savoir si le disque de l’utilisateur n’a pas la place requise pour télécharger le jeu demandé.


\subsection{14h35}
\label{\detokenize{logbook:id122}}
\sphinxAtStartPar
Discussion avec M.Maréchal


\section{18.05.2021}
\label{\detokenize{logbook:id123}}

\subsection{8h05}
\label{\detokenize{logbook:id124}}
\sphinxAtStartPar
Je vais créer une page de connexion pour l’utilisateur


\subsection{10h00}
\label{\detokenize{logbook:id125}}
\sphinxAtStartPar
La connection marche, je vais maintenant télécharger les images des jeux


\subsection{12h00}
\label{\detokenize{logbook:id126}}
\sphinxAtStartPar
Je commence a faire en sorte que quand un nouveau jeu est reçu depuis l’API son image est automatiquement télécharger et mise dans le dossier \%appdata\% de l’application


\subsection{13h30}
\label{\detokenize{logbook:id127}}
\sphinxAtStartPar
J’affiche dans la sidebar le nom des catégories


\subsection{15h00}
\label{\detokenize{logbook:id128}}
\sphinxAtStartPar
Quand je clique sur une catégorie les jeux de la catégorie choisie sont afficher.


\bigskip\hrule\bigskip



\section{19.05.2021}
\label{\detokenize{logbook:id129}}

\subsection{8h05}
\label{\detokenize{logbook:id130}}
\sphinxAtStartPar
je fais en sorte d’afficher tous les jeux quand je lance l’application.


\subsection{9h00}
\label{\detokenize{logbook:id131}}
\sphinxAtStartPar
je vais maintenant faire en sorte de pouvoir voir les détails d’un jeu.


\subsection{9h30}
\label{\detokenize{logbook:id132}}
\sphinxAtStartPar
J’ai un problème avec les image bouton le clic n’est pas pris en compte


\subsection{11h00}
\label{\detokenize{logbook:id133}}
\sphinxAtStartPar
j’ai corriger le soucis des boutons et je modifie un petit peu l’interface


\subsection{13h00}
\label{\detokenize{logbook:id134}}
\sphinxAtStartPar
je commence l’affichage des détails d’un jeu


\subsection{14h00}
\label{\detokenize{logbook:id135}}
\sphinxAtStartPar
je fais des recherche sur la façon de télécharger un jeu


\bigskip\hrule\bigskip



\section{20.05.2021}
\label{\detokenize{logbook:id136}}

\subsection{8h05}
\label{\detokenize{logbook:id137}}
\sphinxAtStartPar
Je crée une route pour pouvoir recevoir le lien de téléchargement d’un jeu.


\subsection{09h15}
\label{\detokenize{logbook:id138}}
\sphinxAtStartPar
Je me rend compte que pour créer la route de téléchargement je dois:

\sphinxAtStartPar
Avoir le lien du site
Avoir le lien pour la console
avoir le nom du fichier

\sphinxAtStartPar
Pour avec ces différents éléments je dois modifier mon API pour pouvoir faire tout ça.


\subsection{11h00}
\label{\detokenize{logbook:id139}}
\sphinxAtStartPar
J’arrive à créer l’URL mais le fichier n’est pas accessible en dehors du serveur donc je ne sais pas vraiment quoi faire


\subsection{12h50}
\label{\detokenize{logbook:id140}}
\sphinxAtStartPar
Correction d’un problème où l’on pouvait se déplacer dans une case qui n’existe pas donc l’application plantait.


\subsection{14h00}
\label{\detokenize{logbook:id141}}
\sphinxAtStartPar
Je commence a faire en sorte que je puisse créer la route pour télécharger le fichier des jeux


\bigskip\hrule\bigskip



\section{21.05.2021}
\label{\detokenize{logbook:id142}}

\subsection{9h00}
\label{\detokenize{logbook:id143}}
\sphinxAtStartPar
Je voulais télécharger un jeu en passant par le webClient en C\# mais ma route est en POST dans mon API. Malheureusement je ne peux pas fournir de paramètres en POST avec la fonction downloadFileAsync donc je dois passer ma route en GET


\subsection{11h00}
\label{\detokenize{logbook:id144}}
\sphinxAtStartPar
j’ai changé ma route et j’en ai ajouté pour pouvoir connaître les dossiers ou je dois stocker les jeux et le nom du fichier d’un jeu.


\subsection{16h00}
\label{\detokenize{logbook:id145}}
\sphinxAtStartPar
J’ai réussi à afficher la liste des téléchargements  et maintenant  je suis en train de faire en sorte de pouvoir ajouter un jeu au favoris. j’ai réglé des soucis de droit d’accès au fichier qui sont en cours de lecture.


\bigskip\hrule\bigskip



\section{24.05.2021}
\label{\detokenize{logbook:id146}}

\subsection{15h15}
\label{\detokenize{logbook:id147}}
\sphinxAtStartPar
les téléchargement se font maintenant l’un après l’autre


\subsection{16h35}
\label{\detokenize{logbook:id148}}
\sphinxAtStartPar
Affichage des jeux téléchargé


\bigskip\hrule\bigskip



\section{25.05.2021}
\label{\detokenize{logbook:id149}}

\subsection{8h10}
\label{\detokenize{logbook:id150}}
\sphinxAtStartPar
je dois modifier l’API pour pouvoir manipuler les jeux favoris des utilisateur


\subsection{10h25}
\label{\detokenize{logbook:h25}}
\sphinxAtStartPar
il est maintenant possible d’ajouter et de supprimer un jeu des favoris.
je vais maintenant faire en sorte de pouvoir exécuter des jeux depuis caiman.


\subsection{10h30}
\label{\detokenize{logbook:id151}}
\sphinxAtStartPar
Je regarde le code que j’ai fais pour le poc pour voir si je peux reprendre des parties de codes.


\subsection{14h00}
\label{\detokenize{logbook:id152}}
\sphinxAtStartPar
J’ai un soucis avec le lancement des jeux de PS2


\subsection{14h25}
\label{\detokenize{logbook:id153}}
\sphinxAtStartPar
Le problème venait des paramètres j’avais oublié une espace entre le nom du fichier à exécuter et les paramètres


\subsection{15h50}
\label{\detokenize{logbook:id154}}
\sphinxAtStartPar
J’ai pus ajouter les émulateurs et appliquer certains paramètres graphique.


\bigskip\hrule\bigskip



\section{26.05.2021}
\label{\detokenize{logbook:id155}}

\subsection{8h05}
\label{\detokenize{logbook:id156}}
\sphinxAtStartPar
Je dois modifier mon api pour pouvoir générer un token à chaque connexion pour ne pas avoir à se connecter à chaque lancement de l’application.


\subsection{10h30}
\label{\detokenize{logbook:id157}}
\sphinxAtStartPar
J’ai modifié l’API, je vais maintenant passer a Caiman.


\subsection{11h00}
\label{\detokenize{logbook:id158}}
\sphinxAtStartPar
Il est maintenant possible de se connecter automatiquement , je vais maintenant modifier un petit peu l’aspect graphique de Caiman.


\subsection{13h00}
\label{\detokenize{logbook:id159}}
\sphinxAtStartPar
je veux afficher le jeux qui est actuellement utilisé par l’utilisateur


\subsection{15h30}
\label{\detokenize{logbook:id160}}
\sphinxAtStartPar
Il est possible de voir le jeu actuellement lancé par l’utilisateur et peut importe comment l’utilisateur quite le jeu Caiman va étre au courant


\subsection{18h00}
\label{\detokenize{logbook:id161}}
\sphinxAtStartPar
Affichage du temps de jeu de la session


\subsection{21h00}
\label{\detokenize{logbook:id162}}
\sphinxAtStartPar
Le nombre de minutes de jeu a comptabilisé


\bigskip\hrule\bigskip



\section{27.05.2021}
\label{\detokenize{logbook:id163}}

\subsection{8h10}
\label{\detokenize{logbook:id164}}
\sphinxAtStartPar
J’ai remarqué que je ne pouvais pas upload de jeu actuellement je vais essayer de comprendre pouquoi.


\subsection{09h00}
\label{\detokenize{logbook:id165}}
\sphinxAtStartPar
J’ai résolu le problème je devais ajouter les droit d’écriture dans mon debian
Par contre hier soir j’ai oublié de push donc je ne peux pas continuer Caiman ce matin


\subsection{10h05}
\label{\detokenize{logbook:id166}}
\sphinxAtStartPar
Je profite de ne pas pouvoir coder pour faire des tests, ajouter des jeux et faire de la doc


\subsection{23h30}
\label{\detokenize{logbook:id167}}
\sphinxAtStartPar
J’ai réussi à scanner un dossier pour savoir si il a y eu une modification.


\bigskip\hrule\bigskip



\section{28.05.2021}
\label{\detokenize{logbook:id168}}

\subsection{9h30}
\label{\detokenize{logbook:id169}}
\sphinxAtStartPar
j’ai reussi a faire en sorte que les sauvegardes de l’utilisateur qui se connecte sont déplacé dans le dossier des sauvegardes des émulateurs.


\subsection{10h00}
\label{\detokenize{logbook:id170}}
\sphinxAtStartPar
Je réfléchi à comment faire pour synchroniser les sauvegardes des utilisateurs. J’ai eu l’idée de créer un sous répertoire pour chaque utilisateur sur le serveur

\sphinxAtStartPar
\#\#\#10h30

\sphinxAtStartPar
J’ai réfléchi et finalement je pense envoyer des fichier zip contenant toutes les sauvegardes d’un utilisateur et cela pour chaque console. Cela a l’avantage de réduire la taille des fichier et de simplifier l’upload et le téléchargement


\subsection{14h00}
\label{\detokenize{logbook:id171}}
\sphinxAtStartPar
J’ai eu un bogue ou les sauvegarde n’était pas envoyé dans le bon dossier


\subsection{14h35}
\label{\detokenize{logbook:id172}}
\sphinxAtStartPar
Documentation


\bigskip\hrule\bigskip



\section{30.05.2021}
\label{\detokenize{logbook:id173}}

\subsection{13h30}
\label{\detokenize{logbook:id174}}
\sphinxAtStartPar
Modification de l’API pour pouvoir télécharger un fichier de sauvegarde


\subsection{16h12}
\label{\detokenize{logbook:h12}}
\sphinxAtStartPar
Je fais en sorte que les sauvegardes se synchronisent sur caiman


\subsection{19h43}
\label{\detokenize{logbook:h43}}
\sphinxAtStartPar
j’ai ENFIN réussi à synchroniser les sauvegardes de Gamecube mais j’ai un soucis avec celle de ps2


\bigskip\hrule\bigskip



\section{31.05.2021}
\label{\detokenize{logbook:id175}}

\subsection{8h05}
\label{\detokenize{logbook:id176}}
\sphinxAtStartPar
Je commente le code de Caiman


\subsection{14h00}
\label{\detokenize{logbook:id177}}
\sphinxAtStartPar
Discussion avec M.Maréchal et ajout de tests pour la classe GameTimer


\bigskip\hrule\bigskip



\section{01.06.2021}
\label{\detokenize{logbook:id178}}

\subsection{8h05}
\label{\detokenize{logbook:id179}}
\sphinxAtStartPar
Ajout de tests pour la classe TimeInGame


\subsection{8h20}
\label{\detokenize{logbook:id180}}
\sphinxAtStartPar
Je veux faire en sorte que Caiman s’exécute sur l’écran principale du pc de l’utilisateur


\subsection{8h30}
\label{\detokenize{logbook:id181}}
\sphinxAtStartPar
J’ai désactiver le .gitIgnore par ce que cela créait des soucis avec les dossier des émulateurs


\bigskip\hrule\bigskip



\section{02.06.2021}
\label{\detokenize{logbook:id182}}

\subsection{8h05}
\label{\detokenize{logbook:id183}}
\sphinxAtStartPar
Commentaire du code de l’API


\subsection{10h30}
\label{\detokenize{logbook:id184}}
\sphinxAtStartPar
documentation


\subsection{15h30}
\label{\detokenize{logbook:id185}}
\sphinxAtStartPar
J’ai corrigé le fait que un compte pouvait être créé sans remplir tous les champ


\bigskip\hrule\bigskip



\section{03.06.2021}
\label{\detokenize{logbook:id186}}

\subsection{8h10}
\label{\detokenize{logbook:id187}}
\sphinxAtStartPar
J’ai eu un problème de conflit avec mon git j’ai du re télécharger tout le projet mais maintenant c’est bon.


\subsection{8h34}
\label{\detokenize{logbook:h34}}
\sphinxAtStartPar
Je continue la documentation


\bigskip\hrule\bigskip



\section{04.06.2021}
\label{\detokenize{logbook:id188}}

\subsection{8h05}
\label{\detokenize{logbook:id189}}
\sphinxAtStartPar
Documentation


\subsection{13h15}
\label{\detokenize{logbook:id190}}
\sphinxAtStartPar
La documentation avance bien je vais donc créer les diagrammes de classe pour le C\#.


\bigskip\hrule\bigskip



\section{07.06.2021}
\label{\detokenize{logbook:id191}}

\subsection{8h10}
\label{\detokenize{logbook:id192}}
\sphinxAtStartPar
Recherche sur comment exporter ma documentation en pdf et je continue ma doc.


\subsection{11h20}
\label{\detokenize{logbook:id193}}
\sphinxAtStartPar
Je vais essayer d’installer un certificat ssl sur le serveur debian


\subsection{13h30}
\label{\detokenize{logbook:id194}}
\sphinxAtStartPar
Je n’arrive pas a installer certbot donc je laisse ca en attente


\subsection{14h00}
\label{\detokenize{logbook:id195}}
\sphinxAtStartPar
Documentation


\bigskip\hrule\bigskip



\section{08.06.2021}
\label{\detokenize{logbook:id196}}

\subsection{8h10}
\label{\detokenize{logbook:id197}}
\sphinxAtStartPar
Documentation


\bigskip\hrule\bigskip



\section{09.06.2021}
\label{\detokenize{logbook:id198}}

\subsection{8h10}
\label{\detokenize{logbook:id199}}
\sphinxAtStartPar
Documentation


\bigskip\hrule\bigskip



\section{10.06.2021}
\label{\detokenize{logbook:id200}}

\subsection{8h10}
\label{\detokenize{logbook:id201}}
\sphinxAtStartPar
Export des fichiers sources en pdf et vérification global



\renewcommand{\indexname}{Index}
\printindex
\end{document}