%% Generated by Sphinx.
\def\sphinxdocclass{report}
\documentclass[a4paper,12pt,french]{sphinxmanual}
\ifdefined\pdfpxdimen
   \let\sphinxpxdimen\pdfpxdimen\else\newdimen\sphinxpxdimen
\fi \sphinxpxdimen=.75bp\relax
%% turn off hyperref patch of \index as sphinx.xdy xindy module takes care of
%% suitable \hyperpage mark-up, working around hyperref-xindy incompatibility
\PassOptionsToPackage{hyperindex=false}{hyperref}
%% memoir class requires extra handling
\makeatletter\@ifclassloaded{memoir}
{\ifdefined\memhyperindexfalse\memhyperindexfalse\fi}{}\makeatother

\PassOptionsToPackage{warn}{textcomp}

\catcode`^^^^00a0\active\protected\def^^^^00a0{\leavevmode\nobreak\ }
\usepackage{cmap}
\usepackage{fontspec}
\defaultfontfeatures[\rmfamily,\sffamily,\ttfamily]{}
\usepackage{amsmath,amssymb,amstext}
\usepackage{babel}



\setmainfont{FreeSerif}[
  Extension      = .otf,
  UprightFont    = *,
  ItalicFont     = *Italic,
  BoldFont       = *Bold,
  BoldItalicFont = *BoldItalic
]
\setsansfont{FreeSans}[
  Extension      = .otf,
  UprightFont    = *,
  ItalicFont     = *Oblique,
  BoldFont       = *Bold,
  BoldItalicFont = *BoldOblique,
]
\setmonofont{FreeMono}[
  Extension      = .otf,
  UprightFont    = *,
  ItalicFont     = *Oblique,
  BoldFont       = *Bold,
  BoldItalicFont = *BoldOblique,
]


\usepackage[Sonny]{fncychap}
\ChNameVar{\Large\normalfont\sffamily}
\ChTitleVar{\Large\normalfont\sffamily}
\usepackage{sphinx}

\fvset{fontsize=\small}
\usepackage{geometry}


% Include hyperref last.
\usepackage{hyperref}
% Fix anchor placement for figures with captions.
\usepackage{hypcap}% it must be loaded after hyperref.
% Set up styles of URL: it should be placed after hyperref.
\urlstyle{same}

\addto\captionsfrench{\renewcommand{\contentsname}{table des matières}}

\usepackage{sphinxmessages}
\setcounter{tocdepth}{1}



\title{Documentation Technique de Caiman}
\date{juin 08, 2021}
\release{1.1.0}
\author{Lorenzo Bauduccio}
\newcommand{\sphinxlogo}{\vbox{}}
\renewcommand{\releasename}{Version}
\makeindex
\begin{document}

\ifdefined\shorthandoff
  \ifnum\catcode`\=\string=\active\shorthandoff{=}\fi
  \ifnum\catcode`\"=\active\shorthandoff{"}\fi
\fi

\pagestyle{empty}
\sphinxmaketitle
\pagestyle{plain}
\sphinxtableofcontents
\pagestyle{normal}
\phantomsection\label{\detokenize{index::doc}}



\chapter{Résumé}
\label{\detokenize{resume:resume}}\label{\detokenize{resume::doc}}
\sphinxAtStartPar
Ce document décrit le processus de conception de mon projet Caiman. Caiman est inspiré par des applications comme Retropie, RetroArch ou Steam. RetroPie est une distribution pour raspberry comprenant des émulateurs pour d’anciennes consoles de jeu. RetroArch est lui une application servant de frontend pour émulateur, RetroArch est disponible sur plusieurs plateformes(Windows, macOS, Xbox).

\sphinxAtStartPar
Caiman est un projet comprenant deux parties distinctes. La première partie consiste en un site web PHP permettant de se créer un compte et de s’identifier. Le site web permet de consulter des informations sur les jeux présent sur le store ainsi que de pouvoir administrer les sites et le store.

\sphinxAtStartPar
La deuxième partie du projet est une application C\# inspirée par le mode Big Picture de steam. Caiman sert de frontend pour différents émulateurs (Dolphin, PCSX2,etc). L’application permet de télécharger les différents jeux qui ont été ajoutés au store prévu pour le projet. Les jeux pris en charge sont ajoutés par des administrateurs via le site internet de Caiman. L’application permet de synchroniser les sauvegardes de l’utilisateur et cela peut importe sur quel pc, il lance l’application.


\chapter{Abstract}
\label{\detokenize{resume:abstract}}
\sphinxAtStartPar
This document describes the design process of my project Caiman. Caiman is inspired by applications like Retropie, RetroArch or Steam. RetroPie is a raspberry distribution including emulators for old game consoles. RetroArch is a frontend application for emulators, RetroArch is available on several platforms (Windows, macOS, Xbox).

\sphinxAtStartPar
Caiman is a project with two distinct parts. The first part consists of a PHP website allowing the visitor to create an account and to identify himself. The website allows you to consult information about the games on the store and to administer the site and the store.

\sphinxAtStartPar
The second part of the project is a C\# application inspired by the Big Picture mode of Steam. Caiman is used as a frontend for different emulators (Dolphin, PCSX2, etc). The application allows you to download the different games that have been added to the store provided for the project. Supported games are added by administrators via the Caiman website. The application allows synchronizing the user’s save, no matter on which PC the user runs the application.


\chapter{Remerciement}
\label{\detokenize{resume:remerciement}}
\sphinxAtStartPar
Pour commencer, je tenais en particulier à remercier les personnes qui ont pris le temps de me faire des retours sur Caiman. Sans les retours que j’ai reçus de leur part, Caiman ne serait pas aussi bien peaufiné qu’il l’est actuellement. Ensuite, j’aimerais remercier M.Maréchal et M.Schmid pour leurs conseils et leurs retours tout le long de mon travail de diplôme. Pour finir, j’aimerais remercier les autres élèves de ma classe avec qui j’ai passé de très bon moment et qui m’ont aussi donné des conseils pertinents. Je tiens en particulier à remercier M.Borel\sphinxhyphen{}Jaquet qui m’a énormément aidé avec la structure de l’API.


\chapter{Légalité du projet}
\label{\detokenize{resume:legalite-du-projet}}

\section{Caiman}
\label{\detokenize{resume:caiman}}
\sphinxAtStartPar
Je pense qu’il est nécessaire d’aborder le sujet de la légalité de Caiman. Selon le droit Suisse, il n’est pas légal de mettre à disposition des fichiers sous droit d’auteur. Il est par contre légal de télécharger des films, séries ou jeux à condition de les utiliser dans un but privé. Caiman ne répondant pas à ces descriptions l’utilisation de l’application n’est pas conforme au droit Suisse.


\section{Utilisation d’émulateurs}
\label{\detokenize{resume:utilisation-demulateurs}}
\sphinxAtStartPar
L’utilisation d’émulateurs n’est pas illégale en soi, en sachant que le code des émulateurs n’est pas la propriété de Sony ou de Nintendo. Par contre, il y a un petit point technique à détailler. L’émulateur PCSX2 doit pour fonctionner utiliser un BIOS de console de Playstation 2. Le BIOS de la Playstation 2 étant soumis à des droits d’auteur, il n’est en théorie pas légal de le distribuer en Suisse.

\sphinxAtStartPar
Je tiens donc à spécifier que Caiman est développé dans un seul but pédagogique. Il n’a aucune vocation à être distribué après la fin de mon travail de diplôme. L’application sera supprimée du serveur de l “école à la fin de mon travail. Par contre, il serait légal de distribuer le projet si je supprime la fonctionnalité de téléchargement de jeu.


\chapter{Installation de Caiman utilisateur}
\label{\detokenize{resume:installation-de-caiman-utilisateur}}
\sphinxAtStartPar
Pour utiliser Caiman, il faut le télécharger depuis \sphinxhref{http://caiman.cfpt.info}{caiman.cfpt.info}. Quand on télécharge l’application, l’utilisateur télécharge un fichier caiman.zip contenant l’application ainsi que les émulateurs, il n’y a donc rien d’autre à télécharger.

\sphinxAtStartPar
Quand l’utilisateur décompresse le fichier caiman.zip, il se retrouve avec un dossier Caiman contenant deux dossiers. Un contenant les émulateurs et un autre contenant les fichiers de Caiman.

\sphinxAtStartPar
\sphinxincludegraphics{{dossier_caiman}.png}

\sphinxAtStartPar
Pour pouvoir exécuter Caiman, il faut se rendre dans le dossier “\textbackslash{}bin\textbackslash{}Debug\textbackslash{}” et ensuite l’utilisateur va trouver le fichier “Caiman.exe”. Ce fichier permet de lancer l’application.

\sphinxAtStartPar
Caiman est donc une application “portable”, c’est\sphinxhyphen{}à\sphinxhyphen{} dire qu’elle ne nécessite pas d’une installation pour pouvoir être lancée.


\chapter{Installation de Caiman serveur}
\label{\detokenize{resume:installation-de-caiman-serveur}}
\sphinxAtStartPar
Pour pouvoir déployer le site et l’API de caiman, il faut que le serveur ait les services suivant:
\begin{itemize}
\item {} 
\sphinxAtStartPar
Apache2

\item {} 
\sphinxAtStartPar
PHP 7.4 (minimum)

\item {} 
\sphinxAtStartPar
mysql

\end{itemize}

\sphinxAtStartPar
Pour la base de données, il faut l’importer dans mysql sans faire de manipulation particulière.

\sphinxAtStartPar
Le site web et l’api doivent se trouver dans le même dossier, en sachant qu’il a des chemins relatifs entre les deux dossiers. ils ne peuvent donc pas être séparés pour l’instant.

\sphinxAtStartPar
Pour pouvoir uploader des jeux, il faut augmenter la taille que le PHP peut recevoir par formulaire, il faut également le configurer pour recevoir minimum 8GB. Pour finir, Il faut autoriser l’écriture dans les dossiers des sauvegardes et des jeux.


\chapter{Introduction}
\label{\detokenize{introduction:introduction}}\label{\detokenize{introduction::doc}}

\section{Contexte}
\label{\detokenize{introduction:contexte}}
\sphinxAtStartPar
Caiman a été réalisé dans le cadre du travail de diplôme de technicien deuxième année. La réussite du travail de diplôme est l’une des conditions pour recevoir son titre de technicien en informatique.


\section{Motivations}
\label{\detokenize{introduction:motivations}}
\sphinxAtStartPar
Le choix de ce travail de diplôme découle de mon envie de faire un travail de diplôme qui serait intéressant, ludique et en lien avec les jeux vidéo. La réalisation d’un jeu vidéo étant trop compliquée pour un travail de diplôme, j’ai décidé de créer une application a mi\sphinxhyphen{}chemin entre un launcher de jeu comme Steam ou GOG et un gestionnaire de VM.

\sphinxAtStartPar
J’utilise depuis des années des émulateurs et je me heurte souvent aux mêmes problèmes, perte de sauvegarde dû à une réinstallation de windows, problème pour ripper mes jeux, téléchargement de chaque émulateur un par un. Donc, J’ai eu l’idée de créer Caiman pour pallier ces différents problèmes.

\sphinxAtStartPar
Un autre aspect important pour moi est la simplicité. Je voulais que l’utilisation de l’application soit le plus simple possible, une personne sans connaissance particulière en informatique doit pouvoir utiliser Caiman sans difficulté. Une autre spécification est que l’utilisation de Caiman puisse se faire à la manette. Pour créer une interface ergonomique, je me suis basé sur l’interface Big Picture de Steam qui a justement été créé pour être utilisé  à la manette.


\chapter{Cahier des charges}
\label{\detokenize{cdc:cahier-des-charges}}\label{\detokenize{cdc::doc}}

\section{Objectifs}
\label{\detokenize{cdc:objectifs}}
\sphinxAtStartPar
Caiman est une application pour Windows regroupant plusieurs émulateurs de console de jeu. L’utilisateur peut exécuter les jeux depuis Caiman.

\sphinxAtStartPar
Caiman a pour but d’être simple d’utilisation, aucune connaissance n’est requise pour l’utiliser. Caiman permet une utilisation complète à la manette ou au clavier/souris. L’avantage de Caiman est qu’il ne requiert aucune configuration. Caiman est utilisable instantanément par l’utilisateur, contrairement  d’un émulateur traditionnel qui requiert une configuration relativement compliquée.

\sphinxAtStartPar
Le téléchargement des jeux se fait directement depuis Caiman, un serveur nommé le “Bunker” contient les fichiers des différents jeux.

\sphinxAtStartPar
Un site web permet de créer un compte pour accéder à l’application , de voir les informations des joueurs ainsi que de télécharger Caiman.


\section{Spécification}
\label{\detokenize{cdc:specification}}

\subsection{Emulateurs et contrôles}
\label{\detokenize{cdc:emulateurs-et-controles}}
\sphinxAtStartPar
Les différents émulateurs qui permettent d’exécuter des jeux dans Caiman sont:
\begin{itemize}
\item {} 
\sphinxAtStartPar
PCSX2 (playstation 2)

\item {} 
\sphinxAtStartPar
Dolphin (gamecube / wii)

\end{itemize}

\sphinxAtStartPar
L’utilisateur doit utiliser une manette de Xbox pour utiliser Caiman à la manette (une seule manette est requise pour jouer mais plusieurs peuvent être connectées, pour jouer en multijoueurs local).

\sphinxAtStartPar
Caiman est utilisable à la manette à l’exception des formulaires (création de compte, importation de jeux).


\subsection{Création d’un compte utilisateur}
\label{\detokenize{cdc:creation-dun-compte-utilisateur}}
\sphinxAtStartPar
L’utilisateur a l’obligation de créer un compte pour pouvoir utiliser Caiman.

\sphinxAtStartPar
L’utilisateur peut créer un compte directement depuis Caiman s’il n’en possède pas.


\subsection{Interface}
\label{\detokenize{cdc:interface}}
\sphinxAtStartPar
Dans l’interface, un système de catégories existe pour permettre de se retrouver plus aisément dans la liste des jeux à télécharger. Les différentes catégories sont créées par un administrateur.

\sphinxAtStartPar
Exemple de catégories:
\begin{itemize}
\item {} 
\sphinxAtStartPar
jeux Mario

\item {} 
\sphinxAtStartPar
jeux Zelda

\item {} 
\sphinxAtStartPar
jeux d’aventure

\item {} 
\sphinxAtStartPar
jeux de réflexion

\item {} 
\sphinxAtStartPar
jeux multijoueur

\item {} 
\sphinxAtStartPar
jeux favoris

\item {} 
\sphinxAtStartPar
etc..

\end{itemize}

\sphinxAtStartPar
L’utilisateur peut ajouter depuis Caiman des jeux favoris pour pouvoir les retrouver plus facilement. ( une catégorie “favoris” est visible contenant les jeux favoris de l’utilisateur authentifié )


\subsection{Paramètre graphiques}
\label{\detokenize{cdc:parametre-graphiques}}
\sphinxAtStartPar
L’utilisateur a la possibilité de modifier certains paramètres d’émulation:
\begin{itemize}
\item {} 
\sphinxAtStartPar
La définition des jeux (définition native, proche de 1080p, proche de 4K).

\item {} 
\sphinxAtStartPar
La langue des machines.

\item {} 
\sphinxAtStartPar
Si les jeux doivent être lancés en plein écran ou non.

\item {} 
\sphinxAtStartPar
Format d’écran 16/9 ou 4/3.

\end{itemize}


\subsection{Gestion des sauvegardes}
\label{\detokenize{cdc:gestion-des-sauvegardes}}
\sphinxAtStartPar
Il existe plusieurs cas où les sauvegardes de l’utilisateur se synchronisent:
\begin{enumerate}
\sphinxsetlistlabels{\arabic}{enumi}{enumii}{}{.}%
\item {} 
\sphinxAtStartPar
Le joueur sauvegarde sa partie directement dans un jeu depuis Caiman. La sauvegarde que l’utilisateur vient de créer est envoyée sur le Bunker et remplace la version présente.

\item {} 
\sphinxAtStartPar
Quand le joueur lance Caiman, une vérification est faite pour savoir si une version plus récente des sauvegardes du joueur sont présentes, si c’est le cas, elles sont téléchargées sur la machine du joueur.

\end{enumerate}


\subsection{Spécifications du “Bunker”}
\label{\detokenize{cdc:specifications-du-bunker}}
\sphinxAtStartPar
Le Bunker contient les fichiers des différents jeux au format (.ISO).

\sphinxAtStartPar
Le Bunker contient les fichiers de sauvegardes des utilisateurs. Ces sauvegardes seront envoyées de la machine de l’utilisateur vers le Bunker quand l’utilisateur sauvegarde sa partie depuis un jeu.

\sphinxAtStartPar
Les données des utilisateurs (email, jeux favoris, heures de jeux) sont stockées dans une base de données.


\subsection{Spécification site web (site présent dans le Bunker)}
\label{\detokenize{cdc:specification-site-web-site-present-dans-le-bunker}}
\sphinxAtStartPar
Sur le site web, il est possible de créer un compte en spécifiant un mail et un mot de passe.

\sphinxAtStartPar
L’interface web permet de rechercher des utilisateurs pour voir leur profil.

\sphinxAtStartPar
L’affichage d’un profil permet de voir les jeux favoris d’un utilisateur ainsi que son nombre d’heures de jeu sur chaque jeu.

\sphinxAtStartPar
Le site web permet à un administrateur d’ajouter des jeux et de modifier les informations des jeux.

\sphinxAtStartPar
Le site web permet de modifier les informations de compte d’un utilisateur.

\sphinxAtStartPar
Un utilisateur peut faire la demande de réinitialiser son mot de passe.


\subsection{Installation}
\label{\detokenize{cdc:installation}}
\sphinxAtStartPar
Caiman est téléchargeable depuis le site web. L’installateur contient les émulateurs, il n’y a donc rien à télécharger d’autre.


\section{Limites du Projet}
\label{\detokenize{cdc:limites-du-projet}}
\sphinxAtStartPar
Caiman sera limité aux jeux que j’ai pu ajouter moi même ainsi qu’aux émulateurs que j’aurais importé, la totalité des consoles de jeu ne sera donc pas prise en charge.


\section{Calendrier}
\label{\detokenize{cdc:calendrier}}
\sphinxAtStartPar
Le début du travail de diplôme est fixé au lundi 19 avril et le rendu est fixé au 11 juin à 12h00.

\sphinxAtStartPar
Le travail est à exécuter soit en présentiel et en télétravail du lundi au vendredi et cela 8h00 par jour.


\section{Matériel}
\label{\detokenize{cdc:materiel}}
\sphinxAtStartPar
J’ai à ma disposition pour la réalisation de Caiman:
\begin{itemize}
\item {} 
\sphinxAtStartPar
Mon pc pour le développement
\begin{itemize}
\item {} 
\sphinxAtStartPar
Intel i7\sphinxhyphen{}7700 3.6 GHz

\item {} 
\sphinxAtStartPar
32 GB de ram

\item {} 
\sphinxAtStartPar
GTX 1060 3GB

\end{itemize}

\item {} 
\sphinxAtStartPar
Un serveur pour héberger le site et les fichiers du store

\item {} 
\sphinxAtStartPar
Une manette de Xbox Series

\end{itemize}


\chapter{Étude d’opportunité}
\label{\detokenize{opportunite:etude-dopportunite}}\label{\detokenize{opportunite::doc}}

\section{RetroPie}
\label{\detokenize{opportunite:retropie}}
\sphinxAtStartPar
\sphinxincludegraphics{{retro_pie}.jpg}

\sphinxAtStartPar
RetroPie est une distribution Linux prévue pour Raspberry et PC. C’est un frontend pour jouer aux jeux d’arcade, d’anciennes consoles et de jeux pc. Retropie est à la différence de Caiman un OS à part entière et chaque jeu doit être ajouté indépendamment par l’utilisateur sur le Rasberry/PC.


\section{RetroArch}
\label{\detokenize{opportunite:retroarch}}
\sphinxAtStartPar
\sphinxincludegraphics{{RetroArch}.jpg}

\sphinxAtStartPar
RetroArch est un frontend pour émulateurs, moteur de jeu et media players. Il embarque un très grand nombre d’émulateurs et a l’avantage d’être disponible sur un très grand nombre de plateformes.

\sphinxAtStartPar
RetroArch possède des fonctionnalités très utiles comme des Shaders intégré, le jeux en réseau, le calcul du temps entre les frames, et d’autres. C’est surement le Frontend pour émulateur le plus complet disponible.

\sphinxAtStartPar
L’interface de RetroArch est inspirée par le XMB de la Playstation 3 et de la PSP, ce qui le rend compatible avec une manette.

\sphinxAtStartPar
RetroArch est disponible sur:
\begin{itemize}
\item {} 
\sphinxAtStartPar
Windows

\item {} 
\sphinxAtStartPar
Linux

\item {} 
\sphinxAtStartPar
Raspberry

\item {} 
\sphinxAtStartPar
Android

\item {} 
\sphinxAtStartPar
Apple macOS (ARM/x64)

\item {} 
\sphinxAtStartPar
Apple macOS/OSX

\item {} 
\sphinxAtStartPar
IOS/Apple TV

\item {} 
\sphinxAtStartPar
Xbox Series / One

\item {} 
\sphinxAtStartPar
OpenDingux

\item {} 
\sphinxAtStartPar
PSvita

\item {} 
\sphinxAtStartPar
PSP

\item {} 
\sphinxAtStartPar
PS2

\item {} 
\sphinxAtStartPar
PS3

\item {} 
\sphinxAtStartPar
PS4

\item {} 
\sphinxAtStartPar
Switch

\item {} 
\sphinxAtStartPar
WII U

\item {} 
\sphinxAtStartPar
WII

\item {} 
\sphinxAtStartPar
Gamecube

\item {} 
\sphinxAtStartPar
3DS / 2DS

\end{itemize}


\section{RomStation}
\label{\detokenize{opportunite:romstation}}
\sphinxAtStartPar
\sphinxincludegraphics{{RomStation}.png}

\sphinxAtStartPar
RomStation est un frontend pour émulateur facilitant l’émulation de nombreuses consoles. RomStation a la différence de RetroPie ou RetroArch, l’utilisateur a la possibilité de  télécharger des jeux directement depuis l’application. L’application a une formule payante qui permet d’améliorer la vitesse des téléchargements des jeux. RomStation est disponible sur Windows (x64)(x86), macOS.


\section{Steam}
\label{\detokenize{opportunite:steam}}
\sphinxAtStartPar
\sphinxincludegraphics{{steam}.png}

\sphinxAtStartPar
Steam est une plateforme de distribution de jeux et d’applications en ligne. Steam est spécialisé dans les jeux vidéo contrairement aux autres projets que j’ai cité ci\sphinxhyphen{}dessus. C’est une application commerciale à but lucratif.

\sphinxAtStartPar
Steam n’est pas directement liée à l’utilisation d’émulateurs, mais j’ai décidé d’en parler car je m’inspire de son interface. Elle est créée spécifiquement pour une utilisation à la manette et la gestion que fais steam dès sauvegardes des utilisateurs.


\section{Conclusion de l’analyse de l’existant}
\label{\detokenize{opportunite:conclusion-de-lanalyse-de-l-existant}}
\sphinxAtStartPar
Il existe un grand nombre de frontend pour émulateur, ils permettent de simplifier l’utilisation pour l’utilisateur mais je n’en ai pas trouvé qui essaie d’atteindre le même but que moi. Le but de Caiman est de permettre la simplification de l’utilisation d’émulateurs et de permettre de synchroniser des sauvegardes tout en fournissant la possibilité de télécharger des jeux.

\sphinxAtStartPar
Steam représente un niveau de finition dans son interface BigPicture que j’aimerais atteindre. La synchronisation des sauvegardes sur steam est peut\sphinxhyphen{}être la plus performante, si l’on compare avec d’autres boutiques comme l’Epic Game Store ou Origin.


\chapter{Analyse fonctionnelle}
\label{\detokenize{fonctionnelleBDD:analyse-fonctionnelle}}\label{\detokenize{fonctionnelleBDD::doc}}

\section{Base de données}
\label{\detokenize{fonctionnelleBDD:base-de-donnees}}

\subsection{Schéma}
\label{\detokenize{fonctionnelleBDD:schema}}
\sphinxAtStartPar
Le schéma suivant représente la structure de la base de données. La base de données est commune au site web et à l’application Caiman. Les tables sont principalement utilisées pour stocker les informations des utilisateurs et des différents jeux.

\sphinxAtStartPar
\sphinxincludegraphics{{caiman_database}.png}


\subsection{Table categorie}
\label{\detokenize{fonctionnelleBDD:table-categorie}}
\sphinxAtStartPar
Cette table est utilisée pour stocker les différentes catégories auxquelles les jeux peuvent appartenir.


\subsection{Table console}
\label{\detokenize{fonctionnelleBDD:table-console}}
\sphinxAtStartPar
Cette table sert à stocker les différentes consoles prises en charge par l’application Caiman. Chaque console doit être reliée à un émulateur.


\subsection{Table emulator}
\label{\detokenize{fonctionnelleBDD:table-emulator}}
\sphinxAtStartPar
La table émulateur sert à lister les différents émulateurs disponible pour l’application, il est possible qu” un émulateur soit compatible avec plusieurs consoles.


\subsection{Table favoritegame}
\label{\detokenize{fonctionnelleBDD:table-favoritegame}}
\sphinxAtStartPar
Sert à lister les jeux favoris des utilisateurs.


\subsection{Table file}
\label{\detokenize{fonctionnelleBDD:table-file}}
\sphinxAtStartPar
La table file sert à lister le nom d’un fichier, cela peut être un fichier .iso d’un jeu, un fichier de sauvegarde ou un fichier de configuration utilisateur.


\subsection{Table filesave}
\label{\detokenize{fonctionnelleBDD:table-filesave}}
\sphinxAtStartPar
Sert  faire le lien entre un émulateur, un utilisateur, et un fichier de sauvegarde. Alors elle permet de connaître les sauvegardes pour un émulateur particulier d’un utilisateur.


\subsection{Table game}
\label{\detokenize{fonctionnelleBDD:table-game}}
\sphinxAtStartPar
Table qui liste les jeux disponibles. Chaque jeu a plusieurs informations: un nom, une description, une console. La console permet de à l’application de savoir quel émulateur doit être utilisé.


\subsection{Table gamehascategorie}
\label{\detokenize{fonctionnelleBDD:table-gamehascategorie}}
\sphinxAtStartPar
Sert à assigner des catégories aux jeux.


\subsection{Table rôle}
\label{\detokenize{fonctionnelleBDD:table-role}}
\sphinxAtStartPar
Sert à lister les différents rôles (utilisateur, administrateur), elle ne sert que pour le site web sachant que toute la partie administration est sur le site.


\subsection{Table timeingame}
\label{\detokenize{fonctionnelleBDD:table-timeingame}}
\sphinxAtStartPar
Sert à stocker le temps de jeu en minutes de chaque utilisateur. Le temps de jeu est spécifique à chaque jeu. Il est mis à jour directement depuis l’application Caiman.


\subsection{Table user}
\label{\detokenize{fonctionnelleBDD:table-user}}
\sphinxAtStartPar
Cette table sert à stocker les informations de compte de chaque utilisateur de Caiman. Le compte est commun au site web et à l’application. Le mot de passe de l’utilisateur est crypté avec les fonctions de sécurité de php.


\subsection{Table userconfigfile}
\label{\detokenize{fonctionnelleBDD:table-userconfigfile}}
\sphinxAtStartPar
Cette table sert à faire le lien entre un fichier de configuration et un utilisateur. Le fichier de configuration sert à connaître la configuration.


\chapter{Site web Caiman}
\label{\detokenize{fonctionnelleWeb:site-web-caiman}}\label{\detokenize{fonctionnelleWeb::doc}}

\section{Création de compte}
\label{\detokenize{fonctionnelleWeb:creation-de-compte}}
\sphinxAtStartPar
L’utilisateur du site a la possibilité de créer un compte qui sera commun au site et à l’application. La création de compte nécessite de renseigner son email, de donner un nom d’utilisateur ainsi que un mot de passe.






\section{connexion}
\label{\detokenize{fonctionnelleWeb:connexion}}
\sphinxAtStartPar
La connexion à son compte utilisateur permet de modifier nos informations de compte et d’ajouter ou de supprimer des jeux à la liste de favoris.

\sphinxAtStartPar
Si l’utilisateur oublie son mot de passe, il a la possibilité de le réinitialiser. L’utilisateur qui se décide de réinitialiser son mot de passe reçoit un mail contenant un lien de réinitialisation.






\section{modification des informations d’un utilisateur}
\label{\detokenize{fonctionnelleWeb:modification-des-informations-dun-utilisateur}}
\sphinxAtStartPar
Un utilisateur connecté a la possibilité de modifier ces informations:
\begin{itemize}
\item {} 
\sphinxAtStartPar
mot de passe

\item {} 
\sphinxAtStartPar
liste de jeux favoris

\item {} 
\sphinxAtStartPar
la visibilité de son profil pour les autres utilisateurs

\end{itemize}






\section{affichage des jeux}
\label{\detokenize{fonctionnelleWeb:affichage-des-jeux}}
\sphinxAtStartPar
Tous les utilisateurs ont la possibilité d’afficher la liste de jeux disponible. Il n’y a pas de restriction particulière.

\sphinxAtStartPar
Les informations disponibles pour chaque jeu sont les suivantes:
\begin{itemize}
\item {} 
\sphinxAtStartPar
Nom

\item {} 
\sphinxAtStartPar
Description

\item {} 
\sphinxAtStartPar
Catégories du jeu

\end{itemize}






\section{affichage d’un profil utilisateur}
\label{\detokenize{fonctionnelleWeb:affichage-dun-profil-utilisateur}}
\sphinxAtStartPar
Il est possible de consulter la page personnelle d’un utilisateur si celui\sphinxhyphen{}ci a rendu son compte publique. Les informations disponibles sont celle\sphinxhyphen{}ci:
\begin{itemize}
\item {} 
\sphinxAtStartPar
Nom d’utilisateur

\item {} 
\sphinxAtStartPar
Jeux favoris

\item {} 
\sphinxAtStartPar
Nombres d’heures de jeux sur chaque jeu

\end{itemize}










\section{ajout d’un jeu à la base de données / sur le Bunker pour le fichier .ISO}
\label{\detokenize{fonctionnelleWeb:ajout-dun-jeu-a-la-base-de-donnees-sur-le-bunker-pour-le-fichier-iso}}
\sphinxAtStartPar
L’ajout d’un jeu se fait grâce à un formulaire, plusieurs champs sont à renseigner:
\begin{itemize}
\item {} 
\sphinxAtStartPar
le nom du jeu

\item {} 
\sphinxAtStartPar
une description

\item {} 
\sphinxAtStartPar
une image

\item {} 
\sphinxAtStartPar
la console du jeu qui est uploadé

\item {} 
\sphinxAtStartPar
le nom que va porter le jeu sur le Bunker

\item {} 
\sphinxAtStartPar
le fichier de base du jeu

\end{itemize}

\sphinxAtStartPar
L’ajout à la base va créer deux entrées. Une dans la table Game et une autre dans la table file. Après avoir été ajouté depuis le site web, le jeu devient accessible depuis l’application Caiman et le site web pour de la consultation.






\section{modification d’un jeu}
\label{\detokenize{fonctionnelleWeb:modification-dun-jeu}}
\sphinxAtStartPar
La modification d’un jeu ne peut être faite que par un administrateur. Les modifications possibles sont les suivantes:
\begin{itemize}
\item {} 
\sphinxAtStartPar
nom

\item {} 
\sphinxAtStartPar
description

\item {} 
\sphinxAtStartPar
console

\item {} 
\sphinxAtStartPar
catégories

\end{itemize}






\section{Administration}
\label{\detokenize{fonctionnelleWeb:administration}}
\sphinxAtStartPar
Les administrateurs ont la possibilité de faire plusieurs choses. Donc, je vais les lister ici, il n’est pas nécessaire de les détailler.
\begin{itemize}
\item {} 
\sphinxAtStartPar
modifier un jeu

\item {} 
\sphinxAtStartPar
ajouter des catégories

\item {} 
\sphinxAtStartPar
ajouter ou supprimer des catégories à un jeu

\end{itemize}






\section{Téléchargement}
\label{\detokenize{fonctionnelleWeb:telechargement}}
\sphinxAtStartPar
L’un des intérêts du site est de pouvoir télécharger l’application Caiman. Le téléchargement de cette application nécessite d’être authentifié sur le site. Si un invité se rend sur la page de téléchargement sans être authentifié, une invitation lui sera faite de s’authentifier ou de créer un compte.










\section{Fonctionnalitées manquante}
\label{\detokenize{fonctionnelleWeb:fonctionnalitees-manquante}}
\sphinxAtStartPar
Malheureusement dû au temps imposé j’ai dû faire des choix organisationnel. J’ai donc décidé de me focaliser sur L’application Caiman au lieu du site internet ce qui explique que certaines fonctionnalités ne soient pas finalisées ou mises en place. Je vais lister les ce qui n’a pas été fini complètement ou abandonné.
\begin{itemize}
\item {} 
\sphinxAtStartPar
Modification complète des données des jeux

\item {} 
\sphinxAtStartPar
Modification d’une catégorie

\item {} 
\sphinxAtStartPar
Réinitialisation de mot de passe

\item {} 
\sphinxAtStartPar
Création de compte administrateur

\end{itemize}


\chapter{API Caiman}
\label{\detokenize{fonctionnelleAPI:api-caiman}}\label{\detokenize{fonctionnelleAPI::doc}}
\sphinxAtStartPar
L’API sert à pouvoir accéder à la base de données depuis l’application Caiman. Je vais détailler les endpoint et la structure de l’API.


\section{Structure de l’API}
\label{\detokenize{fonctionnelleAPI:structure-de-lapi}}
\sphinxAtStartPar
\sphinxincludegraphics{{schema-API}.png}

\sphinxAtStartPar
Pour expliquer la structure de l’API, je vais expliquer étape par étape comment un appel se passe.
\begin{enumerate}
\sphinxsetlistlabels{\arabic}{enumi}{enumii}{}{.}%
\item {} 
\sphinxAtStartPar
L’utilisateur envoie une requête à la page index.php de api.caiman.cfpt.info.

\item {} 
\sphinxAtStartPar
La requête est réceptionnée par index.php. L’url est ensuite traité par le .htaccess pour savoir où doit envoyer à quel controller.

\item {} 
\sphinxAtStartPar
Le contrôleur décide selon les informations reçues quelle méthode il doit exécuter.

\item {} 
\sphinxAtStartPar
Le DAO est appelé et va rechercher dans la base de données les données demandé

\item {} 
\sphinxAtStartPar
Le DAO créé la réponse grâce au model.

\item {} 
\sphinxAtStartPar
La réponse est envoyée à l’utilisateur par l’intermédiaire de la page index.php

\end{enumerate}


\section{Categories}
\label{\detokenize{fonctionnelleAPI:categories}}

\section{GET}
\label{\detokenize{fonctionnelleAPI:get}}
\sphinxAtStartPar
Permet de recevoir la liste des catégories disponibles.

\sphinxAtStartPar
Les informations reçues sont les suivantes:
\begin{itemize}
\item {} 
\sphinxAtStartPar
id

\item {} 
\sphinxAtStartPar
nom

\end{itemize}


\section{Games}
\label{\detokenize{fonctionnelleAPI:games}}

\section{GET}
\label{\detokenize{fonctionnelleAPI:id1}}
\sphinxAtStartPar
Retourne la liste des jeux.

\sphinxAtStartPar
Les informations reçues sont les suivantes:
\begin{itemize}
\item {} 
\sphinxAtStartPar
id

\item {} 
\sphinxAtStartPar
description

\item {} 
\sphinxAtStartPar
nom de l’image

\item {} 
\sphinxAtStartPar
id de la console

\item {} 
\sphinxAtStartPar
id du fichier du jeu

\end{itemize}


\section{GET(?byName)}
\label{\detokenize{fonctionnelleAPI:get-byname}}
\sphinxAtStartPar
Retourne la liste des jeux qui dans le nom contient ce que l’utilisateur a demandé.

\sphinxAtStartPar
Les informations reçues sont les suivantes:
\begin{itemize}
\item {} 
\sphinxAtStartPar
id

\item {} 
\sphinxAtStartPar
description

\item {} 
\sphinxAtStartPar
nom de l’image

\item {} 
\sphinxAtStartPar
id de la console

\item {} 
\sphinxAtStartPar
id du fichier du jeu

\end{itemize}


\section{GET(?byCategory)}
\label{\detokenize{fonctionnelleAPI:get-bycategory}}
\sphinxAtStartPar
Retourne la liste des jeux qui appartiennent à une catégorie.

\sphinxAtStartPar
Il faut spécifier l’id de la catégorie qui est demandée.

\sphinxAtStartPar
Les informations reçues sont les suivantes:
\begin{itemize}
\item {} 
\sphinxAtStartPar
id

\item {} 
\sphinxAtStartPar
description

\item {} 
\sphinxAtStartPar
nom de l’image

\item {} 
\sphinxAtStartPar
id de la console

\item {} 
\sphinxAtStartPar
id du fichier du jeu

\end{itemize}


\section{GET(?byFavoriteUser)}
\label{\detokenize{fonctionnelleAPI:get-byfavoriteuser}}
\sphinxAtStartPar
Retourne la liste des jeux favoris d’un utilisateur.

\sphinxAtStartPar
Il faut spécifier l’id de l’utilisateur.

\sphinxAtStartPar
Les informations reçues sont les suivantes:
\begin{itemize}
\item {} 
\sphinxAtStartPar
id

\item {} 
\sphinxAtStartPar
description

\item {} 
\sphinxAtStartPar
nom de l’image

\item {} 
\sphinxAtStartPar
id de la console

\item {} 
\sphinxAtStartPar
id du fichier du jeu

\end{itemize}


\section{GET(?byUserTime)}
\label{\detokenize{fonctionnelleAPI:get-byusertime}}
\sphinxAtStartPar
Retourne la liste des jeux auxquels un joueur a joué.

\sphinxAtStartPar
Il faut spécifier l’id de l’utilisateur.

\sphinxAtStartPar
Les informations reçues sont les suivantes:
\begin{itemize}
\item {} 
\sphinxAtStartPar
id

\item {} 
\sphinxAtStartPar
description

\item {} 
\sphinxAtStartPar
nom de l’image

\item {} 
\sphinxAtStartPar
id de la console

\item {} 
\sphinxAtStartPar
id du fichier du jeu

\item {} 
\sphinxAtStartPar
nombre de minutes en jeu

\end{itemize}


\section{GET(?gameFileName)}
\label{\detokenize{fonctionnelleAPI:get-gamefilename}}
\sphinxAtStartPar
Retourne le nom du fichier d’un jeu.

\sphinxAtStartPar
Les informations reçues sont les suivantes:
\begin{itemize}
\item {} 
\sphinxAtStartPar
filename

\end{itemize}


\section{GET(?gameConsole)}
\label{\detokenize{fonctionnelleAPI:get-gameconsole}}
\sphinxAtStartPar
Retourne la console d’un jeu.

\sphinxAtStartPar
Les informations reçues sont les suivantes:
\begin{itemize}
\item {} 
\sphinxAtStartPar
name

\item {} 
\sphinxAtStartPar
folderName

\end{itemize}


\section{GET(?idGame\&apiKey)}
\label{\detokenize{fonctionnelleAPI:get-idgame-apikey}}
\sphinxAtStartPar
Retourne le fichier d’un jeu.

\sphinxAtStartPar
Les informations reçues sont les suivantes:
\begin{itemize}
\item {} 
\sphinxAtStartPar
fichier.iso

\end{itemize}


\section{GET(?idGameTime\&idUser)}
\label{\detokenize{fonctionnelleAPI:get-idgametime-iduser}}
\sphinxAtStartPar
Retourne le temps de jeu sur un jeu.

\sphinxAtStartPar
Les informations reçues sont les suivantes:
\begin{itemize}
\item {} 
\sphinxAtStartPar
minutes

\end{itemize}


\section{GET(?idEmulator\&idUser\&apiKey)}
\label{\detokenize{fonctionnelleAPI:get-idemulator-iduser-apikey}}
\sphinxAtStartPar
Retourne un fichier zip contenant les sauvegardes d’un joueur pour un émulateur particulier

\sphinxAtStartPar
Les informations reçues sont les suivantes:
\begin{itemize}
\item {} 
\sphinxAtStartPar
fichier.zip

\end{itemize}


\section{POST(?idEmulator\&idUser\&apiKey)}
\label{\detokenize{fonctionnelleAPI:post-idemulator-iduser-apikey}}
\sphinxAtStartPar
Upload un fichier zip contenant les sauvegardes d’un joueur pour un émulateur particulier


\section{POST(idGameAdd\&idUser)}
\label{\detokenize{fonctionnelleAPI:post-idgameadd-iduser}}
\sphinxAtStartPar
Ajouter un jeu en favoris pour un utilisateur particulier


\section{POST(idGameRemove\&idUser)}
\label{\detokenize{fonctionnelleAPI:post-idgameremove-iduser}}
\sphinxAtStartPar
Supprime un jeu en favoris pour un utilisateur particulier


\section{POST(idGameCheck\&idUser)}
\label{\detokenize{fonctionnelleAPI:post-idgamecheck-iduser}}
\sphinxAtStartPar
Vérifie si un jeu est déjà en favoris et retourne un booléen


\section{POST(idGameTimeAdd\&idUser)}
\label{\detokenize{fonctionnelleAPI:post-idgametimeadd-iduser}}
\sphinxAtStartPar
Ajouter une minute de jeu à un jeu particulier pour un utilisateur


\section{Users}
\label{\detokenize{fonctionnelleAPI:users}}

\section{GET(sans paramétres)}
\label{\detokenize{fonctionnelleAPI:get-sans-parametres}}
\sphinxAtStartPar
Retourne la liste des utilisateurs.

\sphinxAtStartPar
Les informations reçues sont les suivantes:
\begin{itemize}
\item {} 
\sphinxAtStartPar
id

\item {} 
\sphinxAtStartPar
username

\end{itemize}


\section{POST(avec apitoken)}
\label{\detokenize{fonctionnelleAPI:post-avec-apitoken}}
\sphinxAtStartPar
Retourne un utilisateur en particulier

\sphinxAtStartPar
Les informations reçues sont les suivantes:
\begin{itemize}
\item {} 
\sphinxAtStartPar
id

\item {} 
\sphinxAtStartPar
username

\item {} 
\sphinxAtStartPar
password

\item {} 
\sphinxAtStartPar
salt

\item {} 
\sphinxAtStartPar
apitoken

\item {} 
\sphinxAtStartPar
caimanToken

\item {} 
\sphinxAtStartPar
email

\item {} 
\sphinxAtStartPar
idRole

\end{itemize}


\section{User/connection}
\label{\detokenize{fonctionnelleAPI:user-connection}}

\section{POST(?username, password)}
\label{\detokenize{fonctionnelleAPI:post-username-password}}
\sphinxAtStartPar
Permet de vérifier les informations de connexion d’un utilisateur

\sphinxAtStartPar
Les informations reçues sont les suivantes:
\begin{itemize}
\item {} 
\sphinxAtStartPar
id

\item {} 
\sphinxAtStartPar
username

\item {} 
\sphinxAtStartPar
password

\item {} 
\sphinxAtStartPar
salt

\item {} 
\sphinxAtStartPar
apitoken

\item {} 
\sphinxAtStartPar
caimanToken

\item {} 
\sphinxAtStartPar
email

\item {} 
\sphinxAtStartPar
idRole

\end{itemize}


\section{POST(caimanToken)}
\label{\detokenize{fonctionnelleAPI:post-caimantoken}}
\sphinxAtStartPar
Permet de recevoir les informations d’un utilisateur grâce à un token généré à chaque connexion.

\sphinxAtStartPar
Les informations reçues sont les suivantes:
\begin{itemize}
\item {} 
\sphinxAtStartPar
id

\item {} 
\sphinxAtStartPar
username

\item {} 
\sphinxAtStartPar
password

\item {} 
\sphinxAtStartPar
salt

\item {} 
\sphinxAtStartPar
apitoken

\item {} 
\sphinxAtStartPar
caimanToken

\item {} 
\sphinxAtStartPar
email

\item {} 
\sphinxAtStartPar
idRole

\end{itemize}


\chapter{Application  Caiman C\#}
\label{\detokenize{fonctionnelleC_:application-caiman-c}}\label{\detokenize{fonctionnelleC_::doc}}

\section{Connexion}
\label{\detokenize{fonctionnelleC_:connexion}}
\sphinxAtStartPar
L’utilisateur de Caiman doit obligatoirement être connecté pour pouvoir utiliser caiman. La connexion se fait avec le nom d’utilisateur et le mot de passe. Une fois la connexion valable, elle est maintenue tant que l’utilisateur ne se déconnecte pas ou tant que l’utilisateur ne s’est pas connecté sur un autre ordinateur.

\sphinxAtStartPar
\sphinxincludegraphics{{login_caiman}.png}


\section{Inscription}
\label{\detokenize{fonctionnelleC_:inscription}}
\sphinxAtStartPar
L’inscription n’est pas disponible sur caiman mais un bouton est disponible pour être redirigé sur le site web de Caiman.


\section{Téléchargement de jeu}
\label{\detokenize{fonctionnelleC_:telechargement-de-jeu}}
\sphinxAtStartPar
L’utilisateur a la possibilité de télécharger des jeux. Les jeux disponibles dans l’application sont ajoutés depuis le site internet de Caiman. Les téléchargements des jeux se font les un après les autres.

\sphinxAtStartPar
\sphinxincludegraphics{{Caiman_non_downloaded_game}.png}

\sphinxAtStartPar
\sphinxincludegraphics{{Caiman_download_queu}.png}


\section{Lancement d’un jeu}
\label{\detokenize{fonctionnelleC_:lancement-dun-jeu}}
\sphinxAtStartPar
Caiman inclut deux émulateurs Dolphin un PCSX2. Ces deux émulateurs permettent d’exécuter des jeux de Gamecube et Wii pour Dolphin et de Playstation 2 pour PCSX2. Pour lancer un jeu, il suffit de le télécharger, puis de cliquer sur Play. Il n’est donc pas nécessaire de lancer soit même un émulateur.

\sphinxAtStartPar
\sphinxincludegraphics{{caiman_downloaded_game}.png}


\section{Synchronisation des sauvegarde entre le pc client et le Bunker}
\label{\detokenize{fonctionnelleC_:synchronisation-des-sauvegarde-entre-le-pc-client-et-le-bunker}}
\sphinxAtStartPar
Les sauvegardes de l’utilisateur sont synchronisées entre les différents pc qu’il utilise. La synchronisation se fait à la connexion, les sauvegardes sont stockées sur les serveurs de Caiman. La copie des sauvegardes des utilisateurs est envoyée automatiquement donc l’utilisateur n’a pas de manipulation à faire. L’envoie se passe dès que l’utilisateur sauvegarde, cela permet d’éviter de perte de sauvegardes si un problème se produit durant le moment où le joueur est en train de jouer.


\section{Modification de la configuration utilisateur}
\label{\detokenize{fonctionnelleC_:modification-de-la-configuration-utilisateur}}
\sphinxAtStartPar
L’utilisateur a la possibilité de modifier plusieurs paramètres graphiques. Il a la possibilité de choisir en trois mode de configuration global:
\begin{itemize}
\item {} 
\sphinxAtStartPar
Original

\item {} 
\sphinxAtStartPar
1080p

\item {} 
\sphinxAtStartPar
4K

\end{itemize}

\sphinxAtStartPar
C’est différent mode modifie l’antialiasing et la définition native de l’émulateur qui va être utilisé.

\sphinxAtStartPar
L’utilisateur a aussi la possibilité de choisir entre lancer le jeu plein écran ou non et il peut choisir si le jeu doit être en 16/9 ou en 4/3.

\sphinxAtStartPar
\sphinxincludegraphics{{caiman_configuration}.png}


\section{Application de la configuration utilisateur}
\label{\detokenize{fonctionnelleC_:application-de-la-configuration-utilisateur}}
\sphinxAtStartPar
La configuration est appliquée avant de lancer un jeu, c’est à dire que la configuration n’est pas appliquée si l’utilisateur est déjà en jeu.


\section{Ajout/suppresion de jeu en favoris}
\label{\detokenize{fonctionnelleC_:ajout-suppresion-de-jeu-en-favoris}}
\sphinxAtStartPar
L’utilisateur a la possibilité de modifier ses jeux favoris directement depuis caiman.

\sphinxAtStartPar
\sphinxincludegraphics{{caiman_add_favorite}.png}

\sphinxAtStartPar
Suppression de jeu des favoris

\sphinxAtStartPar
\sphinxincludegraphics{{caiman_remove_favorite}.png}


\section{Affichage de jeux par catégories}
\label{\detokenize{fonctionnelleC_:affichage-de-jeux-par-categories}}
\sphinxAtStartPar
L’utilisateur a la possibilité d’afficher les jeux par catégories. Les catégories sont créées manuellement par un administrateur. Il y a certaines catégories « spéciales », par exemple les jeux favoris et les jeux uniques téléchargés par chaque utilisateur.

\sphinxAtStartPar
\sphinxincludegraphics{{caiman_categories}.png}


\section{Nombre d’heures de jeu}
\label{\detokenize{fonctionnelleC_:nombre-dheures-de-jeu}}
\sphinxAtStartPar
Le nombre de minutes de jeu d’un utilisateur est mis à jour à chaque minute de jeu. le nombre de minutes de jeu est visible sur la page de chaque jeu si l’utilisateur n’a pas encore joué.

\sphinxAtStartPar
\sphinxincludegraphics{{caiman_time_played}.png}


\section{Gestion des manettes}
\label{\detokenize{fonctionnelleC_:gestion-des-manettes}}
\sphinxAtStartPar
Pour Caiman, les manettes supportées sont les manettes pour xbox qui fonctionne avec Xinput. Il est possible d’utiliser d’autres manettes pour cela, il faut passer par un programme qui va convertir les inputs de la manette non compatible en input de manette xbox.

\sphinxAtStartPar
Pour gérer les déplacements, j’utilise les touches “haut, bas, gauche,droite”, le stick gauche, la validation se fait avec la touche “A” et le retour arrière avec la touche “B”.


\chapter{Interface utilisateur utilisable à la manette}
\label{\detokenize{fonctionnelleInterface:interface-utilisateur-utilisable-a-la-manette}}\label{\detokenize{fonctionnelleInterface::doc}}

\section{Réception des input des manettes connecté}
\label{\detokenize{fonctionnelleInterface:reception-des-input-des-manettes-connecte}}
\sphinxAtStartPar
L’utilisateur de Caiman a la possibilité de pouvoir utiliser l’application au clavier souris mais aussi à la manette. Pour ce faire, j’ai utilisé le paquet nuGet “SharpDX.XInput”. Ce paquet me permet de connaître les manettes connectées au pc ainsi que les touches appuyées par l’utilisateur.

\sphinxAtStartPar
La seule manette qui peut se déplacer dans l’application est la manette 1. Pour connaître les boutons de la manette, j’utilise la fonction getInput(). Cette fonction me permet de connaître les touches qui sont pressées à un instant T.  Je vais chercher les inputs toutes les 2ms pour être sûr de ne pas louper d’inputs.

\sphinxAtStartPar
Les inputs sont ensuite traités par l’interface de l’application qui décide quoi en faire selon le contexte.


\section{Transformation des input de la manette en événement}
\label{\detokenize{fonctionnelleInterface:transformation-des-input-de-la-manette-en-evenement}}
\sphinxAtStartPar
Les inputs de la manette sont analysés par la form principale de Caiman. Selon la ou les touches qui sont pressées, l’application exécute des actions différentes. Par exemple, quand la touche “A” est pressée, alors le programme envoie la touche ENTER à l’application ce qui me permet de cliquer sur des boutons.

\sphinxAtStartPar
Pour gérer les événements, je fais un test pour savoir si l’utilisateur utilise l’application ou non . Si l’application n’est pas focus par l’utilisateur, seul une partie des actions sont possibles pour éviter que des actions inattendues puissent arriver alors que l’application n’est plus visible.


\section{Structure de l’affichage}
\label{\detokenize{fonctionnelleInterface:structure-de-laffichage}}
\sphinxAtStartPar
L’affichage est constitué d’un “XboxMainForm”. Il sert à contenir tous les autres panels, il est aussi chargé de la gestion des inputs de la manette de l’utilisateur 1.

\sphinxAtStartPar
Un “XboxMainForm” contient une liste de XboxUserController qui elles contiennent différentes choses comme des boutons des images ou des labels.

\sphinxAtStartPar
Le XboxMainForm est aussi responsable de la gestion des demandes de l’utilisateur, par exemple si l’utilisateur veut afficher l’accueil de Caiman, il va lui passer un objet contenant sa demande. Il va donc traiter les demandes et afficher les panels selon les besoins de l’utilisateur.


\section{Déplacement dans un panel de l’application}
\label{\detokenize{fonctionnelleInterface:deplacement-dans-un-panel-de-lapplication}}
\sphinxAtStartPar
L’application est conçue avec des “panels”, c’est\sphinxhyphen{}à\sphinxhyphen{}dire une liste de listes de controls. Cette liste de controls est propre à chaque panel. Les panels contiennent aussi une variable position\_x et position\_y qui permettent de connaître le control actuellement sélectionné par l’utilisateur. Quand l’utilisateur décide de se déplacer, il demande au panel de modifier ses variables x et y. Avant de valider ce changement, le panel regarde si le déplacement demandé par l’utilisateur est possible ou non.

\sphinxAtStartPar
Il existe 3 possibilités:
\begin{enumerate}
\sphinxsetlistlabels{\arabic}{enumi}{enumii}{}{.}%
\item {} 
\sphinxAtStartPar
Le déplacement est possible, alors la position sur l’axe x,y est modifiée.

\item {} 
\sphinxAtStartPar
Le déplacement est impossible car il n’y a rien à l’emplacement demandé. Dans ce cas, le panel va décider de bouger le curseur sur un des emplacements possibles.

\item {} 
\sphinxAtStartPar
L’utilisateur est à la fin du panel et “sors du panel” dans ce cas il va se diriger dans un autre panel s’il y en a un dans la direction demandée.

\end{enumerate}


\section{Déplacement de panel en panel}
\label{\detokenize{fonctionnelleInterface:deplacement-de-panel-en-panel}}
\sphinxAtStartPar
Chaque “panel” possède un pointeur sur le panel du haut, du bas, de droite et de gauche.

\sphinxAtStartPar
Ces pointers ne sont pas forcément utilisés, ils ont le droit d’être nul.

\sphinxAtStartPar
Si on prend l’exemple suivant:

\sphinxAtStartPar
\sphinxincludegraphics{{exemple_3_panel}.jpeg}

\sphinxAtStartPar
Nous avons 3 panels différents qui contiennent chacun plusieurs controls.

\sphinxAtStartPar
Le panel 1 possède donc deux pointeurs différents, un sur le panel 2 et un autre sur le panel 3.

\sphinxAtStartPar
trop confus

\sphinxAtStartPar
Si l’utilisateur se trouve en bas du panel 1 et qu’il décide de se déplacer encore plus  bas, il ne pourra pas car aucune case n’est disponible dans ce panel. C’est pourquoi une vérification  sera faite pour savoir si un panel est indiqué comme le panel “down”, si tel est le cas le focus va changer de panel.

\sphinxAtStartPar
Un autre cas possible est que l’utilisateur va peut\sphinxhyphen{}être décider de retourner sur le panel 1. La contrainte est de savoir où doit pointer le panel haut du panel 3. Actuellement, un seul panel peut être défini par côté, mais la solution est de créer de petits panels pour éviter que ces situations arrivent.


\chapter{Logbook}
\label{\detokenize{logbook:logbook}}\label{\detokenize{logbook::doc}}

\section{19.04.2021}
\label{\detokenize{logbook:id1}}

\subsection{8h05}
\label{\detokenize{logbook:h05}}
\sphinxAtStartPar
Entretiens avec M. Garcia


\subsection{9h05}
\label{\detokenize{logbook:id2}}
\sphinxAtStartPar
Copie de mon disque pour Gawen


\subsection{9h20}
\label{\detokenize{logbook:h20}}
\sphinxAtStartPar
Création du git


\subsection{9h30}
\label{\detokenize{logbook:h30}}
\sphinxAtStartPar
je réfléchi à ce par quoi je vais commencer j’hésite entre commencé entre le site web ou les téléchargement


\subsection{9h40}
\label{\detokenize{logbook:h40}}
\sphinxAtStartPar
Modélisation de la BDD

\sphinxAtStartPar
\sphinxincludegraphics{{bdd_tableau}.jpg}


\subsection{10h30}
\label{\detokenize{logbook:id3}}
\sphinxAtStartPar
Installation de laragon


\subsection{10h40}
\label{\detokenize{logbook:id4}}
\sphinxAtStartPar
Création de la base de données


\subsection{12h40}
\label{\detokenize{logbook:id5}}
\sphinxAtStartPar
Création de la structure du site web


\subsection{14h35}
\label{\detokenize{logbook:h35}}
\sphinxAtStartPar
Création des différente pages et mise en place de bootstrap


\subsection{résumé}
\label{\detokenize{logbook:resume}}
\sphinxAtStartPar
j’ai créer la base de donnée et le site web


\bigskip\hrule\bigskip



\section{20.04.2021}
\label{\detokenize{logbook:id6}}

\subsection{8h05}
\label{\detokenize{logbook:id7}}
\sphinxAtStartPar
je continue à créer le site web je crée les formulaires pour la connexion


\subsection{15h20}
\label{\detokenize{logbook:id8}}
\sphinxAtStartPar
J’ai impliqué la création de compte et la connexion, le mail doit être unique et le username aussi.

\sphinxAtStartPar
j’ai selon les indication de M. Schmid utilisé les fonctions

\sphinxAtStartPar
password\_hash et password\_verify de php.


\subsection{15h30}
\label{\detokenize{logbook:id9}}
\sphinxAtStartPar
j’ajoute des donnée a la main dans la bdd


\subsection{15h56}
\label{\detokenize{logbook:h56}}
\sphinxAtStartPar
aide de M.Schmid pour du sql


\bigskip\hrule\bigskip



\section{21.04.2021}
\label{\detokenize{logbook:id10}}

\subsection{8h10}
\label{\detokenize{logbook:h10}}
\sphinxAtStartPar
Modification de la structure du site et ajout de l’update de mots de passe


\subsection{13h00}
\label{\detokenize{logbook:h00}}
\sphinxAtStartPar
l’affichage des jeux est disponible ainsi qu’une recherche sur les jeux grâce à leurs noms.


\subsection{13h20}
\label{\detokenize{logbook:id11}}
\sphinxAtStartPar
Ajout de champs dans la table game
\begin{itemize}
\item {} 
\sphinxAtStartPar
description

\item {} 
\sphinxAtStartPar
imageName

\end{itemize}


\subsection{15h00}
\label{\detokenize{logbook:id12}}
\sphinxAtStartPar
l’affichage de la recherche et du détail d’un jeu fonctionne mais n’est pas beau.


\bigskip\hrule\bigskip



\section{22.04.2021}
\label{\detokenize{logbook:id13}}

\subsection{8h05}
\label{\detokenize{logbook:id14}}
\sphinxAtStartPar
Ajout de l’affichage des catégories de chaque jeu


\subsection{9h00}
\label{\detokenize{logbook:id15}}
\sphinxAtStartPar
Affichage des jeux qui appartiennent à une catégorie.


\subsection{10h20}
\label{\detokenize{logbook:id16}}
\sphinxAtStartPar
modification de l’interface de recherche


\subsection{notes personnelles}
\label{\detokenize{logbook:notes-personnelles}}\begin{itemize}
\item {} 
\sphinxAtStartPar
je dois ajouter une table pour savoir le nombre d’heure de jeu de chaque utilisateurs

\item {} 
\sphinxAtStartPar
je dois ajouter une gestion des messages d’erreurs.

\item {} 
\sphinxAtStartPar
je peux ajouter une photo de profil

\end{itemize}


\subsection{11h0}
\label{\detokenize{logbook:h0}}
\sphinxAtStartPar
modification de l’interface de connexion et de d’inscription


\subsection{11h50}
\label{\detokenize{logbook:h50}}
\sphinxAtStartPar
suppression de la page de création de compte


\subsection{12h40}
\label{\detokenize{logbook:id17}}
\sphinxAtStartPar
Ajout d’un jeux en favoris


\subsection{15h00}
\label{\detokenize{logbook:id18}}
\sphinxAtStartPar
suppression d’un jeu en favoris


\subsection{a faire demain}
\label{\detokenize{logbook:a-faire-demain}}\begin{itemize}
\item {} 
\sphinxAtStartPar
l’affichage des card dans le dashboard n’est pas bon

\end{itemize}


\bigskip\hrule\bigskip



\section{23.04.2021}
\label{\detokenize{logbook:id19}}

\subsection{8h05}
\label{\detokenize{logbook:id20}}
\sphinxAtStartPar
modification de la l’interface du Dashboard


\subsection{8h40}
\label{\detokenize{logbook:id21}}
\sphinxAtStartPar
Ajout d’un champ dans la table utilisateur pour spécifier si l’utilisateur est privé ou non
si l’utilisateur n’est pas privé tout le monde va pouvoir voir son profils


\subsection{9h30}
\label{\detokenize{logbook:id22}}
\sphinxAtStartPar
la modification du paramètre pour savoir si le compte est privé ou non


\subsection{12h20}
\label{\detokenize{logbook:id23}}
\sphinxAtStartPar
test de Git Hook


\subsection{13h30}
\label{\detokenize{logbook:id24}}
\sphinxAtStartPar
modification de l’interface theme blanc \sphinxhyphen{}> dark


\subsection{15h20}
\label{\detokenize{logbook:id25}}
\sphinxAtStartPar
Création de la page de téléchargement et mise en place de du téléchargement de Caiman depuis le site


\subsection{notes pour la prochaine fois}
\label{\detokenize{logbook:notes-pour-la-prochaine-fois}}\begin{itemize}
\item {} 
\sphinxAtStartPar
Je dois créer la partie Administrateur du site

\item {} 
\sphinxAtStartPar
Je dois créer un une fonctionnalité qui me permet de gérer les messages d’erreurs

\item {} 
\sphinxAtStartPar
Je dois sécuriser l’accès au pages

\item {} 
\sphinxAtStartPar
je dois sécuriser les différents formulaires

\item {} 
\sphinxAtStartPar
je dois je dois me renseigner comment uploader des gros fichier depuis un poste clients

\item {} 
\sphinxAtStartPar
je dois changer de navbar

\end{itemize}


\bigskip\hrule\bigskip



\section{26.04.2021}
\label{\detokenize{logbook:id26}}

\subsection{8h05}
\label{\detokenize{logbook:id27}}
\sphinxAtStartPar
notes personnelles:
\begin{itemize}
\item {} 
\sphinxAtStartPar
je dois ajouter la possibilité d’afficher la page d’un utilisateur

\item {} 
\sphinxAtStartPar
je dois corriger mon script d’export de base de données

\end{itemize}


\subsection{8h10}
\label{\detokenize{logbook:id28}}
\sphinxAtStartPar
Création de la page dédiée aux administrateurs.


\subsection{8h30}
\label{\detokenize{logbook:id29}}
\sphinxAtStartPar
Ajout de catégorie


\subsection{9h00}
\label{\detokenize{logbook:id30}}
\sphinxAtStartPar
Ajout de jeu


\subsection{notes personnels}
\label{\detokenize{logbook:notes-personnels}}
\sphinxAtStartPar
j’ai regardé plusieurs méthodes pour envoyer un fichier depuis un formulaire en php. Pour l’instant j’utilise les fonctions de base de php et elle fonctionne donc je vais faire des tests une fois le site uploadé sur le Bunker.


\subsection{11h00}
\label{\detokenize{logbook:id31}}
\sphinxAtStartPar
le fichier .iso est uploadé avec le bon nom mais pas encore dans la base de données


\subsection{15h00}
\label{\detokenize{logbook:id32}}
\sphinxAtStartPar
Le jeu est bien ajouté avec le bon nom ainsi que la bonne image.


\subsection{15h45}
\label{\detokenize{logbook:h45}}
\sphinxAtStartPar
il est maintenant possible de mettre à jour le nom, la description ou la console d’un jeu.


\subsection{15h50}
\label{\detokenize{logbook:id33}}
\sphinxAtStartPar
modification de la structure du git


\bigskip\hrule\bigskip



\section{27.04.2021}
\label{\detokenize{logbook:id34}}

\subsection{8h05}
\label{\detokenize{logbook:id35}}
\sphinxAtStartPar
ajout/ suppression de catégories a un jeu


\subsection{9h20}
\label{\detokenize{logbook:id36}}
\sphinxAtStartPar
modification mineur de l’interface


\subsection{10h05}
\label{\detokenize{logbook:id37}}
\sphinxAtStartPar
recherche d’un profil utilisateur


\subsection{12h15}
\label{\detokenize{logbook:h15}}
\sphinxAtStartPar
la recherche et l’affichage d’un profil utilisateur est fonctionnel


\subsection{13h00}
\label{\detokenize{logbook:id38}}
\sphinxAtStartPar
modification de l’interface pour que les jeux s’affiche correctement


\subsection{13h30}
\label{\detokenize{logbook:id39}}
\sphinxAtStartPar
Le site est fonctionnel mais il manque des détail comme les message d’erreur et les droit sur les pages


\subsection{notes personnelles}
\label{\detokenize{logbook:id40}}
\sphinxAtStartPar
pour finir le site il me reste les choses suivante a faire:
\begin{itemize}
\item {} 
\sphinxAtStartPar
Sécuriser les pages

\item {} 
\sphinxAtStartPar
afficher des messages d’erreur

\item {} 
\sphinxAtStartPar
suppression de catégories

\item {} 
\sphinxAtStartPar
suppression de jeu

\item {} 
\sphinxAtStartPar
mot de passe oublié

\item {} 
\sphinxAtStartPar
commenter mon code

\item {} 
\sphinxAtStartPar
pagination pour les recherches

\end{itemize}


\subsection{13h40}
\label{\detokenize{logbook:id41}}
\sphinxAtStartPar
documentation


\bigskip\hrule\bigskip



\section{28.04.2021}
\label{\detokenize{logbook:id42}}

\subsection{8h05}
\label{\detokenize{logbook:id43}}
\sphinxAtStartPar
documentation


\subsection{10h40}
\label{\detokenize{logbook:id44}}
\sphinxAtStartPar
correction d’un bogue sur le nombre d’heure de jeu et riage des jeux par heure de jeu


\subsection{11h00}
\label{\detokenize{logbook:id45}}
\sphinxAtStartPar
gestion des droit d’accès au page


\subsection{13h10}
\label{\detokenize{logbook:id46}}
\sphinxAtStartPar
modification de la navbar


\subsection{14h00}
\label{\detokenize{logbook:id47}}
\sphinxAtStartPar
gestion des erreurs de du login


\bigskip\hrule\bigskip



\section{29.04.2021}
\label{\detokenize{logbook:id48}}

\subsection{8h30}
\label{\detokenize{logbook:id49}}
\sphinxAtStartPar
Documentation


\subsection{10h20}
\label{\detokenize{logbook:id50}}
\sphinxAtStartPar
Mr. Garcia m’a aidé à mettre en place ma documentation doxygen


\subsection{11h00}
\label{\detokenize{logbook:id51}}
\sphinxAtStartPar
documentation


\bigskip\hrule\bigskip



\section{30.04.2021}
\label{\detokenize{logbook:id52}}

\subsection{8h05}
\label{\detokenize{logbook:id53}}
\sphinxAtStartPar
Correction du logbook


\subsection{8h20}
\label{\detokenize{logbook:id54}}
\sphinxAtStartPar
supression de code inutile sur le site et mis a jour de la doccumentation


\subsection{9h30}
\label{\detokenize{logbook:id55}}
\sphinxAtStartPar
ajout de contrainte dans la base de données


\subsection{10h45}
\label{\detokenize{logbook:id56}}
\sphinxAtStartPar
Création du planning effectif


\subsection{13h00}
\label{\detokenize{logbook:id57}}
\sphinxAtStartPar
Configuration de debian

\sphinxAtStartPar
user tfp : FTPdiplomant
password ftp : SuperCfpt@

\sphinxAtStartPar
J’ai pus installer apache, php, et mysql mais je n’arrive pas à me connecter en ftp. A la connexion je bloque sur l’erreur : “Impossible de récupérer le contenu du dossier”. Malgré l’aide de Mr.Schmidt je n’ai toujours pas réussi.


\subsection{15h20}
\label{\detokenize{logbook:id58}}
\sphinxAtStartPar
Je n’arrive toujours pas a me connecter au ftp, je ne sais pas si le probléme viens de ma configuration ou du firewall


\section{notes pour le premier rendu}
\label{\detokenize{logbook:notes-pour-le-premier-rendu}}
\sphinxAtStartPar
J’ai durant ces deux premières semaines, créé le site internet de Caiman. Le site n’est pas fini à 100\% mais les fonctionnalités de base sont toutes implémentées. Les fonctionnalités actuelles permettent de faire toutes les choses nécessaires au fonctionnement de l’application. J’ai pris plus de temps que prévu à réaliser le site mais durant la création de mon planning prévisionnel j’ai fait des erreurs, j’ai par exemple oublié de planifier la création des fonctionnalités d’administration ( Ajout de jeux, ajout de catégories, assignation de catégories à un jeu, upload de jeux, etc ).

\sphinxAtStartPar
La documentation du site n’est pas forcément touffue mais le site n’est pas particulièrement complexe, il m’a pris du temps dû au nombre de tables à gérer. Il reste à mettre en place la récupération de mot de passe mais j’ai décidé de passer à l’interface graphique de l’application C\# dès la semaine prochaine. Étant donné que le projet est la partie la plus importante de l’application, je ne vais pas configurer la récupération de mot de passe maintenant.


\bigskip\hrule\bigskip



\section{03.05.2021}
\label{\detokenize{logbook:id59}}

\subsection{8h10}
\label{\detokenize{logbook:id60}}
\sphinxAtStartPar
Réflexion sur l’interface graphique et création du projet


\subsection{8h15}
\label{\detokenize{logbook:id61}}
\sphinxAtStartPar
Importation de la classe XboxController.cs que j’ai créé précédemment.

\sphinxAtStartPar
\sphinxincludegraphics{{schema-form}.jpg}

\sphinxAtStartPar
La table MainForm contient un XboxController() cette classe permet de connaître les manettes connectés au pc et de recevoir leur inputs.

\sphinxAtStartPar
Elle contient aussi une liste de form, ces formes sont les différentes fenêtres de l’application (si possible j’aimerai faire que seul une fenêtre soit active).

\sphinxAtStartPar
Je fais des test avec les usercontrols


\subsection{10h00}
\label{\detokenize{logbook:id62}}
\sphinxAtStartPar
je n’arrive pas a afficher dynamiquement quand une action est faite.


\subsection{12h00}
\label{\detokenize{logbook:id63}}
\sphinxAtStartPar
Mon problème venait du fait que je n’initialisait pas l “usercontrol.


\subsection{13h00}
\label{\detokenize{logbook:id64}}
\sphinxAtStartPar
Je vais essayer de me baser sur la structure des div en html pour gérer le contenu de l’affichage.
\sphinxincludegraphics{{schema-div}.jpeg}


\subsection{14h00}
\label{\detokenize{logbook:id65}}
\sphinxAtStartPar
FInalisation de la configuration du serveur


\bigskip\hrule\bigskip



\section{04.05.2021}
\label{\detokenize{logbook:id66}}

\subsection{8h05}
\label{\detokenize{logbook:id67}}
\sphinxAtStartPar
Je continue à faire des test pour pouvoir bouger le focus d’une “div” a une autre


\subsection{11h00}
\label{\detokenize{logbook:id68}}
\sphinxAtStartPar
Je continue à faire des test mais j’ai apporté des modifications:

\sphinxAtStartPar
le MainForm ne contient pas une liste de list de control, ce n’est pas nécessaire sachant que le lien entre les control est seulement connu des sous contrôle

\sphinxAtStartPar
dorénavant chaque sous control possède une liste de listes de control pour pouvoir se déplacer.

\sphinxAtStartPar
quand on déplace la position maximum dans un sous control il y a deux possibilité.:
un control est disponible dans la direction souhaité
rien n’est disponible
Si un control est disponible alors la main form est informée qu’elle doit changer d’activeForm pour pouvoir se déplacer dans de bonne condition.


\subsection{12h05}
\label{\detokenize{logbook:id69}}
\sphinxAtStartPar
Le déplacement dans chaque form fonctionne et l’on peut passer d’une form à une autre.


\bigskip\hrule\bigskip



\section{05.05.2021}
\label{\detokenize{logbook:id70}}

\subsection{8h05}
\label{\detokenize{logbook:id71}}
\sphinxAtStartPar
Je continue à améliorer le fonctionnement de l’interface.

\sphinxAtStartPar
J’ai décidé de diviser l’interface en 3 “partie”
la navbar
la sidebar
le main contenu

\sphinxAtStartPar
\sphinxincludegraphics{{schema-home-page}.jpeg}


\subsection{10h50}
\label{\detokenize{logbook:id72}}
\sphinxAtStartPar
J’ai créé une nouvelle classe ButtonContext.cs, elle contient les paramètres qui doivent être passer a la forme pour quel sache l’action à exécuter.

\sphinxAtStartPar
Il est maintenant possible de cliquer sur un bouton et la manette va reprendre la ou l’utilisateur a cliqué.


\subsection{12h40}
\label{\detokenize{logbook:id73}}
\sphinxAtStartPar
Je permet le déplacement grâce au joystick gauche


\subsection{13h00}
\label{\detokenize{logbook:id74}}
\sphinxAtStartPar
J’essaie de faire en sorte que je puisse revenir en arrière dans les fenétre


\subsection{15h00}
\label{\detokenize{logbook:id75}}
\sphinxAtStartPar
J’ai essayé de sauver dans une liste les choses précédemment affichée mais j’ai des soucis avec les liens entre les différents panel.

\sphinxAtStartPar
Je vais essayer de recréer les anciens panel a chaque fois au lieu de les recharger.


\bigskip\hrule\bigskip



\section{06.05.2021}
\label{\detokenize{logbook:id76}}

\subsection{8h05}
\label{\detokenize{logbook:id77}}
\sphinxAtStartPar
Je continue a faire en sorte que je puisse retourner en arrière grâce à la touche “B”


\subsection{9h00}
\label{\detokenize{logbook:id78}}
\sphinxAtStartPar
Il est maintenant possible de revenir en arrière dans la navigation


\subsection{10h00}
\label{\detokenize{logbook:id79}}
\sphinxAtStartPar
si il y a des “trou” dans la navigation le curseur le contourne


\subsection{10h40}
\label{\detokenize{logbook:id80}}
\sphinxAtStartPar
J’essaie d’afficher des images dans l’application


\subsection{13h00}
\label{\detokenize{logbook:id81}}
\sphinxAtStartPar
J’ai discuté avec Mr Maréchal de mon git. Pour pouvoir appliquer correctement mon gitignore j’ai dus supprimer les fichier du git


\subsection{14h00}
\label{\detokenize{logbook:id82}}
\sphinxAtStartPar
J’ai fais des recherche sur la publication de projet et j’ai corrigé des bugs


\bigskip\hrule\bigskip



\section{07.05.2021}
\label{\detokenize{logbook:id83}}

\subsection{8h05}
\label{\detokenize{logbook:id84}}
\sphinxAtStartPar
J’ai essayé le paquet “Microsoft Visual Studio Installer Projects” que m’a conseillé M. Schmid. L’installation marche mais certains dossiers de PCSX2 et de Dolphin ne sont pas inclus dans l’installation.


\subsection{10h00}
\label{\detokenize{logbook:id85}}
\sphinxAtStartPar
Je commente les classe que j’ai créé et je supprimer les fonctions qui ne sont pas utilisé


\subsection{12h40}
\label{\detokenize{logbook:id86}}
\sphinxAtStartPar
documentation + Création d’une release pour pouvoir essayer l’interface

\sphinxAtStartPar
Je déplace la documentation du projet web au projet desktop la seul documentation qui reste dans le projet web est celle qui concern son propre code (doxygen)


\subsection{notes personnelles}
\label{\detokenize{logbook:id87}}
\sphinxAtStartPar
Je dois modifier la connexion a la base de données  pour passer des fonctions de cryptographie de PHP au mds. Le problème ce que je ne peux pas me connecter depuis le c\# avec les fonctions de cryptographie de php.


\subsection{15h30}
\label{\detokenize{logbook:id88}}
\sphinxAtStartPar
J’ai mis à jour ma documentation du projet.

\sphinxAtStartPar
J’ai essayé d’ajouter des tâches à mon git mais j’ai eu des soucis pour la signature du projet donc je remet ça à plus tard.


\bigskip\hrule\bigskip



\section{10.05.2021}
\label{\detokenize{logbook:id89}}

\subsection{8h05}
\label{\detokenize{logbook:id90}}
\sphinxAtStartPar
Je modifie la création de mot de passe et la connexion pour qu’un utilisateur puisse se connecter depuis le site web et Caiman.

\sphinxAtStartPar
J’utilisais les fonctions de php (password\_hash et password\_verify) mais je passe a du md5 + salt. La raison est que je ne pouvais pas utiliser ces fonctions pour me connecter depuis l’application c\#.


\subsection{10h10}
\label{\detokenize{logbook:id91}}
\sphinxAtStartPar
je vais essayer de créer un login en C\#

\sphinxAtStartPar
Pour ce faire, je commence par créer une classe “AccessDatabase.cs” pour communiquer avec la base. Pour stocker la la connexionString je l’ai mise dans les settings de l’application.


\subsection{10h45}
\label{\detokenize{logbook:id92}}
\sphinxAtStartPar
Pour pouvoir me connecter a la base de donnée je dois passer par le port 1433 mais il est fermé dans le firewall du coup je dois essayer de me connecter par le port 433


\subsection{13h30}
\label{\detokenize{logbook:id93}}
\sphinxAtStartPar
Je commence à créer la “vrai” interface je commence par les menu de configurations et le menu pour quitter l’application.


\subsection{14h00}
\label{\detokenize{logbook:id94}}
\sphinxAtStartPar
Après une discussion avec M. Scmid j’ai décidé de faire une api.


\subsection{14h30}
\label{\detokenize{logbook:id95}}
\sphinxAtStartPar
J’ai demandé à M.Borel de l’aide pour la structure de mon api, il a pu m’indiquer une structure correcte mais elle est très verbeuse donc je vais surement avoir pour minimum 3 jours a faire mon API.


\bigskip\hrule\bigskip



\section{11.05.2021}
\label{\detokenize{logbook:id96}}

\subsection{8h00}
\label{\detokenize{logbook:id97}}
\sphinxAtStartPar
Je continue mon api, je commence par essayer la structure de M.Borel sur une seule table pour essayer puis je vais faire les autres.


\subsection{11h40}
\label{\detokenize{logbook:id98}}
\sphinxAtStartPar
Je peux maintenant faire des requêtes mais j’ai un souci avec les headers.
Le hearder Autorization n’est pas correctement recu
Je pense que vu le temps que faire une API prend je fais faire en sorte que l’API ne soit que faite pour Caiman. Je vais donc seulement créer les requêtes nécessaires


\subsection{13h00}
\label{\detokenize{logbook:id99}}
\sphinxAtStartPar
Je vais lister les requêtes qui me seront potentiellement utile pour Caiman

\sphinxAtStartPar
liste des jeux
recherche de jeu
affichage des informations d’un jeu
recherche des informations d’un utilisateur
recherche des jeux avec un nombre qui ont été joué par un utilisateur particulier
recherche d’un jeu selon sa catégorie
jeux favoris d’un utilisateur
recherche des jeux qui ont été joué
connexion d’un utilisateur
Création de compte
réception d’un fichier de sauvegarde
réception d’un fichier de configuration


\subsection{13h40}
\label{\detokenize{logbook:id100}}
\sphinxAtStartPar
Je fais un test d’appel a l’API depuis Caiman


\subsection{14h20}
\label{\detokenize{logbook:id101}}
\sphinxAtStartPar
Pour pouvoir utiliser correctement les appels à l’api je dois utiliser des objets que je je rempli avec chaque appel.

\sphinxAtStartPar
Par exemple, si je veux recevoir les informations d’un jeu je créer un objet jeu a qui je vais attribuer les données que je viens de recevoir.

\sphinxAtStartPar
Il me faut donc créer une classe pour l’utilisateur et une classe pour les jeux.


\subsection{15h50}
\label{\detokenize{logbook:id102}}
\sphinxAtStartPar
je dois aussi pouvoir connaître la liste des catégories grâce à l’API


\subsection{16h00}
\label{\detokenize{logbook:id103}}
\sphinxAtStartPar
Il est maintenant possible de récupérer les catégories


\bigskip\hrule\bigskip



\section{12.05.2021}
\label{\detokenize{logbook:id104}}

\subsection{8h05}
\label{\detokenize{logbook:id105}}
\sphinxAtStartPar
Documentation et suppression de code inutile


\subsection{13h00}
\label{\detokenize{logbook:id106}}
\sphinxAtStartPar
J’ai discuté avec M.Smid il m’a donner des conseil sur pour mon api


\subsection{14h30}
\label{\detokenize{logbook:id107}}
\sphinxAtStartPar
J’ai changé les routes de mon api
exemple:
/games/userFavorites/8  => /games/?byUserFavorites=8


\subsection{15h30}
\label{\detokenize{logbook:id108}}
\sphinxAtStartPar
J’ai essayé de mettre l’api en ligne mais quand j’arrive sur une page j’ai une erreur 500.


\bigskip\hrule\bigskip



\section{13.05.2021}
\label{\detokenize{logbook:id109}}

\subsection{11h00}
\label{\detokenize{logbook:id110}}
\sphinxAtStartPar
L’erreur 500 que j’avais était liée à une version de PHP, le serveur avait une version de php 7.3 alors que mon API a besoin d’une version PHP minimum en 7.4.
J’ai donc mis à jour la version présente sur le serveur.


\subsection{12h00}
\label{\detokenize{logbook:id111}}
\sphinxAtStartPar
J’ai corrigé différentes erreurs liées à l’api par exemple quand on envoyait que le username et pas de password pour se connecter une erreur apparaissait.


\subsection{13h00}
\label{\detokenize{logbook:id112}}
\sphinxAtStartPar
J’ai une erreur sur le serveur quand je veux faire une requête d’utilisateur avec sa clé d’API j’ai une erreur 404.

\sphinxAtStartPar
Je vais peut etre passer la recherche d’api en POST et non en GET


\section{14.05.2021}
\label{\detokenize{logbook:id113}}

\subsection{8h00}
\label{\detokenize{logbook:id114}}
\sphinxAtStartPar
J’ai toujours une erreur quand je veux récupérer les informations d’un utilisateur grâce à son apitoken, l’erreur n’est présente que sur le caiman.cfpt.info.
Pour corriger l’erreur je vais passer par du post pour éviter de perdre trop de temps.


\subsection{09h22}
\label{\detokenize{logbook:h22}}
\sphinxAtStartPar
j’ai push sur le serveur une version de l’API qui fonctionne bien, je vais maintenant continuer ma documentation


\subsection{09h30}
\label{\detokenize{logbook:id115}}
\sphinxAtStartPar
je me rend compte qu j’ai une requête qui ne marche pas


\subsection{11h30}
\label{\detokenize{logbook:id116}}
\sphinxAtStartPar
J’ai dû modifier le .htaccess pour pouvoir appliquer les rewriteRules. Le fichier n’était pas pris en compte sur le serveur je l’ai donc mis dans le dossier ou point le virtual host.


\subsection{12h00}
\label{\detokenize{logbook:id117}}
\sphinxAtStartPar
documentation


\bigskip\hrule\bigskip



\section{17.05.2021}
\label{\detokenize{logbook:id118}}

\subsection{8h05}
\label{\detokenize{logbook:id119}}
\sphinxAtStartPar
Création d’une ébauche de diagram de classe sur le tableau.

\sphinxAtStartPar
J’ai repris en grande partie le diagramme de mon POC en y ajoutant une gestion des sauvegardes et des téléchagement.


\subsection{9h00}
\label{\detokenize{logbook:id120}}
\sphinxAtStartPar
Création du schéma sous UMLetino

\sphinxAtStartPar
Première version du diagramme de la logique de Caiman.


\subsection{13h20}
\label{\detokenize{logbook:id121}}
\sphinxAtStartPar
J’essaie de réfléchir à la manière de télécharger les sauvegardes et la façon dont je dois stocker les jeux que l’utilisateur a téléchargé.

\sphinxAtStartPar
Je pense que je vais faire en sorte de vérifier si les jeux qui devraient être présents sur le disque de l’utilisateur le sont réellement.

\sphinxAtStartPar
Je vais faire en sorte que l’utilisateur ait la possibilité de pouvoir télécharger les jeux dans un dossier spécifique. Il va devoir au premier lancement de l’application spécifier ou l’installation doit se faire. (c’est peut etre contraire a mon but de faire une application simple d’utilisation).

\sphinxAtStartPar
Je dois aussi savoir si le disque de l’utilisateur n’a pas la place requise pour télécharger le jeu demandé.


\subsection{14h35}
\label{\detokenize{logbook:id122}}
\sphinxAtStartPar
Discussion avec M.Maréchal


\section{18.05.2021}
\label{\detokenize{logbook:id123}}

\subsection{8h05}
\label{\detokenize{logbook:id124}}
\sphinxAtStartPar
Je vais créer une page de connexion pour l’utilisateur


\subsection{10h00}
\label{\detokenize{logbook:id125}}
\sphinxAtStartPar
La connection marche, je vais maintenant télécharger les images des jeux


\subsection{12h00}
\label{\detokenize{logbook:id126}}
\sphinxAtStartPar
Je commence a faire en sorte que quand un nouveau jeu est reçu depuis l’API son image est automatiquement télécharger et mise dans le dossier \%appdata\% de l’application


\subsection{13h30}
\label{\detokenize{logbook:id127}}
\sphinxAtStartPar
J’affiche dans la sidebar le nom des catégories


\subsection{15h00}
\label{\detokenize{logbook:id128}}
\sphinxAtStartPar
Quand je clique sur une catégorie les jeux de la catégorie choisie sont afficher.


\bigskip\hrule\bigskip



\section{19.05.2021}
\label{\detokenize{logbook:id129}}

\subsection{8h05}
\label{\detokenize{logbook:id130}}
\sphinxAtStartPar
je fais en sorte d’afficher tous les jeux quand je lance l’application.


\subsection{9h00}
\label{\detokenize{logbook:id131}}
\sphinxAtStartPar
je vais maintenant faire en sorte de pouvoir voir les détails d’un jeu.


\subsection{9h30}
\label{\detokenize{logbook:id132}}
\sphinxAtStartPar
J’ai un problème avec les image bouton le clic n’est pas pris en compte


\subsection{11h00}
\label{\detokenize{logbook:id133}}
\sphinxAtStartPar
j’ai corriger le soucis des boutons et je modifie un petit peu l’interface


\subsection{13h00}
\label{\detokenize{logbook:id134}}
\sphinxAtStartPar
je commence l’affichage des détails d’un jeu


\subsection{14h00}
\label{\detokenize{logbook:id135}}
\sphinxAtStartPar
je fais des recherche sur la façon de télécharger un jeu


\bigskip\hrule\bigskip



\section{20.05.2021}
\label{\detokenize{logbook:id136}}

\subsection{8h05}
\label{\detokenize{logbook:id137}}
\sphinxAtStartPar
Je crée une route pour pouvoir recevoir le lien de téléchargement d’un jeu.


\subsection{09h15}
\label{\detokenize{logbook:id138}}
\sphinxAtStartPar
Je me rend compte que pour créer la route de téléchargement je dois:

\sphinxAtStartPar
Avoir le lien du site
Avoir le lien pour la console
avoir le nom du fichier

\sphinxAtStartPar
Pour avec ces différents éléments je dois modifier mon API pour pouvoir faire tout ça.


\subsection{11h00}
\label{\detokenize{logbook:id139}}
\sphinxAtStartPar
J’arrive à créer l’URL mais le fichier n’est pas accessible en dehors du serveur donc je ne sais pas vraiment quoi faire


\subsection{12h50}
\label{\detokenize{logbook:id140}}
\sphinxAtStartPar
Correction d’un problème où l’on pouvait se déplacer dans une case qui n’existe pas donc l’application plantait.


\subsection{14h00}
\label{\detokenize{logbook:id141}}
\sphinxAtStartPar
Je commence a faire en sorte que je puisse créer la route pour télécharger le fichier des jeux


\bigskip\hrule\bigskip



\section{21.05.2021}
\label{\detokenize{logbook:id142}}

\subsection{9h00}
\label{\detokenize{logbook:id143}}
\sphinxAtStartPar
Je voulais télécharger un jeu en passant par le webClient en C\# mais ma route est en POST dans mon API. Malheureusement je ne peux pas fournir de paramètres en POST avec la fonction downloadFileAsync donc je dois passer ma route en GET


\subsection{11h00}
\label{\detokenize{logbook:id144}}
\sphinxAtStartPar
j’ai changé ma route et j’en ai ajouté pour pouvoir connaître les dossiers ou je dois stocker les jeux et le nom du fichier d’un jeu.


\subsection{16h00}
\label{\detokenize{logbook:id145}}
\sphinxAtStartPar
J’ai réussi à afficher la liste des téléchargements  et maintenant  je suis en train de faire en sorte de pouvoir ajouter un jeu au favoris. j’ai réglé des soucis de droit d’accès au fichier qui sont en cours de lecture.


\bigskip\hrule\bigskip



\section{24.05.2021}
\label{\detokenize{logbook:id146}}

\subsection{15h15}
\label{\detokenize{logbook:id147}}
\sphinxAtStartPar
les téléchargement se font maintenant l’un après l’autre


\subsection{16h35}
\label{\detokenize{logbook:id148}}
\sphinxAtStartPar
Affichage des jeux téléchargé


\bigskip\hrule\bigskip



\section{25.05.2021}
\label{\detokenize{logbook:id149}}

\subsection{8h10}
\label{\detokenize{logbook:id150}}
\sphinxAtStartPar
je dois modifier l’API pour pouvoir manipuler les jeux favoris des utilisateur


\subsection{10h25}
\label{\detokenize{logbook:h25}}
\sphinxAtStartPar
il est maintenant possible d’ajouter et de supprimer un jeu des favoris.
je vais maintenant faire en sorte de pouvoir exécuter des jeux depuis caiman.


\subsection{10h30}
\label{\detokenize{logbook:id151}}
\sphinxAtStartPar
Je regarde le code que j’ai fais pour le poc pour voir si je peux reprendre des parties de codes.


\subsection{14h00}
\label{\detokenize{logbook:id152}}
\sphinxAtStartPar
J’ai un soucis avec le lancement des jeux de PS2


\subsection{14h25}
\label{\detokenize{logbook:id153}}
\sphinxAtStartPar
Le problème venait des paramètres j’avais oublié une espace entre le nom du fichier à exécuter et les paramètres


\subsection{15h50}
\label{\detokenize{logbook:id154}}
\sphinxAtStartPar
J’ai pus ajouter les émulateurs et appliquer certains paramètres graphique.


\bigskip\hrule\bigskip



\section{26.05.2021}
\label{\detokenize{logbook:id155}}

\subsection{8h05}
\label{\detokenize{logbook:id156}}
\sphinxAtStartPar
Je dois modifier mon api pour pouvoir générer un token à chaque connexion pour ne pas avoir à se connecter à chaque lancement de l’application.


\subsection{10h30}
\label{\detokenize{logbook:id157}}
\sphinxAtStartPar
J’ai modifié l’API, je vais maintenant passer a Caiman.


\subsection{11h00}
\label{\detokenize{logbook:id158}}
\sphinxAtStartPar
Il est maintenant possible de se connecter automatiquement , je vais maintenant modifier un petit peu l’aspect graphique de Caiman.


\subsection{13h00}
\label{\detokenize{logbook:id159}}
\sphinxAtStartPar
je veux afficher le jeux qui est actuellement utilisé par l’utilisateur


\subsection{15h30}
\label{\detokenize{logbook:id160}}
\sphinxAtStartPar
Il est possible de voir le jeu actuellement lancé par l’utilisateur et peut importe comment l’utilisateur quite le jeu Caiman va étre au courant


\subsection{18h00}
\label{\detokenize{logbook:id161}}
\sphinxAtStartPar
Affichage du temps de jeu de la session


\subsection{21h00}
\label{\detokenize{logbook:id162}}
\sphinxAtStartPar
Le nombre de minutes de jeu a comptabilisé


\bigskip\hrule\bigskip



\section{27.05.2021}
\label{\detokenize{logbook:id163}}

\subsection{8h10}
\label{\detokenize{logbook:id164}}
\sphinxAtStartPar
J’ai remarqué que je ne pouvais pas upload de jeu actuellement je vais essayer de comprendre pouquoi.


\subsection{09h00}
\label{\detokenize{logbook:id165}}
\sphinxAtStartPar
J’ai résolu le problème je devais ajouter les droit d’écriture dans mon debian
Par contre hier soir j’ai oublié de push donc je ne peux pas continuer Caiman ce matin


\subsection{10h05}
\label{\detokenize{logbook:id166}}
\sphinxAtStartPar
Je profite de ne pas pouvoir coder pour faire des tests, ajouter des jeux et faire de la doc


\subsection{23h30}
\label{\detokenize{logbook:id167}}
\sphinxAtStartPar
J’ai réussi à scanner un dossier pour savoir si il a y eu une modification.


\bigskip\hrule\bigskip



\section{28.05.2021}
\label{\detokenize{logbook:id168}}

\subsection{9h30}
\label{\detokenize{logbook:id169}}
\sphinxAtStartPar
j’ai reussi a faire en sorte que les sauvegardes de l’utilisateur qui se connecte sont déplacé dans le dossier des sauvegardes des émulateurs.


\subsection{10h00}
\label{\detokenize{logbook:id170}}
\sphinxAtStartPar
Je réfléchi à comment faire pour synchroniser les sauvegardes des utilisateurs. J’ai eu l’idée de créer un sous répertoire pour chaque utilisateur sur le serveur

\sphinxAtStartPar
\#\#\#10h30

\sphinxAtStartPar
J’ai réfléchi et finalement je pense envoyer des fichier zip contenant toutes les sauvegardes d’un utilisateur et cela pour chaque console. Cela a l’avantage de réduire la taille des fichier et de simplifier l’upload et le téléchargement


\subsection{14h00}
\label{\detokenize{logbook:id171}}
\sphinxAtStartPar
J’ai eu un bogue ou les sauvegarde n’était pas envoyé dans le bon dossier


\subsection{14h35}
\label{\detokenize{logbook:id172}}
\sphinxAtStartPar
Documentation


\bigskip\hrule\bigskip



\section{30.05.2021}
\label{\detokenize{logbook:id173}}

\subsection{13h30}
\label{\detokenize{logbook:id174}}
\sphinxAtStartPar
Modification de l’API pour pouvoir télécharger un fichier de sauvegarde


\subsection{16h12}
\label{\detokenize{logbook:h12}}
\sphinxAtStartPar
Je fais en sorte que les sauvegardes se synchronisent sur caiman


\subsection{19h43}
\label{\detokenize{logbook:h43}}
\sphinxAtStartPar
j’ai ENFIN réussi à synchroniser les sauvegardes de Gamecube mais j’ai un soucis avec celle de ps2


\bigskip\hrule\bigskip



\section{31.05.2021}
\label{\detokenize{logbook:id175}}

\subsection{8h05}
\label{\detokenize{logbook:id176}}
\sphinxAtStartPar
Je commente le code de Caiman


\subsection{14h00}
\label{\detokenize{logbook:id177}}
\sphinxAtStartPar
Discussion avec M.Maréchal et ajout de tests pour la classe GameTimer


\bigskip\hrule\bigskip



\section{01.06.2021}
\label{\detokenize{logbook:id178}}

\subsection{8h05}
\label{\detokenize{logbook:id179}}
\sphinxAtStartPar
Ajout de tests pour la classe TimeInGame


\subsection{8h20}
\label{\detokenize{logbook:id180}}
\sphinxAtStartPar
Je veux faire en sorte que Caiman s’exécute sur l’écran principale du pc de l’utilisateur


\subsection{8h30}
\label{\detokenize{logbook:id181}}
\sphinxAtStartPar
J’ai désactiver le .gitIgnore par ce que cela créait des soucis avec les dossier des émulateurs


\bigskip\hrule\bigskip



\section{02.06.2021}
\label{\detokenize{logbook:id182}}

\subsection{8h05}
\label{\detokenize{logbook:id183}}
\sphinxAtStartPar
Commentaire du code de l’API


\subsection{10h30}
\label{\detokenize{logbook:id184}}
\sphinxAtStartPar
documentation


\subsection{15h30}
\label{\detokenize{logbook:id185}}
\sphinxAtStartPar
J’ai corrigé le fait que un compte pouvait être créé sans remplir tous les champ


\bigskip\hrule\bigskip



\section{03.06.2021}
\label{\detokenize{logbook:id186}}

\subsection{8h10}
\label{\detokenize{logbook:id187}}
\sphinxAtStartPar
J’ai eu un problème de conflit avec mon git j’ai du re télécharger tout le projet mais maintenant c’est bon.


\subsection{8h34}
\label{\detokenize{logbook:h34}}
\sphinxAtStartPar
Je continue la documentation


\bigskip\hrule\bigskip



\section{04.06.2021}
\label{\detokenize{logbook:id188}}

\subsection{8h05}
\label{\detokenize{logbook:id189}}
\sphinxAtStartPar
Documentation


\subsection{13h15}
\label{\detokenize{logbook:id190}}
\sphinxAtStartPar
La documentation avance bien je vais donc créer les diagrammes de classe pour le C\#.


\bigskip\hrule\bigskip



\section{07.06.2021}
\label{\detokenize{logbook:id191}}

\subsection{8h10}
\label{\detokenize{logbook:id192}}
\sphinxAtStartPar
Recherche sur comment exporter ma documentation en pdf et je continue ma doc.


\subsection{11h20}
\label{\detokenize{logbook:id193}}
\sphinxAtStartPar
Je vais essayer d’installer un certificat ssl sur le serveur debian


\subsection{13h30}
\label{\detokenize{logbook:id194}}
\sphinxAtStartPar
Je n’arrive pas a installer certbot donc je laisse ca en attente


\subsection{14h00}
\label{\detokenize{logbook:id195}}
\sphinxAtStartPar
Documentation



\renewcommand{\indexname}{Index}
\printindex
\end{document}